% \documentclass[a4paper,10pt]{article}
% \usepackage[utf8]{inputenc}
% \usepackage{geometry}
% \usepackage[table]{xcolor}
% \usepackage{colortbl}
% \usepackage{color,soul}
% \geometry{margin=0.8in}
% \usepackage{xcolor}
% \usepackage{tikz}
% \usepackage{minted}
% \definecolor{bgcolor}{rgb}{0.8, 0.9, 0.5} % 
% \definecolor{bgcolor1}{rgb}{0.95, 0.95, 0.95} % Light Gray
% \definecolor{bgcolor2}{rgb}{0.85, 0.92, 1.0}  % Soft Blue
% \definecolor{bgcolor3}{rgb}{0.9, 0.85, 1.0}   % Light Purple
% \definecolor{bgcolor4}{rgb}{0.95, 0.88, 0.76} % Warm Beige
% \definecolor{bgcolor5}{rgb}{0.8, 0.95, 0.8}   % Gentle Green
% \definecolor{bgcolor6}{rgb}{1.0, 0.87, 0.87}  % Pastel Red
% \definecolor{bgcolor7}{rgb}{0.86, 0.93, 0.83} % Mint Green
% \definecolor{bgcolor8}{rgb}{0.98, 0.85, 0.94} % Soft Pink
% \definecolor{bgcolor9}{rgb}{0.87, 0.94, 0.98} % Sky Blue
% \definecolor{bgcolor10}{rgb}{0.96, 0.96, 0.82} % Pale Yellow
% 
% % Define a custom background colo
% \begin{document}
\section*{Matrix Problem Solutions}
\noindent\textbf{Problem: Matrix Rotation (90° Clockwise)}
\begin{minted}[
bgcolor=bgcolor5,
frame=lines,
framesep=5mm,
rulecolor=\color{black},
linenos,
numbersep=5pt,
fontsize=\normalsize
]{python}
from typing import List


def rotate_matrix(matrix: List[List[int]]) -> None:
    """
    Rotates an n x n matrix 90 degrees clockwise in-place.
    Time Complexity: O(n^2)
    """
    n = len(matrix)
    # Transpose the matrix
    for i in range(n):
        for j in range(i, n):
            matrix[i][j], matrix[j][i] = matrix[j][i], matrix[i][j]
    # Reverse each row
    for i in range(n):
        matrix[i].reverse()
\end{minted}

\noindent\textbf{Problem: Spiral Traversal of Matrix}
\begin{minted}[
bgcolor=bgcolor1,
frame=lines,
framesep=5mm,
rulecolor=\color{black},
linenos,
numbersep=5pt,
fontsize=\normalsize
]{python}
from typing import List


def spiral_order(matrix: List[List[int]]) -> List[int]:
    """
    Traverses a matrix in spiral order.
    Time Complexity: O(n * m)
    """
    if not matrix or not matrix:
        return []

    m, n = len(matrix), len(matrix)
    result = []
    top, bottom, left, right = 0, m - 1, 0, n - 1

    while top <= bottom and left <= right:
        # Traverse right
        for col in range(left, right + 1):
            result.append(matrix[top][col])
        top += 1

        # Traverse down
        for row in range(top, bottom + 1):
            result.append(matrix[row][right])
        right -= 1

        # Traverse left
        if top <= bottom:
            for col in range(right, left - 1, -1):
                result.append(matrix[bottom][col])
            bottom -= 1

        # Traverse up
        if left <= right:
            for row in range(bottom, top - 1, -1):
                result.append(matrix[row][left])
            left += 1
    return result
\end{minted}

\noindent\textbf{Problem: Median of Row Wise Sorted Matrix}
\begin{minted}[
bgcolor=bgcolor3,
frame=lines,
framesep=5mm,
rulecolor=\color{black},
linenos,
numbersep=5pt,
fontsize=\normalsize
]{python}
def median_row_wise_sorted_matrix(matrix: List[List[int]]) -> int:
    """
    Finds the median of a row-wise sorted matrix.
    Time Complexity: O(r * log(max - min) * log(c))
    Note: Requires all rows to be sorted and finds median of all elements combined.
    """
    rows = len(matrix)
    cols = len(matrix)

    min_val = float('inf')
    max_val = float('-inf')

    for r in range(rows):
        min_val = min(min_val, matrix[r][0])
        max_val = max(max_val, matrix[r][cols - 1])

    desired_count = (rows * cols + 1) // 2

    while min_val <= max_val:
        mid_val = (min_val + max_val) // 2
        count = 0 #Count how many entries <= mid across all rows
        for r in range(rows):
            count += bisect.bisect_right(matrix[r], mid_val)
        # If fewer than desired are <= mid, median must be bigger
        if count < desired_count:
            min_val = mid_val + 1
        else:
            max_val = mid_val - 1
    # lo == hi is the median value
    return min_val
\end{minted}

\noindent\textbf{Problem: Search in Row-wise and Column-wise Sorted Matrix}
\begin{minted}[
bgcolor=bgcolor7,
frame=lines,
framesep=5mm,
rulecolor=\color{black},
linenos,
numbersep=5pt,
fontsize=\normalsize
]{python}
from typing import List


def search_matrix_rc_sorted(matrix: List[List[int]], target: int) -> bool:
    """
    Searches for a target in a matrix sorted row-wise and column-wise.
    Time Complexity: O(n + m)
    """
    if not matrix or not matrix:
        return False

    rows = len(matrix)
    cols = len(matrix)
    row, col = 0, cols - 1 # Start from top-right corner

    while row < rows and col >= 0:
        if matrix[row][col] == target:
            return True
        elif matrix[row][col] < target:
            row += 1 # Target is larger, move down
        else:
            col -= 1 # Target is smaller, move left
    return False
\end{minted}

\noindent\textbf{Problem: Determinant of a Matrix}
\begin{minted}[
bgcolor=bgcolor10,
frame=lines,
framesep=5mm,
rulecolor=\color{black},
linenos,
numbersep=5pt,
fontsize=\normalsize
]{python}
from typing import List


def _get_cofactor(matrix: List[List[int]], p: int, q: int, n: int) -> List[List[int]]:
    """Helper to get cofactor matrix for determinant calculation."""
    temp = []
    for row in range(n):
        if row != p:
            new_row = []
            for col in range(n):
                if col != q:
                    new_row.append(matrix[row][col])
            if new_row:
                temp.append(new_row)
    return temp


def determinant_of_matrix(matrix: List[List[int]]) -> int:
    """
    Computes the determinant of a square matrix using recursive cofactor expansion.
    Time Complexity: O(n!) for this naive recursive approach.
    The source notes O(n^3) for optimized methods like Gaussian Elimination.
    """
    n = len(matrix)
    if n == 1:
        return matrix[0][0]
    if n == 2:
        return matrix[0][0] * matrix[1][1] - matrix[0][1] * matrix[1][0]

    det = 0
    sign = 1
    for f in range(n):
        cofactor = _get_cofactor(matrix, 0, f, n)
        det += sign * matrix[0][f] * determinant_of_matrix(cofactor)
        sign = -sign
    return det

    ###################################GAUSSIAN ELIMINATION##############################
from typing import List

def determinant_gaussian(matrix: List[List[float]]) -> float:
    """
    Compute the determinant of a square matrix using
    Gaussian elimination with partial pivoting in O(n^3).
    """
    n = len(matrix)
    # Make a working copy so we don't clobber the input
    A = [row.copy() for row in matrix]
    det = 1.0

    for i in range(n):
        # 1) Partial pivot: find the row with largest abs value in column i
        pivot = max(range(i, n), key=lambda r: abs(A[r][i]))
        if abs(A[pivot][i]) < 1e-12:
            return 0.0  # singular matrix

        # 2) Swap current row with pivot row (if needed)
        if pivot != i:
            A[i], A[pivot] = A[pivot], A[i]
            det = -det   # swapping rows flips det’s sign

        # 3) Multiply det by the pivot element
        det *= A[i][i]

        # 4) Eliminate below: make all entries A[j][i] zero for j>i
        for j in range(i+1, n):
            factor = A[j][i] / A[i][i]
            # subtract factor * pivot row from row j
            for k in range(i, n):
                A[j][k] -= factor * A[i][k]

    return det

\end{minted}

\noindent\textbf{Problem: Search in Row-wise Sorted Matrix}
\begin{minted}[
bgcolor=bgcolor8,
frame=lines,
framesep=5mm,
rulecolor=\color{black},
linenos,
numbersep=5pt,
fontsize=\normalsize
]{python}
from typing import List

def search_row_sorted_matrix(matrix: List[List[int]], target: int) -> bool:
    """
    Search in a matrix where:
      - each row is sorted left→right, and
      - row[i][0] >= row[i-1][-1] (so the whole matrix is one big sorted list).
    Time: O(log(rows * cols)), Space: O(1)
    """
    if not matrix or not matrix[0]:
        return False

    rows, cols = len(matrix), len(matrix[0])
    left, right = 0, rows * cols - 1

    while left <= right:
        mid = (left + right) // 2
        r, c = divmod(mid, cols)
        val = matrix[r][c]

        if val == target:
            return True
        elif val < target:
            left = mid + 1
        else:
            right = mid - 1
    return False
\end{minted}

\noindent\textbf{Problem: Peak Element in 2D Matrix}
\begin{minted}[
bgcolor=bgcolor9,
frame=lines,
framesep=5mm,
rulecolor=\color{black},
linenos,
numbersep=5pt,
fontsize=\normalsize
]{python}
from typing import List


def find_peak_element_2d(matrix: List[List[int]]) -> List[int]:
    """
    Finds a peak element in a 2D matrix (where matrix[i][j] > all its neighbors).
    Time Complexity: O(n log m)
    Note: Returns coordinates [row, col] of one such peak.
    """
    if not matrix or not matrix:
        return []

    rows = len(matrix)
    cols = len(matrix)
    low_col, high_col = 0, cols - 1

    while low_col <= high_col:
        mid_col = low_col + (high_col - low_col) // 2
        max_row = 0
        for r in range(rows):
            if matrix[r][mid_col] > matrix[max_row][mid_col]:
                max_row = r

        is_left_greater = False
        if mid_col > 0 and matrix[max_row][mid_col - 1] > matrix[max_row][mid_col]:
            is_left_greater = True

        if not is_left_greater and 
           (mid_col == cols-1 or matrix[max_row][mid_col + 1] < matrix[max_row][mid_col]):
            return [max_row, mid_col] # Found a peak

        elif is_left_greater:
            high_col = mid_col - 1
        else:
            low_col = mid_col + 1
    return [] # Should not reach here if a peak always exists as per problem definition
\end{minted}
% \end{document}
