% \documentclass[a4paper,10pt]{article}
% \usepackage[utf8]{inputenc}
% \usepackage{geometry}
% \usepackage[table]{xcolor}
% \usepackage{colortbl}
% \usepackage{color,soul}
% \geometry{margin=0.8in}
% \usepackage{xcolor}
% \usepackage{tikz}
% \usepackage{minted}
% \usepackage{standalone}
% \definecolor{bgcolor}{rgb}{0.8, 0.9, 0.5} % 
% \definecolor{bgcolor1}{rgb}{0.95, 0.95, 0.95} % Light Gray
% \definecolor{bgcolor2}{rgb}{0.85, 0.92, 1.0}  % Soft Blue
% \definecolor{bgcolor3}{rgb}{0.9, 0.85, 1.0}   % Light Purple
% \definecolor{bgcolor4}{rgb}{0.95, 0.88, 0.76} % Warm Beige
% \definecolor{bgcolor5}{rgb}{0.8, 0.95, 0.8}   % Gentle Green
% \definecolor{bgcolor6}{rgb}{1.0, 0.87, 0.87}  % Pastel Red
% \definecolor{bgcolor7}{rgb}{0.86, 0.93, 0.83} % Mint Green
% \definecolor{bgcolor8}{rgb}{0.98, 0.85, 0.94} % Soft Pink
% \definecolor{bgcolor9}{rgb}{0.87, 0.94, 0.98} % Sky Blue
% \definecolor{bgcolor10}{rgb}{0.96, 0.96, 0.82} % Pale Yellow
% 
% % Define a custom background colo
% \begin{document}
\section*{Mathematics Problem Solutions}

\noindent\textbf{Problem: Palindrome Number}
\begin{minted}[
    bgcolor=bgcolor7,
    frame=lines,
    framesep=5mm,
    rulecolor=\color{black},
    linenos,
    numbersep=5pt,
    fontsize=\normalsize
]{python}
def is_palindrome_number(n: int) -> bool:
    """
    Check if n reads the same forward and backward.
    Time Complexity: O(log10 N)
    """
    if n < 0 or (n % 10 == 0 and n != 0):
        return False
    rev_half = 0
    while n > rev_half:
        rev_half = rev_half * 10 + n % 10
        n //= 10
    return n == rev_half or n == rev_half // 10
\end{minted}
\noindent\textbf{Problem: Factorial of a Number}
\begin{minted}[
    bgcolor=bgcolor2,
    frame=lines,
    framesep=5mm,
    rulecolor=\color{black},
    linenos,
    numbersep=5pt,
    fontsize=\normalsize
]{python}
def factorial(n: int, memo: dict[int,int]=None) -> int:
    """
    Compute n! iteratively or via memoization.
    Time Complexity: O(N)
    """
    if memo is None:
        memo = {}
    if n < 2:
        return 1
    if n in memo:
        return memo[n]
    result = 1
    for i in range(2, n + 1):
        result *= i
    memo[n] = result
    return result
\end{minted}
\noindent\textbf{Problem: Trailing Zeros in Factorial}
\begin{minted}[
    bgcolor=bgcolor9,
    frame=lines,
    framesep=5mm,
    rulecolor=\color{black},
    linenos,
    numbersep=5pt,
    fontsize=\normalsize
]{python}
def trailing_zeros(n: int) -> int:
    """
    Count how many trailing zeros n! has by counting factors of 5.
    Time Complexity: O(log_5 N)"""
    count = 0
    while n >= 5:
        n //= 5
        count += n
    return count
\end{minted}
\noindent\textbf{Problem: GCD / HCF}
\begin{minted}[
    bgcolor=bgcolor5,
    frame=lines,
    framesep=5mm,
    rulecolor=\color{black},
    linenos,
    numbersep=5pt,
    fontsize=\normalsize
]{python}
def gcd(a: int, b: int) -> int:
    """
    Compute the greatest common divisor using Euclid’s algorithm.
    Time Complexity: O(log min(a, b))
    """
    a, b = abs(a), abs(b)
    while b:
        a, b = b, a % b
    return a
\end{minted}
\noindent\textbf{Problem: Check for Prime}
\begin{minted}[
  bgcolor=bgcolor6,
  frame=lines,
  framesep=5mm,
  rulecolor=\color{black},
  linenos,
  numbersep=5pt,
  fontsize=\normalsize
]{python}
def is_prime(n: int) -> bool:
    """
    Check if n is prime using 6k±1 optimization.
    Time Complexity: O(root(N)) 
    """
    if n < 2:
        return False
    if n in (2, 3):
        return True
    if n % 2 == 0 or n % 3 == 0:
        return False
    i = 5
    while i * i <= n:
        if n % i == 0 or n % (i + 2) == 0:
            return False
        i += 6

    return True
\end{minted}

\noindent\textbf{Problem: Prime Factors}
\begin{minted}[
    bgcolor=bgcolor1,
    frame=lines,
    framesep=5mm,
    rulecolor=\color{black},
    linenos,
    numbersep=5pt,
    fontsize=\normalsize
]{python}
def prime_factors(n: int) -> list[int]:
    """
    Return the list of prime factors of n using 6k±1 optimization.
    Time Complexity: O(root(N))
    """
    factors = []
    # remove all 2s
    while n % 2 == 0:
        factors.append(2)
        n //= 2
    # remove all 3s
    while n % 3 == 0:
        factors.append(3)
        n //= 3
    # now test 6k±1
    i = 5
    while i * i <= n:
        # factor = i
        while n % i == 0:
            factors.append(i)
            n //= i
        # factor = i + 2
        while n % (i + 2) == 0:
            factors.append(i + 2)
            n //= (i + 2)
        i += 6
    # if n is still >1, it’s prime
    if n > 1:
        factors.append(n)
    return factors
\end{minted}
\noindent\textbf{Problem: All Divisors of a Number}
\begin{minted}[
    bgcolor=bgcolor10,
    frame=lines,
    framesep=5mm,
    rulecolor=\color{black},
    linenos,
    numbersep=5pt,
    fontsize=\normalsize
]{python}
def all_divisors(n: int) -> list[int]:
    """
    Return all divisors of n in ascending order.
    Time Complexity: O(root(N))"""
    small, large = [], []
    i = 1
    while i * i <= n:
        if n % i == 0:
            small.append(i)
            if i != n // i:
                large.append(n // i)
        i += 1
    return small + large[::-1]
\end{minted}
\noindent\textbf{Problem: Sieve of Eratosthenes}
\begin{minted}[
    bgcolor=bgcolor3,
    frame=lines,
    framesep=5mm,
    rulecolor=\color{black},
    linenos,
    numbersep=5pt,
    fontsize=\normalsize
]{python}
def sieve(n: int) -> list[bool]:
    """
    Generate a boolean list is_prime[0..n] using the sieve.
    Time Complexity: O(n log log n)
    """
    if n < 2:
        return [False] * (n + 1)
    is_prime = [True] * (n + 1)
    is_prime[0:2] = [False, False]
    for p in range(3, int(n**0.5) + 1, 2):
        if is_prime[p]:
            for multiple in range(p*p, n + 1, 2*p):
                is_prime[multiple] = False
    return is_prime
\end{minted}
\noindent\textbf{Problem: Computing Power (Recursive)}
\begin{minted}[
    bgcolor=bgcolor6,
    frame=lines,
    framesep=5mm,
    rulecolor=\color{black},
    linenos,
    numbersep=5pt,
    fontsize=\normalsize
]{python}
def power_recursive(a: float, b: int) -> float:
    """
    Compute a**b via exponentiation by squaring (recursive).
    Time Complexity: O(log b)    """
    if b == 0:
        return 1
    half = power_recursive(a, b // 2)
    return half * half if b % 2 == 0 else half * half * a
\end{minted}
\noindent\textbf{Problem: Iterative Power}
\begin{minted}[
    bgcolor=bgcolor2,
    frame=lines,
    framesep=5mm,
    rulecolor=\color{black},
    linenos,
    numbersep=5pt,
    fontsize=\normalsize
]{python}
def power_iterative(a: float, b: int) -> float:
    """
    Compute a**b via binary exponentiation (iterative).
    Time Complexity: O(log b)    """
    result = 1.0
    base = a
    exp = b
    while exp > 0:
        if exp & 1:
            result *= base
        base *= base
        exp >>= 1
    return result
\end{minted}
\noindent\textbf{Problem: Modular Arithmetic}
\begin{minted}[
    bgcolor=bgcolor4,
    frame=lines,
    framesep=5mm,
    rulecolor=\color{black},
    linenos,
    numbersep=5pt,
    fontsize=\normalsize
]{python}
def mod_pow(a: int, b: int, m: int) -> int:
    """
    Compute (a**b) mod m using fast exponentiation.
    Time Complexity: O(log b)    """
    result = 1
    base = a % m
    exp = b
    while exp > 0:
        if exp & 1:
            result = (result * base) % m
        base = (base * base) % m
        exp >>= 1
    return result
\end{minted}
% \end{document}
