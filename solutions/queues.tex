% \documentclass[a4paper,10pt]{article}
% \usepackage[utf8]{inputenc}
% \usepackage{geometry}
% \usepackage[table]{xcolor}
% \usepackage{colortbl}
% \usepackage{color,soul}
% \geometry{margin=0.8in}
% \usepackage{xcolor}
% \usepackage{tikz}
% \usepackage{minted}
% \definecolor{bgcolor}{rgb}{0.8, 0.9, 0.5} % 
% \definecolor{bgcolor1}{rgb}{0.95, 0.95, 0.95} % Light Gray
% \definecolor{bgcolor2}{rgb}{0.85, 0.92, 1.0}  % Soft Blue
% \definecolor{bgcolor3}{rgb}{0.9, 0.85, 1.0}   % Light Purple
% \definecolor{bgcolor4}{rgb}{0.95, 0.88, 0.76} % Warm Beige
% \definecolor{bgcolor5}{rgb}{0.8, 0.95, 0.8}   % Gentle Green
% \definecolor{bgcolor6}{rgb}{1.0, 0.87, 0.87}  % Pastel Red
% \definecolor{bgcolor7}{rgb}{0.86, 0.93, 0.83} % Mint Green
% \definecolor{bgcolor8}{rgb}{0.98, 0.85, 0.94} % Soft Pink
% \definecolor{bgcolor9}{rgb}{0.87, 0.94, 0.98} % Sky Blue
% \definecolor{bgcolor10}{rgb}{0.96, 0.96, 0.82} % Pale Yellow
% 
% \begin{document}
% Circular Queue Implementation
\noindent\textbf{Problem: Circular Implementation of Queue}
\begin{minted}[
bgcolor=bgcolor4,
frame=lines,
framesep=5mm,
rulecolor=\color{black},
linenos,
numbersep=5pt,
fontsize=\normalsize
]{python}
class CircularQueue:
    """
    Circular queue implementation with fixed capacity.
    Time Complexity: O(1) for all operations, Space Complexity: O(n)
    """
    def __init__(self, capacity: int):
        self.capacity = capacity
        self.queue = [None] * capacity
        self.front = self.rear = -1
        self.size = 0

    def enqueue(self, value: int) -> bool:
        if self.is_full():
            return False
            
        if self.is_empty():
            self.front = self.rear = 0
        else:
            self.rear = (self.rear + 1) % self.capacity
            
        self.queue[self.rear] = value
        self.size += 1
        return True

    def dequeue(self) -> bool:
        if self.is_empty():
            return False
            
        if self.front == self.rear:
            self.front = self.rear = -1
        else:
            self.front = (self.front + 1) % self.capacity
            
        self.size -= 1
        return True

    def get_front(self) -> int:
        return -1 if self.is_empty() else self.queue[self.front]

    def get_rear(self) -> int:
        return -1 if self.is_empty() else self.queue[self.rear]

    def is_empty(self) -> bool:
        return self.size == 0

    def is_full(self) -> bool:
        return self.size == self.capacity
\end{minted}

% Stack using Queue
\noindent\textbf{Problem: Implementing Stack using Queue}
\begin{minted}[
bgcolor=bgcolor7,
frame=lines,
framesep=5mm,
rulecolor=\color{black},
linenos,
numbersep=5pt,
fontsize=\normalsize
]{python}
from collections import deque

class StackUsingQueue:
    """
    Stack implementation using two queues.
    Time Complexity: 
        Push: O(1), Pop: O(n), 
        Top: O(1), Empty: O(1)
    """
    def __init__(self):
        self.main = deque()
        self.aux = deque()

    def push(self, x: int) -> None:
        self.main.append(x)

    def pop(self) -> int:
        # Move all except last to auxiliary queue
        while len(self.main) > 1:
            self.aux.append(self.main.popleft())
        
        # Remove and return last element
        val = self.main.popleft()
        
        # Swap queues
        self.main, self.aux = self.aux, self.main
        return val

    def top(self) -> int:
        return self.main[-1] if self.main else -1

    def empty(self) -> bool:
        return len(self.main) == 0
\end{minted}

% Queue using Single Stack
\noindent\textbf{Problem: Implementing Queue using Single Stack}
\begin{minted}[
bgcolor=bgcolor5,
frame=lines,
framesep=5mm,
rulecolor=\color{black},
linenos,
numbersep=5pt,
fontsize=\normalsize
]{python}
class QueueUsingStack:
    """
    Queue implementation using two stacks (single stack in interface).
    Time Complexity: 
        Enqueue: O(1), Dequeue: Amortized O(1)
    """
    def __init__(self):
        self.in_stack = []
        self.out_stack = []

    def enqueue(self, x: int) -> None:
        self.in_stack.append(x)

    def dequeue(self) -> int:
        if not self.out_stack:
            while self.in_stack:
                self.out_stack.append(self.in_stack.pop())
        return self.out_stack.pop() if self.out_stack else -1

    def peek(self) -> int:
        if not self.out_stack:
            while self.in_stack:
                self.out_stack.append(self.in_stack.pop())
        return self.out_stack[-1] if self.out_stack else -1

    def empty(self) -> bool:
        return not self.in_stack and not self.out_stack
\end{minted}

% Reverse First K Elements of Queue
\noindent\textbf{Problem: Reversing First K Elements of a Queue}
\begin{minted}[
bgcolor=bgcolor3,
frame=lines,
framesep=5mm,
rulecolor=\color{black},
linenos,
numbersep=5pt,
fontsize=\normalsize
]{python}
from collections import deque

def reverse_k(queue: deque, k: int) -> deque:
    """
    Reverses first k elements of a queue using stack.
    Time Complexity: O(n), Space Complexity: O(k)
    """
    if k > len(queue) or k <= 0:
        return queue
        
    stack = []
    # Push first k elements to stack
    for _ in range(k):
        stack.append(queue.popleft())
    
    # Pop from stack to end of queue (reversed order)
    while stack:
        queue.append(stack.pop())
    
    # Move remaining elements to end
    for _ in range(len(queue) - k):
        queue.append(queue.popleft())
        
    return queue
\end{minted}

% Sliding Window Maximum
\noindent\textbf{Problem: Sliding Window Maximum}
\begin{minted}[
bgcolor=bgcolor8,
frame=lines,
framesep=5mm,
rulecolor=\color{black},
linenos,
numbersep=5pt,
fontsize=\normalsize
]{python}
from collections import deque

def max_sliding_window(nums: List[int], k: int) -> List[int]:
    """
    Finds max in each sliding window using monotonic deque.
    Time Complexity: O(n), Space Complexity: O(k)
    """
    dq = deque()
    result = []
    
    for i in range(len(nums)):
        # Remove indices outside current window
        if dq and dq[0] == i - k:
            dq.popleft()
            
        # Maintain decreasing order
        while dq and nums[dq[-1]] < nums[i]:
            dq.pop()
            
        dq.append(i)
        
        # Add to result once window size reached
        if i >= k - 1:
            result.append(nums[dq[0]])
    
    return result
\end{minted}

% Generate Numbers with Given Digits
\noindent\textbf{Problem: Generate Numbers with Given Digits Only}
\begin{minted}[
bgcolor=bgcolor1,
frame=lines,
framesep=5mm,
rulecolor=\color{black},
linenos,
numbersep=5pt,
fontsize=\normalsize
]{python}
from collections import deque

def generate_numbers(digits: List[int], n: int) -> List[int]:
    """
    Generates first n numbers containing only given digits.
    Time Complexity: O(n), Space Complexity: O(n)
    """
    result = []
    queue = deque()
    # Start with each digit
    for d in sorted(digits):
        queue.append(d)
    
    count = 0
    while queue and count < n:
        num = queue.popleft()
        result.append(num)
        count += 1
        
        # Generate next numbers by appending digits
        for d in sorted(digits):
            new_num = num * 10 + d
            queue.append(new_num)
    
    return result
\end{minted}

% Circular Deque Implementation
\noindent\textbf{Problem: Circular Implementation of Deque}
\begin{minted}[
bgcolor=bgcolor9,
frame=lines,
framesep=5mm,
rulecolor=\color{black},
linenos,
numbersep=5pt,
fontsize=\normalsize
]{python}
class CircularDeque:
    """
    Circular deque implementation with O(1) operations.
    Space Complexity: O(n)
    """
    def __init__(self, k: int):
        self.capacity = k
        self.deque = [None] * k
        self.front = self.rear = -1
        self.size = 0

    def insert_front(self, value: int) -> bool:
        if self.is_full():
            return False
            
        if self.is_empty():
            self.front = self.rear = 0
        else:
            self.front = (self.front - 1) % self.capacity
            
        self.deque[self.front] = value
        self.size += 1
        return True

    def insert_last(self, value: int) -> bool:
        if self.is_full():
            return False
            
        if self.is_empty():
            self.front = self.rear = 0
        else:
            self.rear = (self.rear + 1) % self.capacity
            
        self.deque[self.rear] = value
        self.size += 1
        return True

    def delete_front(self) -> bool:
        if self.is_empty():
            return False
            
        if self.front == self.rear:
            self.front = self.rear = -1
        else:
            self.front = (self.front + 1) % self.capacity
            
        self.size -= 1
        return True

    def delete_last(self) -> bool:
        if self.is_empty():
            return False
            
        if self.front == self.rear:
            self.front = self.rear = -1
        else:
            self.rear = (self.rear - 1) % self.capacity
            
        self.size -= 1
        return True

    def get_front(self) -> int:
        return -1 if self.is_empty() else self.deque[self.front]

    def get_rear(self) -> int:
        return -1 if self.is_empty() else self.deque[self.rear]

    def is_empty(self) -> bool:
        return self.size == 0

    def is_full(self) -> bool:
        return self.size == self.capacity
\end{minted}

% Gas Station Problem
\noindent\textbf{Problem: First Circular Tour (Gas Station Problem)}
\begin{minted}[
bgcolor=bgcolor6,
frame=lines,
framesep=5mm,
rulecolor=\color{black},
linenos,
numbersep=5pt,
fontsize=\normalsize
]{python}
def can_complete_circuit(gas: List[int], cost: List[int]) -> int:
    """
    Finds starting gas station index to complete circular tour.
    Time Complexity: O(n), Space Complexity: O(1)
    """
    total_tank = current_tank = start = 0
    
    for i in range(len(gas)):
        total_tank += gas[i] - cost[i]
        current_tank += gas[i] - cost[i]
        
        # If current tank negative, reset from next station
        if current_tank < 0:
            start = i + 1
            current_tank = 0
    
    return start if total_tank >= 0 else -1
\end{minted}

% Min-Max Data Structure
\noindent\textbf{Problem: Design a Data Structure with Min and Max}
\begin{minted}[
bgcolor=bgcolor4,
frame=lines,
framesep=5mm,
rulecolor=\color{black},
linenos,
numbersep=5pt,
fontsize=\normalsize
]{python}
from collections import deque

class MinMaxDeque:
    """
    Deque that supports O(1) min/max operations using auxiliary deques.
    Time Complexity: All operations O(1) amortized.
    """
    def __init__(self):
        self.main = deque()
        self.min_deque = deque()
        self.max_deque = deque()

    def append(self, x: int) -> None:
        self.main.append(x)
        # Maintain min deque (increasing)
        while self.min_deque and self.min_deque[-1] > x:
            self.min_deque.pop()
        self.min_deque.append(x)
        # Maintain max deque (decreasing)
        while self.max_deque and self.max_deque[-1] < x:
            self.max_deque.pop()
        self.max_deque.append(x)

    def popleft(self) -> int:
        if not self.main:
            return -1
        val = self.main.popleft()
        if val == self.min_deque[0]:
            self.min_deque.popleft()
        if val == self.max_deque[0]:
            self.max_deque.popleft()
        return val

    def pop(self) -> int:
        if not self.main:
            return -1
        val = self.main.pop()
        if val == self.min_deque[-1]:
            self.min_deque.pop()
        if val == self.max_deque[-1]:
            self.max_deque.pop()
        return val

    def get_min(self) -> int:
        return self.min_deque[0] if self.min_deque else -1

    def get_max(self) -> int:
        return self.max_deque[0] if self.max_deque else -1
\end{minted}

% \end{document}
