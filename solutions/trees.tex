% \documentclass[a4paper,10pt]{article}
% \usepackage[utf8]{inputenc}
% \usepackage{geometry}
% \usepackage[table]{xcolor}
% \usepackage{colortbl}
% \usepackage{color,soul}
% \geometry{margin=0.8in}
% \usepackage{xcolor}
% \usepackage{tikz}
% \usepackage{minted}
% \definecolor{bgcolor}{rgb}{0.8, 0.9, 0.5} % 
% \definecolor{bgcolor1}{rgb}{0.95, 0.95, 0.95} % Light Gray
% \definecolor{bgcolor2}{rgb}{0.85, 0.92, 1.0}  % Soft Blue
% \definecolor{bgcolor3}{rgb}{0.9, 0.85, 1.0}   % Light Purple
% \definecolor{bgcolor4}{rgb}{0.95, 0.88, 0.76} % Warm Beige
% \definecolor{bgcolor5}{rgb}{0.8, 0.95, 0.8}   % Gentle Green
% \definecolor{bgcolor6}{rgb}{1.0, 0.87, 0.87}  % Pastel Red
% \definecolor{bgcolor7}{rgb}{0.86, 0.93, 0.83} % Mint Green
% \definecolor{bgcolor8}{rgb}{0.98, 0.85, 0.94} % Soft Pink
% \definecolor{bgcolor9}{rgb}{0.87, 0.94, 0.98} % Sky Blue
% \definecolor{bgcolor10}{rgb}{0.96, 0.96, 0.82} % Pale Yellow
% 
% % Define a custom background colo
% \begin{document}
\noindent\textbf{Problem: Iterative Inorder Traversal}
\begin{minted}[
bgcolor=bgcolor7,
frame=lines,
framesep=5mm,
rulecolor=\color{black},
linenos,
numbersep=5pt,
fontsize=\normalsize
]{python}
class TreeNode:
    def __init__(self, val=0, left=None, right=None):
        self.val = val
        self.left = left
        self.right = right

def inorder_traversal(root: TreeNode) -> List[int]:
    """
    Performs iterative inorder traversal of a binary tree.
    Time Complexity: O(n)      Space Complexity: O(n)
    """
    stack = []
    result = []
    current = root
    
    while current or stack:
        # Traverse to the leftmost node
        while current:
            stack.append(current)
            current = current.left
        # Process the node at the top of the stack
        current = stack.pop()
        result.append(current.val)
        # Move to the right subtree
        current = current.right
    
    return result
\end{minted}

\noindent\textbf{Problem: Iterative Preorder Traversal}
\begin{minted}[
bgcolor=bgcolor2,
frame=lines,
framesep=5mm,
rulecolor=\color{black},
linenos,
numbersep=5pt,
fontsize=\normalsize
]{python}
def preorder_traversal(root: TreeNode) -> List[int]:
    """
    Performs iterative preorder traversal of a binary tree.
    Time Complexity: O(n)      Space Complexity: O(n)
    """
    if not root:
        return []
    
    stack = [root]
    result = []
    
    while stack:
        node = stack.pop()
        result.append(node.val)
        # Push right child first (so left is processed first)
        if node.right:
            stack.append(node.right)
        if node.left:
            stack.append(node.left)
    
    return result
\end{minted}

\noindent\textbf{Problem: Iterative Postorder Traversal}
\begin{minted}[
bgcolor=bgcolor6,
frame=lines,
framesep=5mm,
rulecolor=\color{black},
linenos,
numbersep=5pt,
fontsize=\normalsize
]{python}
def postorder_traversal(root: TreeNode) -> List[int]:
    """
    Performs iterative postorder traversal using two stacks.
    Time Complexity: O(n)      Space Complexity: O(n)
    """
    if not root:
        return []
    
    stack1 = [root]
    stack2 = []
    
    while stack1:
        node = stack1.pop()
        stack2.append(node.val)
        # Push left then right to stack1 (so right then left in stack2)
        if node.left:
            stack1.append(node.left)
        if node.right:
            stack1.append(node.right)
    
    # Reverse stack2 to get postorder
    return stack2[::-1]
    
def postorder(root):
    stack = []
    current = root
    last_visited = None
    
    while stack or current:
        if current:
            stack.append(current)
            current = current.left
        else:
            peek = stack[-1]
             # if right child exists and is not visited
            if peek.right and last_visited != peek.right:
                current = peek.right
            else:
                print(peek.val)
                last_visited = stack.pop()

\end{minted}
\noindent\textbf{Problem: Pre, In, Post Order in Single DFS}
\begin{minted}[
bgcolor=bgcolor1,
frame=lines,
framesep=5mm,
rulecolor=\color{black},
linenos,
numbersep=5pt,
fontsize=\normalsize
]{python}
def all_orders(root: TreeNode) -> tuple:
    """
    Returns all three traversals (pre, in, post) in single DFS pass.
    Time Complexity: O(n)      Space Complexity: O(n)
    """
    preorder, inorder, postorder = [], [], []
    stack = [(root, 1)]
    while stack:
        node, state = stack.pop()
        if state == 1:  # Preorder
            preorder.append(node.val)
            stack.append((node, 2))
            if node.left:
                stack.append((node.left, 1))
        elif state == 2:  # Inorder
            inorder.append(node.val)
            stack.append((node, 3))
            if node.right:
                stack.append((node.right, 1))
        else:  # Postorder
            postorder.append(node.val)
    return preorder, inorder, postorder
\end{minted}
\noindent\textbf{Problem: Morris Inorder Traversal}
\begin{minted}[
bgcolor=bgcolor2,
frame=lines,
framesep=5mm,
rulecolor=\color{black},
linenos,
numbersep=5pt,
fontsize=\normalsize
]{python}
def morris_inorder(root: TreeNode) -> List[int]:
    """
    Performs Morris inorder traversal (threaded binary tree).
    Time Complexity: O(n)      Space Complexity: O(1)
    """
    result = []
    current = root
    
    while current:
        if not current.left:
            # If no left child, visit node and move right
            result.append(current.val)
            current = current.right
        else:
            # Find inorder predecessor
            pre = current.left
            while pre.right and pre.right != current:
                pre = pre.right
            # Create threaded link
            if not pre.right:
                pre.right = current
                current = current.left
            # Break threaded link
            else:
                pre.right = None
                result.append(current.val)
                current = current.right
    return result
\end{minted}
\noindent\textbf{Problem: Morris Preorder Traversal}
\begin{minted}[
bgcolor=bgcolor3,
frame=lines,
framesep=5mm,
rulecolor=\color{black},
linenos,
numbersep=5pt,
fontsize=\normalsize
]{python}
def morris_preorder(root: TreeNode) -> List[int]:
    """
    Performs Morris preorder traversal (threaded binary tree).
    Time Complexity: O(n)      Space Complexity: O(1)
    """
    result = []
    current = root
    while current:
        if not current.left:
            result.append(current.val)
            current = current.right
        else:
            pre = current.left
            while pre.right and pre.right != current:
                pre = pre.right
            if not pre.right:
                result.append(current.val)  # Visit before threading
                pre.right = current
                current = current.left
            else:
                pre.right = None
                current = current.right
    return result
\end{minted}
\noindent\textbf{Problem: Left, Right, Top, Bottom View of Tree}
\begin{minted}[
bgcolor=bgcolor4,
frame=lines,
framesep=5mm,
rulecolor=\color{black},
linenos,
numbersep=5pt,
fontsize=\normalsize
]{python}
from collections import deque

def tree_views(root: TreeNode) -> dict:
    """
    Returns left, right, top, and bottom views of binary tree.
    Time Complexity: O(n)      Space Complexity: O(n)
    """
    if not root:
        return {}
    
    # Initialize dictionaries for views
    left_view = {}
    right_view = {}
    top_view = {}
    bottom_view = {}
    queue = deque([(root, 0, 0)])  # (node, col, level)
    
    while queue:
        node, col, level = queue.popleft()
        
        # Left view (first node at each level)
        if level not in left_view:
            left_view[level] = node.val
        
        # Right view (last node at each level)
        right_view[level] = node.val
        
        # Top view (first node at each vertical)
        if col not in top_view:
            top_view[col] = node.val
        
        # Bottom view (last node at each vertical)
        bottom_view[col] = node.val
        
        if node.left:
            queue.append((node.left, col - 1, level + 1))
        if node.right:
            queue.append((node.right, col + 1, level + 1))
    
    return {
        'left': list(left_view.values()),
        'right': list(right_view.values()),
        'top': [val for _, val in sorted(top_view.items())],
        'bottom': [val for _, val in sorted(bottom_view.items())]
    }
\end{minted}

\noindent\textbf{Problem: Level Order Traversal}
\begin{minted}[
bgcolor=bgcolor,
frame=lines,
framesep=5mm,
rulecolor=\color{black},
linenos,
numbersep=5pt,
fontsize=\normalsize
]{python}
from collections import deque

def level_order(root: TreeNode) -> List[List[int]]:
    """
    Performs level order traversal (BFS) of a binary tree.
    Time Complexity: O(n)      Space Complexity: O(n)
    """
    if not root:
        return []
    
    queue = deque([root])
    result = []
    
    while queue:
        level = []
        level_size = len(queue)
        for _ in range(level_size):
            node = queue.popleft()
            level.append(node.val)
            if node.left:
                queue.append(node.left)
            if node.right:
                queue.append(node.right)
        result.append(level)
    
    return result
\end{minted}
\noindent\textbf{Problem: Check for Height  Balanced Binary Tree}
\begin{minted}[
bgcolor=bgcolor6,
frame=lines,
framesep=5mm,
rulecolor=\color{black},
linenos,
numbersep=5pt,
fontsize=\normalsize
]{python}
def isBalanced(root: TreeNode) -> bool:
    """
    Checks if a binary tree is height-balanced.
    Time Complexity: O(n)      Space Complexity: O(h) - recursion stack
    """
    def dfs(node):
        if not node:
            return 0, True
        left_height, left_balanced = dfs(node.left)
        right_height, right_balanced = dfs(node.right)
        balanced = left_balanced and right_balanced and abs(left_height - right_height)<=1
        return max(left_height, right_height) + 1, balanced
    return dfs(root)[1]
\end{minted}

\noindent\textbf{Problem: Maximum Width of Binary Tree}
\begin{minted}[
bgcolor=bgcolor7,
frame=lines,
framesep=5mm,
rulecolor=\color{black},
linenos,
numbersep=5pt,
fontsize=\normalsize
]{python}
from collections import deque

def widthOfBinaryTree(root):
    """
    Finds the maximum width of a binary tree (including null nodes between ends).
    Time Complexity: O(n)      Space Complexity: O(n)
    """
    if not root:
        return 0
    queue = deque([(root, 0)])
    max_width = 0

    while queue:
        level_size = len(queue)
        _, first_index = queue[0]
        last_index = first_index  # initialize last_index for current level

        for _ in range(level_size):
            node, col_index = queue.popleft()
            last_index = col_index  # update last_index to the current node's index

            if node.left:
                queue.append((node.left, 2 * col_index + 1))
            if node.right:
                queue.append((node.right, 2 * col_index + 2))

        max_width = max(max_width, last_index - first_index + 1)

    return max_width
\end{minted}

\noindent\textbf{Problem: Construct Tree from Inorder and Preorder}
\begin{minted}[
bgcolor=bgcolor8,
frame=lines,
framesep=5mm,
rulecolor=\color{black},
linenos,
numbersep=5pt,
fontsize=\normalsize
]{python}
def buildTree(preorder: List[int], inorder: List[int]) -> TreeNode:
    """
    Constructs a binary tree from inorder and preorder traversal arrays.
    Time Complexity: O(n)      Space Complexity: O(n) - for hashmap
    """
    inorder_map = {val: idx for idx, val in enumerate(inorder)}
    pre_idx = 0
    
    def array_to_tree(left, right):
        nonlocal pre_idx
        if left > right:
            return None
        root_val = preorder[pre_idx]
        pre_idx += 1
        root = TreeNode(root_val)
        idx = inorder_map[root_val]
        root.left = array_to_tree(left, idx - 1)
        root.right = array_to_tree(idx + 1, right)
        return root
    
    return array_to_tree(0, len(inorder) - 1)
\end{minted}

\noindent\textbf{Problem: Tree Traversal in Spiral Form}
\begin{minted}[
bgcolor=bgcolor9,
frame=lines,
framesep=5mm,
rulecolor=\color{black},
linenos,
numbersep=5pt,
fontsize=\normalsize
]{python}
from collections import deque

def spiralOrder(root: TreeNode) -> List[List[int]]:
    """
    Returns the level order traversal in spiral form (zigzag).
    Time Complexity: O(n)      Space Complexity: O(n)
    """
    if not root:
        return []
    result = []
    queue = deque([root])
    left_to_right = False
    while queue:
        level_size = len(queue)
        level = deque()
        for _ in range(level_size):
            node = queue.popleft()
            if left_to_right:
                level.appendleft(node.val)
            else:
                level.append(node.val)
            if node.left:
                queue.append(node.left)
            if node.right:
                queue.append(node.right)
        result.append(list(level))
        left_to_right = not left_to_right
    return result
\end{minted}

\noindent\textbf{Problem: Child Sum Property in Tree}
\begin{minted}[
bgcolor=bgcolor3,
frame=lines,
framesep=5mm,
rulecolor=\color{black},
linenos,
numbersep=5pt,
fontsize=\normalsize
]{python}
def childSum(root: TreeNode) -> bool:
    """
    Checks if tree satisfies child sum property (node = left + right).
    Time Complexity: O(n)      Space Complexity: O(h)
    """
    if not root or (not root.left and not root.right):
        return True
    left_val = root.left.val if root.left else 0
    right_val = root.right.val if root.right else 0
    if root.val != left_val + right_val:
        return False
    return childSum(root.left) and childSum(root.right)
\end{minted}
\noindent\textbf{Problem: Maintain Child Sum Property in Tree}
\begin{minted}[
bgcolor=bgcolor9,
frame=lines,
framesep=5mm,
rulecolor=\color{black},
linenos,
numbersep=5pt,
fontsize=\normalsize
]{python}

def change_tree(node):
    if node is None or (node.left is None and node.right is None):
        return

    # Recursively fix left and right subtree first
    change_tree(node.left)
    change_tree(node.right)

    # Calculate sum of children
    left_val = node.left.val if node.left else 0
    right_val = node.right.val if node.right else 0
    children_sum = left_val + right_val

    if children_sum >= node.val:
        # If children sum is greater or equal, update current node
        node.val = children_sum
    else:
        # If node value is greater, increment children to match node.val
        diff = node.val - children_sum
        increment_child(node, diff)

def increment_child(node, diff):
    # Increment children values to maintain property recursively
    if node.left:
        node.left.val += diff
        change_tree(node.left)
    elif node.right:
        node.right.val += diff
        change_tree(node.right)
\end{minted}

\noindent\textbf{Problem: Convert Binary Tree to Doubly Linked List}
\begin{minted}[
bgcolor=bgcolor10,
frame=lines,
framesep=5mm,
rulecolor=\color{black},
linenos,
numbersep=5pt,
fontsize=\normalsize
]{python}
def BTToDLL(root: TreeNode) -> TreeNode:
    """
    Converts binary tree to DLL using inorder traversal (in-place).
    Time Complexity: O(n)      Space Complexity: O(h)
    """
    prev = None
    head = None
    
    def inorder(node):
        nonlocal prev, head
        if not node:
            return
        inorder(node.left)
        if not prev:
            head = node
        else:
            prev.right = node
            node.left = prev
        prev = node
        inorder(node.right)
    
    inorder(root)
    return head
\end{minted}
\noindent\textbf{Problem: Convert Binary Tree to Singly Linked List}
\begin{minted}[
bgcolor=bgcolor,
frame=lines,
framesep=5mm,
rulecolor=\color{black},
linenos,
numbersep=5pt,
fontsize=\normalsize
]{python}
def flatten(root: TreeNode) -> None:
    """
    Converts binary tree to singly linked list (in-place, preorder).
    Time Complexity: O(n)      Space Complexity: O(1) - Morris traversal
    """
    current = root
    while current:
        if current.left:
            pre = current.left
            while pre.right:
                pre = pre.right
            pre.right = current.right
            current.right = current.left
            current.left = None
        current = current.right
\end{minted}

\noindent\textbf{Problem: Finding LCA in Binary Tree}
\begin{minted}[
bgcolor=bgcolor2,
frame=lines,
framesep=5mm,
rulecolor=\color{black},
linenos,
numbersep=5pt,
fontsize=\normalsize
]{python}
def lowestCommonAncestor(root: TreeNode, p: TreeNode, q: TreeNode) -> TreeNode:
    """
    Finds lowest common ancestor of two nodes in binary tree.
    Time Complexity: O(n)      Space Complexity: O(h)
    """
    if not root or root == p or root == q:
        return root
    left = lowestCommonAncestor(root.left, p, q)
    right = lowestCommonAncestor(root.right, p, q)
    if left and right:
        return root
    return left if left else right
\end{minted}
\noindent\textbf{Problem: Burn a Binary Tree From a Leaf}
\begin{minted}[
bgcolor=bgcolor,
frame=lines,
framesep=5mm,
rulecolor=\color{black},
linenos,
numbersep=5pt,
fontsize=\normalsize
]{python}
def burn_time(root: TreeNode, leaf: TreeNode) -> int:
    """
    Calculates time to burn entire tree starting from a leaf node.
    Time Complexity: O(n)      Space Complexity: O(n)
    """
    parent_map = {}
    # Build parent pointers with BFS
    queue = deque([root])
    while queue:
        node = queue.popleft()
        if node.left:
            parent_map[node.left] = node
            queue.append(node.left)
        if node.right:
            parent_map[node.right] = node
            queue.append(node.right)
    
    # BFS from leaf node
    visited = set()
    queue = deque([(leaf, 0)])
    max_time = 0
    while queue:
        for _ in range(len(queue)):
            node, time = queue.popleft()
            max_time = max(max_time, time)
            visited.add(node)
            # Add parent
            if node in parent_map and parent_map[node] not in visited:
                queue.append((parent_map[node], time + 1))
            # Add left child
            if node.left and node.left not in visited:
                queue.append((node.left, time + 1))
            # Add right child
            if node.right and node.right not in visited:
                queue.append((node.right, time + 1))
    return max_time
\end{minted}
\noindent\textbf{Problem: Burn a Binary Tree From Any Node}
\begin{minted}[
bgcolor=bgcolor1,
frame=lines,
framesep=5mm,
rulecolor=\color{black},
linenos,
numbersep=5pt,
fontsize=\normalsize
]{python}
def burn_tree(root: TreeNode, start: TreeNode) -> int:
    """
    Calculates time to burn entire tree starting from any node.
    Time Complexity: O(n)      Space Complexity: O(n)
    """
    # Build parent pointers and find start node
    parent_map = {}
    target = None
    queue = deque([root])
    while queue:
        node = queue.popleft()
        if node == start:
            target = node
        if node.left:
            parent_map[node.left] = node
            queue.append(node.left)
        if node.right:
            parent_map[node.right] = node
            queue.append(node.right)
    
    # BFS from target node
    visited = set([target])
    queue = deque([(target, 0)])
    max_time = 0
    while queue:
        size = len(queue)
        for _ in range(size):
            node, time = queue.popleft()
            max_time = max(max_time, time)
            # Check parent
            if node in parent_map and parent_map[node] not in visited:
                visited.add(parent_map[node])
                queue.append((parent_map[node], time + 1))
            # Check left child
            if node.left and node.left not in visited:
                visited.add(node.left)
                queue.append((node.left, time + 1))
            # Check right child
            if node.right and node.right not in visited:
                visited.add(node.right)
                queue.append((node.right, time + 1))
    return max_time
\end{minted}
\noindent\textbf{Problem: Serialize and Deserialize Binary Tree}
\begin{minted}[
bgcolor=bgcolor10,
frame=lines,
framesep=5mm,
rulecolor=\color{black},
linenos,
numbersep=5pt,
fontsize=\normalsize
]{python}
def serialize(root: TreeNode) -> str:
    """
    Encodes tree to a single string.
    Time Complexity: O(n)      Space Complexity: O(n)
    """
    if not root:
        return "null"
    return f"{root.val},{serialize(root.left)},{serialize(root.right)}"

def deserialize(data: str) -> TreeNode:
    """
    Decodes encoded data to tree.
    Time Complexity: O(n)      Space Complexity: O(n)
    """
    nodes = data.split(',')
    idx = 0
    
    def build_tree():
        nonlocal idx
        if idx >= len(nodes) or nodes[idx] == "null":
            idx += 1
            return None
        node = TreeNode(int(nodes[idx]))
        idx += 1
        node.left = build_tree()
        node.right = build_tree()
        return node
    
    return build_tree()
\end{minted}
\noindent\textbf{Problem: Insertion in Binary Tree}
\begin{minted}[
bgcolor=bgcolor9,
frame=lines,
framesep=5mm,
rulecolor=\color{black},
linenos,
numbersep=5pt,
fontsize=\normalsize
]{python}
from collections import deque

def insert(root: TreeNode, val: int) -> TreeNode:
    """
    Inserts node at first available position in level order.
    Time Complexity: O(n)      Space Complexity: O(n)
    """
    if not root:
        return TreeNode(val)
    queue = deque([root])
    while queue:
        node = queue.popleft()
        if not node.left:
            node.left = TreeNode(val)
            return root
        else:
            queue.append(node.left)
        if not node.right:
            node.right = TreeNode(val)
            return root
        else:
            queue.append(node.right)
    return root
\end{minted}
\noindent\textbf{Problem: Deletion in Binary Tree}
\begin{minted}[
bgcolor=bgcolor,
frame=lines,
framesep=5mm,
rulecolor=\color{black},
linenos,
numbersep=5pt,
fontsize=\normalsize
]{python}
def delete_node(root: TreeNode, key: int) -> TreeNode:
    if not root:
        return None

    if root.left is None and root.right is None:
        return None if root.val == key else root

    key_node = None
    queue = deque([root])
    last_node = None
    parent_of_last = None

    # Level order traversal to find key_node and last_node
    while queue:
        last_node = queue.popleft()
        if last_node.val == key:
            key_node = last_node
        if last_node.left:
            parent_of_last = last_node
            queue.append(last_node.left)
        if last_node.right:
            parent_of_last = last_node
            queue.append(last_node.right)

    if key_node:
        key_node.val = last_node.val  # Replace key_node's value with last_node's
        # Delete the deepest node
        if parent_of_last.right == last_node:
            parent_of_last.right = None
        else:
            parent_of_last.left = None

    return root
\end{minted}
\noindent\textbf{Problem: Convert Binary Tree into Mirror Tree}
\begin{minted}[
bgcolor=bgcolor8,
frame=lines,
framesep=5mm,
rulecolor=\color{black},
linenos,
numbersep=5pt,
fontsize=\normalsize
]{python}
def mirror(root: TreeNode) -> None:
    """
    Converts binary tree into its mirror (in-place).
    Time Complexity: O(n)      Space Complexity: O(h)
    """
    if not root:
        return
    # Swap left and right subtrees
    root.left, root.right = root.right, root.left
    mirror(root.left)
    mirror(root.right)
\end{minted}
\noindent\textbf{Problem: Count Nodes in Complete Binary Tree}
\begin{minted}[
bgcolor=bgcolor7,
frame=lines,
framesep=5mm,
rulecolor=\color{black},
linenos,
numbersep=5pt,
fontsize=\normalsize
]{python}
def countNodes(root: TreeNode) -> int:
    """
    Counts nodes in complete binary tree in less than O(n) time.
    Time Complexity: O((log n)^2)      Space Complexity: O(log n)
    """
    if not root:
        return 0
    left_depth = 0
    right_depth = 0
    node = root
    while node:
        left_depth += 1
        node = node.left
    node = root
    while node:
        right_depth += 1
        node = node.right
    if left_depth == right_depth:
        return (1 << left_depth) - 1
    return 1 + countNodes(root.left) + countNodes(root.right)
\end{minted}

\noindent\textbf{Problem: Diameter of a Binary Tree}
\begin{minted}[
bgcolor=bgcolor6,
frame=lines,
framesep=5mm,
rulecolor=\color{black},
linenos,
numbersep=5pt,
fontsize=\normalsize
]{python}
def diameterOfBinaryTree(root: TreeNode) -> int:
    """
    Calculates diameter (longest path between any two nodes) of binary tree.
    Time Complexity: O(n)      Space Complexity: O(h)
    """
    max_diameter = 0
    
    def dfs(node):
        nonlocal max_diameter
        if not node:
            return 0
        left = dfs(node.left)
        right = dfs(node.right)
        max_diameter = max(max_diameter, left + right)
        return max(left, right) + 1
    
    dfs(root)
    return max_diameter
\end{minted}

\noindent\textbf{Problem: Max Path Sum (Any Node to Any Node)}
\begin{minted}[
bgcolor=bgcolor4,
frame=lines,
framesep=5mm,
rulecolor=\color{black},
linenos,
numbersep=5pt,
fontsize=\normalsize
]{python}
def maxPathSum(root: TreeNode) -> int:
    """
    Finds maximum path sum between any two nodes (path may not pass through root).
    Time Complexity: O(n)      Space Complexity: O(h)
    """
    max_sum = float('-inf')
    
    def dfs(node):
        nonlocal max_sum
        if not node:
            return 0
        left = max(dfs(node.left), 0)
        right = max(dfs(node.right), 0)
        max_sum = max(max_sum, node.val + left + right)
        return node.val + max(left, right)
    
    dfs(root)
    return max_sum
\end{minted}

\noindent\textbf{Problem: Max Path Sum Leaf to Leaf}
\begin{minted}[
bgcolor=bgcolor4,
frame=lines,
framesep=5mm,
rulecolor=\color{black},
linenos,
numbersep=5pt,
fontsize=\normalsize
]{python}
def maxPathSumLeaf(root: TreeNode) -> int:
    """
    Finds maximum path sum from leaf to leaf.
    Time Complexity: O(n)      Space Complexity: O(h)
    """
    max_sum = float('-inf')
    
    def dfs(node):
        nonlocal max_sum
        if not node:
            return 0
        left = dfs(node.left)
        right = dfs(node.right)
        # Only update if both children exist
        if node.left and node.right:
            max_sum = max(max_sum, node.val + left + right)
        # Return max path from this node to leaf
        return node.val + max(left, right)
    
    dfs(root)
    return max_sum if max_sum != float('-inf') else -1
\end{minted}
\noindent\textbf{Problem: Diameter of N-ary Tree}
\begin{minted}[
bgcolor=bgcolor10,
frame=lines,
framesep=5mm,
rulecolor=\color{black},
linenos,
numbersep=5pt,
fontsize=\normalsize
]{python}
class NaryNode:
    def __init__(self, val=None, children=None):
        self.val = val
        self.children = children if children else []

def diameter_Nary(root: NaryNode) -> int:
    """
    Calculates diameter of N-ary tree (longest path between nodes).
    Time Complexity: O(n)      Space Complexity: O(h)
    """
    diameter = 0
    
    def dfs(node):
        nonlocal diameter
        if not node:
            return 0
        max1 = max2 = 0  # Two largest depths
        for child in node.children:
            depth = dfs(child)
            if depth > max1:
                max2 = max1
                max1 = depth
            elif depth > max2:
                max2 = depth
        diameter = max(diameter, max1 + max2)
        return max1 + 1
    
    dfs(root)
    return diameter
\end{minted}
\noindent\textbf{Problem: Boundary Traversal of Binary Tree}
\begin{minted}[
bgcolor=bgcolor3,
frame=lines,
framesep=5mm,
rulecolor=\color{black},
linenos,
numbersep=5pt,
fontsize=\normalsize
]{python}
def boundaryOfBinaryTree(root: TreeNode) -> List[int]:
    """
    Performs boundary traversal (anti-clockwise) of binary tree.
    Time Complexity: O(n)      Space Complexity: O(h)
    """
    if not root:
        return []
    result = [root.val]
    
    def left_boundary(node):
        if not node or (not node.left and not node.right):
            return
        result.append(node.val)
        if node.left:
            left_boundary(node.left)
        else:
            left_boundary(node.right)
    
    def leaves(node):
        if not node:
            return
        if not node.left and not node.right and node != root:
            result.append(node.val)
            return
        leaves(node.left)
        leaves(node.right)
    
    def right_boundary(node):
        if not node or (not node.left and not node.right):
            return
        if node.right:
            right_boundary(node.right)
        else:
            right_boundary(node.left)
        if node != root:
            result.append(node.val)
    
    left_boundary(root.left)
    leaves(root)
    right_boundary(root.right)
    return result
\end{minted}

% \end{document}
