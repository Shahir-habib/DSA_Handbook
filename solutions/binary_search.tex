% \documentclass[a4paper,10pt]{article}
% \usepackage[utf8]{inputenc}
% \usepackage{geometry}
% \usepackage[table]{xcolor}
% \usepackage{colortbl}
% \usepackage{color,soul}
% \geometry{margin=0.8in}
% \usepackage{xcolor}
% \usepackage{tikz}
% \usepackage{minted}
% \definecolor{bgcolor}{rgb}{0.8, 0.9, 0.5} % 
% \definecolor{bgcolor1}{rgb}{0.95, 0.95, 0.95} % Light Gray
% \definecolor{bgcolor2}{rgb}{0.85, 0.92, 1.0}  % Soft Blue
% \definecolor{bgcolor3}{rgb}{0.9, 0.85, 1.0}   % Light Purple
% \definecolor{bgcolor4}{rgb}{0.95, 0.88, 0.76} % Warm Beige
% \definecolor{bgcolor5}{rgb}{0.8, 0.95, 0.8}   % Gentle Green
% \definecolor{bgcolor6}{rgb}{1.0, 0.87, 0.87}  % Pastel Red
% \definecolor{bgcolor7}{rgb}{0.86, 0.93, 0.83} % Mint Green
% \definecolor{bgcolor8}{rgb}{0.98, 0.85, 0.94} % Soft Pink
% \definecolor{bgcolor9}{rgb}{0.87, 0.94, 0.98} % Sky Blue
% \definecolor{bgcolor10}{rgb}{0.96, 0.96, 0.82} % Pale Yellow
% 
% % Define a custom background colo
% \begin{document}
\section*{Binary Search Based Problems Codes}

\noindent\textbf{Problem: Ternary Search}
\begin{minted}[
bgcolor=bgcolor1,
frame=lines,
framesep=5mm,
rulecolor=\color{black},
linenos,
numbersep=5pt,
fontsize=\normalsize
]{python}
def ternary_search(f: Callable[float, float], left: float, right: float, iterations: int = 100) -> float:
    """
    Performs ternary search to find the minimum (or maximum) of a unimodal function
    f within a given range [left, right].
    Time Complexity: O(log N)
    """
    for _ in range(iterations):
        # Divide the interval into three parts
        m1 = left + (right - left) / 3
        m2 = right - (right - left) / 3

        # Compare function values at m1 and m2
        if f(m1) < f(m2):
            # Minimum is in the left two-thirds
            right = m2
        else:
            # Minimum is in the right two-thirds
            left = m1
    return (left + right) / 2
\end{minted}

\noindent\textbf{Problem: Jump Search}
\begin{minted}[
bgcolor=bgcolor2,
frame=lines,
framesep=5mm,
rulecolor=\color{black},
linenos,
numbersep=5pt,
fontsize=\normalsize
]{python}
def jump_search(arr: List[int], x: int) -> int:
    """
    Searches for an element x in a sorted array using Jump Search.
    Time Complexity: O(sqrt(n))
    """
    n = len(arr)
    # Find block size to jump
    step = int(math.sqrt(n))

    # Finding the block where element is present (if it is present)
    prev = 0
    while prev < n and arr[min(step, n) - 1] < x:
        prev = step
        step += int(math.sqrt(n))
        if prev >= n:
            return -1

    # Doing a linear search for x in block beginning with prev.
    while prev < n and arr[prev] < x:
        prev += 1

    # If element is found
    if prev < n and arr[prev] == x:
        return prev
    return -1
\end{minted}

\noindent\textbf{Problem: Missing and Repeating Number}
\begin{minted}[
bgcolor=bgcolor3,
frame=lines,
framesep=5mm,
rulecolor=\color{black},
linenos,
numbersep=5pt,
fontsize=\normalsize
]{python}
def find_missing_repeating(nums: List[int]) -> Tuple[int, int]:
    """
    Finds the missing and repeating numbers in an array using XOR properties.
    Assumes array contains numbers from 1 to n, with one repeating and one missing.
    Time Complexity: O(n)
    """
    n = len(nums)
    # XOR all array elements and numbers from 1 to n
    xor_all = 0
    for i in range(1, n + 1):
        xor_all ^= i
    for num in nums:
        xor_all ^= num

    # Find the rightmost set bit
    set_bit = xor_all & -xor_all

    # Divide numbers into two groups based on the set bit
    # x: numbers with set bit, y: numbers without set bit
    x, y = 0, 0
    for i in range(1, n + 1):
        if (i & set_bit) != 0:
            x ^= i
        else:
            y ^= i
    for num in nums:
        if (num & set_bit) != 0:
            x ^= num
        else:
            y ^= num

    # Check which one is repeating and which is missing
    if x in nums:
        return x, y  # x is repeating, y is missing
    return y, x      # y is repeating, x is missing
\end{minted}

\noindent\textbf{Problem: Find Peak Element in Unsorted Array}
\begin{minted}[
bgcolor=bgcolor4,
frame=lines,
framesep=5mm,
rulecolor=\color{black},
linenos,
numbersep=5pt,
fontsize=\normalsize
]{python}
def find_peak_element(nums: List[int]) -> int:
    """
    Finds a peak element in an array using binary search.
    A peak element is an element that is strictly greater than its neighbours.
    Time Complexity: O(log n)
    """
    n = len(nums)
    left, right = 0, n - 1

    while left < right:
        mid = left + (right - left) // 2
        if nums[mid] > nums[mid + 1]:
            # Peak is in the left half including mid
            right = mid
        else:
            # Peak is in the right half excluding mid
            left = mid + 1
    return left # left or right, they will converge to the peak index
\end{minted}

\noindent\textbf{Problem: Search in an Infinite Sized Array}
\begin{minted}[
bgcolor=bgcolor5,
frame=lines,
framesep=5mm,
rulecolor=\color{black},
linenos,
numbersep=5pt,
fontsize=\normalsize
]{python}
# Assume ArrayReader is a class that allows accessing elements,
# but its size is unknown (infinite). It returns -1 for out-of-bounds.
class ArrayReader:
    def __init__(self, arr: List[int]):
        self.arr = arr

    def get(self, index: int) -> int:
        if index < len(self.arr):
            return self.arr[index]
        return 2**31 - 1 # Represents a very large number for "infinity"

def search_in_infinite_array(reader: ArrayReader, target: int) -> int:
    """
    Searches for a target in an effectively infinite sorted array.
    Time Complexity: O(log n)
    """
    # Exponentially increase bounds
    left, right = 0, 1
    while reader.get(right) < target:
        left = right
        right *= 2

    # Perform binary search within the found bounds
    while left <= right:
        mid = left + (right - left) // 2
        mid_val = reader.get(mid)
        if mid_val == target:
            return mid
        elif mid_val < target:
            left = mid + 1
        else:
            right = mid - 1
    return -1
\end{minted}

\noindent\textbf{Problem: Median of Two Sorted Arrays}
\begin{minted}[
bgcolor=bgcolor6,
frame=lines,
framesep=5mm,
rulecolor=\color{black},
linenos,
numbersep=5pt,
fontsize=\normalsize
]{python}
def find_kth_element_in_two_sorted_arrays(nums1: List[int], nums2: List[int], k: int) -> int:
    """
    Finds the k-th smallest element in two sorted arrays.
    Time Complexity: O(log(min(len(nums1), len(nums2))))
    """
    n1, n2 = len(nums1), len(nums2)

    # Ensure nums1 is the shorter array for efficiency
    if n1 > n2:
        return find_kth_element_in_two_sorted_arrays(nums2, nums1, k)

    # Base cases
    if n1 == 0:
        return nums2[k - 1]
    if k == 1:
        return min(nums1[0], nums2[0])

    # Divide k into two parts
    i = min(k // 2, n1) # Take k/2 from nums1, or all if k/2 is too much
    j = k - i           # Take remaining from nums2

    if nums1[i - 1] < nums2[j - 1]:
        # Discard first i elements from nums1
        return find_kth_element_in_two_sorted_arrays(nums1[i:], nums2, k - i)
    else:
        # Discard first j elements from nums2
        return find_kth_element_in_two_sorted_arrays(nums1, nums2[j:], k - j)

def find_median_sorted_arrays(nums1: List[int], nums2: List[int]) -> float:
    """
    Finds the median of two sorted arrays.
    Time Complexity: O(log(min(len(nums1), len(nums2))))
    """
    n = len(nums1) + len(nums2)
    if n % 2 == 1:
        # Odd total length: median is the (n//2 + 1)-th element
        return find_kth_element_in_two_sorted_arrays(nums1, nums2, n // 2 + 1)
    else:
        # Even total length: median is average of (n//2)-th and (n//2 + 1)-th elements
        mid1 = find_kth_element_in_two_sorted_arrays(nums1, nums2, n // 2)
        mid2 = find_kth_element_in_two_sorted_arrays(nums1, nums2, n // 2 + 1)
        return (mid1 + mid2) / 2.0
\end{minted}

\noindent\textbf{Problem: Search in Sorted Rotated Array}
\begin{minted}[
bgcolor=bgcolor7,
frame=lines,
framesep=5mm,
rulecolor=\color{black},
linenos,
numbersep=5pt,
fontsize=\normalsize
]{python}
def search_rotated_array(nums: List[int], target: int) -> int:
    """
    Searches for a target value in a sorted and rotated array.
    Time Complexity: O(log n)
    """
    left, right = 0, len(nums) - 1

    while left <= right:
        mid = left + (right - left) // 2

        if nums[mid] == target:
            return mid

        # Determine which half is sorted
        if nums[left] <= nums[mid]: # Left half is sorted
            if nums[left] <= target < nums[mid]:
                right = mid - 1
            else:
                left = mid + 1
        else: # Right half is sorted
            if nums[mid] < target <= nums[right]:
                left = mid + 1
            else:
                right = mid - 1
    return -1
\end{minted}

\noindent\textbf{Problem: Find Triplet with Given Sum (sorted)}
\begin{minted}[
bgcolor=bgcolor8,
frame=lines,
framesep=5mm,
rulecolor=\color{black},
linenos,
numbersep=5pt,
fontsize=\normalsize
]{python}
def find_triplets_with_sum(nums: List[int], target: int) -> List[Tuple[int, int, int]]:
    """
    Finds all unique triplets in the array that sum up to the target.
    The array is first sorted.
    Time Complexity: O(n^2)
    """
    nums.sort() # Sort the array first
    n = len(nums)
    result = []

    for i in range(n - 2):
        # Skip duplicate elements for `i`
        if i > 0 and nums[i] == nums[i - 1]:
            continue

        left, right = i + 1, n - 1
        while left < right:
            current_sum = nums[i] + nums[left] + nums[right]
            if current_sum == target:
                result.append((nums[i], nums[left], nums[right]))
                # Skip duplicate elements for `left` and `right`
                while left < right and nums[left] == nums[left + 1]:
                    left += 1
                while left < right and nums[right] == nums[right - 1]:
                    right -= 1
                left += 1
                right -= 1
            elif current_sum < target:
                left += 1
            else:
                right -= 1
    return result
\end{minted}

\noindent\textbf{Problem: One Repeating Element in Array from 1 to N (size N)}
\begin{minted}[
bgcolor=bgcolor9,
frame=lines,
framesep=5mm,
rulecolor=\color{black},
linenos,
numbersep=5pt,
fontsize=\normalsize
]{python}
def find_one_repeating_element_floyd(nums: List[int]) -> int:
    """
    Finds the one repeating element in an array of n+1 integers where
    each integer is in the range [1, n] using Floyd's Cycle Detection.
    Time Complexity: O(n)
    Space Complexity: O(1)
    """
    # Assuming numbers are 1-indexed for the 'linked list' interpretation
    # nums[i] points to nums[nums[i]]
    
    # Phase 1: Find the intersection point of the two pointers
    slow = nums[0]
    fast = nums[nums[0]]

    while slow != fast:
        slow = nums[slow]
        fast = nums[nums[fast]]

    # Phase 2: Find the entrance to the cycle
    # (which is the repeating number)
    slow = nums[0] # Reset slow to the start
    while slow != fast:
        slow = nums[slow]
        fast = nums[fast]
    
    return slow # Both pointers now point to the repeating number
\end{minted}

\noindent\textbf{Problem: Multiple Repeating Elements in Array from 1 to N (size N)}
\begin{minted}[
bgcolor=bgcolor10,
frame=lines,
framesep=5mm,
rulecolor=\color{black},
linenos,
numbersep=5pt,
fontsize=\normalsize
]{python}
def find_multiple_repeating_elements(nums: List[int]) -> List[int]:
    """
    Finds multiple repeating elements in an array where numbers are from 1 to n.
    Modifies the array in-place by negating elements to mark visited.
    Time Complexity: O(n)
    Space Complexity: O(1) (excluding result list)
    """
    repeating_elements = []
    for i in range(len(nums)):
        # Use absolute value as index (assuming numbers are positive)
        index = abs(nums[i]) - 1
        if nums[index] < 0:
            # Already visited, so this number is a repeat
            repeating_elements.append(abs(nums[i]))
        else:
            # Mark as visited by negating the value at the index
            nums[index] = -nums[index]
    
    # Optional: restore array to original state if needed
    # for i in range(len(nums)):
    #     nums[i] = abs(nums[i])

    return repeating_elements
\end{minted}

\noindent\textbf{Problem: Allocate Minimum Pages}
\begin{minted}[
bgcolor=bgcolor1,
frame=lines,
framesep=5mm,
rulecolor=\color{black},
linenos,
numbersep=5pt,
fontsize=\normalsize
]{python}
def allocate_minimum_pages(pages: List[int], num_students: int) -> int:
    """
    Allocates books to students such that the maximum pages assigned to a student is minimized.
    Uses binary search on the answer space.
    Time Complexity: O(N * log(Sum_of_Pages))
    """

    def is_possible_allocation(pages:List[int],num_students:int,max_pages_per_student: int)-> bool:
        """
        Helper function to check if it's possible to allocate books such that
        no student gets more than max_pages_per_student.
        """
        students_needed = 1
        current_pages_sum = 0
        for book_pages in pages:
            if book_pages > max_pages_per_student:
                return False # A single book is larger than allowed max
            if current_pages_sum + book_pages <= max_pages_per_student:
                current_pages_sum += book_pages
            else:
                students_needed += 1
                current_pages_sum = book_pages
                if students_needed > num_students:
                    return False
        return True

    
    if num_students > len(pages):
        return -1 # Not possible to assign at least one book to each student

    low = max(pages) # Minimum possible answer is max pages of a single book
    high = sum(pages) # Maximum possible answer is sum of all pages
    result = high

    while low <= high:
        mid = low + (high - low) // 2
        if is_possible_allocation(pages, num_students, mid):
            result = mid
            high = mid - 1 # Try to find a smaller maximum
        else:
            low = mid + 1 # Need to increase the maximum allowed pages
    return result
\end{minted}

\noindent\textbf{Problem: Minimum Days to Make n Bouquets}
\begin{minted}[
bgcolor=bgcolor2,
frame=lines,
framesep=5mm,
rulecolor=\color{black},
linenos,
numbersep=5pt,
fontsize=\normalsize
]{python}
def can_make_bouquets(bloom_days: List[int], k_fl_per_b: int, m_bouq: int, day: int) -> bool:
    """
    Helper function to check if 'm_bouq' can be made by 'day'.
    """
    bouquets_made = 0
    flowers_collected = 0
    for bloom_day in bloom_days:
        if bloom_day <= day:
            flowers_collected += 1
            if flowers_collected == k_fl_per_b:
                bouquets_made += 1
                flowers_collected = 0 # Reset for next bouquet
        else:
            flowers_collected = 0 # Reset if a non-bloomed flower breaks sequence
    return bouquets_made >= m_bouq

def min_days_to_make_bouquets(bloom_days: List[int], k: int, m: int) -> int:
    """
    Finds the minimum number of days to make 'm' bouquets,
    where each bouquet requires 'k' adjacent flowers.
    Uses binary search on the possible range of days.
    Time Complexity: O(N * log(max_day - min_day))
    """
    if m * k > len(bloom_days):
        return -1 # Not enough flowers in total

    low, high = min(bloom_days), max(bloom_days)
    result = -1

    while low <= high:
        mid_day = low + (high - low) // 2
        if can_make_bouquets(bloom_days, k, m, mid_day):
            result = mid_day
            high = mid_day - 1 # Try for an earlier day
        else:
            low = mid_day + 1 # Need more days
    return result
\end{minted}

\noindent\textbf{Problem: Capacity to Ship Packages in D Days}
\begin{minted}[
bgcolor=bgcolor3,
frame=lines,
framesep=5mm,
rulecolor=\color{black},
linenos,
numbersep=5pt,
fontsize=\normalsize
]{python}
def can_ship_packages(weights: List[int], max_capacity: int, days_limit: int) -> bool:
    """
    Helper function to check if all packages can be shipped within 'days_limit'
    with a given 'max_capacity' per day.
    """
    current_day = 1
    current_weight = 0
    for weight in weights:
        if weight > max_capacity:
            return False # A single package is too heavy
        if current_weight + weight <= max_capacity:
            current_weight += weight
        else:
            current_day += 1
            current_weight = weight
    return current_day <= days_limit

def capacity_to_ship_packages(weights: List[int], D: int) -> int:
    """
    Finds the minimum ship capacity required to ship all packages within D days.
    Uses binary search on the possible range of capacities.
    Time Complexity: O(N * log(Sum_of_Weights))
    """
    low = max(weights) # Minimum possible capacity is the heaviest package
    high = sum(weights) # Maximum possible capacity is sum of all packages
    result = high

    while low <= high:
        mid_capacity = low + (high - low) // 2
        if can_ship_packages(weights, mid_capacity, D):
            result = mid_capacity
            high = mid_capacity - 1 # Try for a smaller capacity
        else:
            low = mid_capacity + 1 # Need a larger capacity
    return result
\end{minted}

\noindent\textbf{Problem: Aggressive Cows}
\begin{minted}[
bgcolor=bgcolor4,
frame=lines,
framesep=5mm,
rulecolor=\color{black},
linenos,
numbersep=5pt,
fontsize=\normalsize
]{python}
def can_place_cows(stalls: List[int], num_cows: int, min_distance: int) -> bool:
    """
    Helper function to check if 'num_cows' can be placed in stalls such that
    the minimum distance between any two cows is at least 'min_distance'.
    """
    cows_placed = 1
    last_stall = stalls[0]
    for i in range(1, len(stalls)):
        if stalls[i] - last_stall >= min_distance:
            cows_placed += 1
            last_stall = stalls[i]
            if cows_placed == num_cows:
                return True
    return False

def aggressive_cows(stalls: List[int], k: int) -> int:
    """
    Finds the largest minimum distance between 'k' cows placed in given stalls.
    The stalls must be sorted first. Uses binary search on the possible distances.
    Time Complexity: O(N log N + N log(max_coord - min_coord))
    """
    if k > len(stalls):
        return 0 
    stalls.sort() # Sort the stall coordinates
    n = len(stalls)

    low = 1 # Minimum possible distance between cows (at least 1)
    high = stalls[-1] - stalls[0] # Maximum possible distance

    result = 0
    while low <= high:
        mid_distance = low + (high - low) // 2
        if can_place_cows(stalls, k, mid_distance):
            result = mid_distance
            low = mid_distance + 1 # Try for a larger minimum distance
        else:
            high = mid_distance - 1 # Need a smaller minimum distance
    return result
\end{minted}

\noindent\textbf{Problem: Kth Missing Positive Number}
\begin{minted}[
bgcolor=bgcolor5,
frame=lines,
framesep=5mm,
rulecolor=\color{black},
linenos,
numbersep=5pt,
fontsize=\normalsize
]{python}
def find_kth_positive(arr: List[int], k: int) -> int:
    """
    Finds the k-th positive integer that is missing from a strictly increasing array.
    Uses binary search.
    Time Complexity: O(log n)
    """
    left, right = 0, len(arr) - 1

    while left <= right:
        mid = left + (right - left) // 2
        # Number of missing positive integers before arr[mid]
        # is arr[mid] - (mid + 1)
        missing_count = arr[mid] - (mid + 1)

        if missing_count < k:
            # The k-th missing number is in the right half or is larger than arr[mid]
            left = mid + 1
        else:
            # The k-th missing number is in the left half or is arr[mid] itself
            right = mid - 1
    
    # At the end of binary search, 'left' will be the smallest index
    # such that arr[left] - (left + 1) >= k.
    # The number of missing integers up to arr[left-1] is (arr[left-1] - left).
    # The k-th missing number is (arr[left-1] + k - (arr[left-1] - left))
    # which simplifies to (left + k).
    # If left == 0, it means all k missing numbers are before arr.
    # So the k-th missing number is k.
    
    return left + k
\end{minted}

\noindent\textbf{Problem: Kth Element in Two Sorted Arrays}
\begin{minted}[
bgcolor=bgcolor6,
frame=lines,
framesep=5mm,
rulecolor=\color{black},
linenos,
numbersep=5pt,
fontsize=\normalsize
]{python}
def find_kth_element(nums1: List[int], nums2: List[int], k: int) -> int:
    """
    Finds the k-th smallest element in two sorted arrays.
    Time Complexity: O(log(min(len(nums1), len(nums2))))
    """
    n1, n2 = len(nums1), len(nums2)

    # Ensure nums1 is the shorter array for efficiency
    if n1 > n2:
        return find_kth_element(nums2, nums1, k)

    # Base cases
    if n1 == 0:
        return nums2[k - 1]
    if k == 1:
        return min(nums1[0], nums2[0])

    # Divide k into two parts
    # i: number of elements to consider from nums1
    # j: number of elements to consider from nums2
    i = min(k // 2, n1) # Take k/2 from nums1, or all if k/2 is too much
    j = k - i           # Take remaining from nums2

    if nums1[i - 1] < nums2[j - 1]:
        # Discard first i elements from nums1, they are too small
        return find_kth_element(nums1[i:], nums2, k - i)
    else:
        # Discard first j elements from nums2, they are too small
        return find_kth_element(nums1, nums2[j:], k - j)
\end{minted}

\noindent\textbf{Problem: Longest Repeating Substring (Binary Search)} 
\\ 

Note: You’re looking for the maximum length \(L\) such that a duplicate of length \(L\) exists. The key facts that make binary-search valid are:

1.  Monotonicity of the predicate  
    Define  

      P(L) = “there is a length-\(L\) substring that appears ≥2 times.”
    \\
    Observe:
    -  If \(P(L)\) is true, then for any \(k < L\), \(P(k)\) must also be true: taking the two length-\(L\) repeats and truncating to length \(k\) still gives two equal substrings.
    -  If \(P(L)\) is false, then for any \(k > L\), \(P(k)\) is also false: you can’t suddenly create a longer repeat if no repeat of length \(L\) existed.

    That makes \(P(L)\) a “true–true–…–true–false–false–…–false” step function over \(L\in[0,n]\). Binary-search zeroes in on the transition point in \(\log n\) checks, rather than scanning all \(n\) lengths.

2.  Cost comparison  
    -  Each check \(P(L)\) runs in \(O(n)\) via rolling-hash.  
    -  Scanning \(L=1,2,\dots,n\) linearly costs \(O(n)\times O(n)=O(n^2)\).  
    -  Binary-search costs \(O(\log n)\) checks × \(O(n)\) each = \(O(n\log n)\).



In short: because the “can-repeat?” test is monotonic in \(L\), you can skip huge swaths of impossible (or trivially possible) lengths in one step. That’s exactly the scenario where binary-search on the answer space shines, giving you an \(O(n\log n)\) algorithm instead of \(O(n^2)\).
\begin{minted}[
bgcolor=bgcolor7,
frame=lines,
framesep=5mm,
rulecolor=\color{black},
linenos,
numbersep=5pt,
fontsize=\normalsize
]{python}
def check_for_duplicate_substring_with_rolling_hash(s: str, length: int) -> bool:
    """
    Helper function to check if there is any repeating substring of 'length'
    in string 's' using a rolling hash.
    Time Complexity: O(N)
    """
    n = len(s)
    if length == 0:
        return True # Empty string always repeats
    if length > n:
        return False

    base = 26 # or a prime like 31, 53
    mod = 10**9 + 7 # A large prime modulus

    current_hash = 0
    power = 1

    # Calculate hash for the first substring
    for i in range(length):
        current_hash = (current_hash * base + (ord(s[i]) - ord('a') + 1)) % mod
        if i < length - 1:
            power = (power * base) % mod
    
    seen_hashes = {current_hash}

    # Roll the hash for subsequent substrings
    for i in range(length, n):
        # Remove s[i - length] and add s[i]
        current_hash = (current_hash - (ord(s[i - length]) - ord('a') + 1) * power) % mod
        current_hash = (current_hash * base + (ord(s[i]) - ord('a') + 1)) % mod
        if current_hash < 0: # Ensure hash stays positive after modulo
            current_hash += mod
        
        if current_hash in seen_hashes:
            return True
        seen_hashes.add(current_hash)
    
    return False

def longest_repeating_substring(s: str) -> int:
    """
    Finds the length of the longest repeating substring in 's'.
    Uses binary search on the possible lengths.
    Time Complexity: O(N log N) (due to O(N) check function)
    """
    n = len(s)
    left, right = 0, n - 1 # Search space for length of repeating substring
    longest = 0

    while left <= right:
        mid_length = left + (right - left) // 2
        if check_for_duplicate_substring_with_rolling_hash(s, mid_length):
            longest = mid_length
            left = mid_length + 1 # Try for a longer repeating substring
        else:
            right = mid_length - 1 # Mid_length is too long, try smaller
    return longest
\end{minted}
% \end{document}
