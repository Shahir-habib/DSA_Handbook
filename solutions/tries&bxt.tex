% \documentclass[a4paper,10pt]{article}
% \usepackage[utf8]{inputenc}
% \usepackage{geometry}
% \usepackage[table]{xcolor}
% \usepackage{colortbl}
% \usepackage{color,soul}
% \geometry{margin=0.8in}
% \usepackage{xcolor}
% \usepackage{tikz}
% \usepackage{minted}
% \definecolor{bgcolor}{rgb}{0.8, 0.9, 0.5} % 
% \definecolor{bgcolor1}{rgb}{0.95, 0.95, 0.95} % Light Gray
% \definecolor{bgcolor2}{rgb}{0.85, 0.92, 1.0}  % Soft Blue
% \definecolor{bgcolor3}{rgb}{0.9, 0.85, 1.0}   % Light Purple
% \definecolor{bgcolor4}{rgb}{0.95, 0.88, 0.76} % Warm Beige
% \definecolor{bgcolor5}{rgb}{0.8, 0.95, 0.8}   % Gentle Green
% \definecolor{bgcolor6}{rgb}{1.0, 0.87, 0.87}  % Pastel Red
% \definecolor{bgcolor7}{rgb}{0.86, 0.93, 0.83} % Mint Green
% \definecolor{bgcolor8}{rgb}{0.98, 0.85, 0.94} % Soft Pink
% \definecolor{bgcolor9}{rgb}{0.87, 0.94, 0.98} % Sky Blue
% \definecolor{bgcolor10}{rgb}{0.96, 0.96, 0.82} % Pale Yellow
% 
% \begin{document}
\section*{Tries and Binary Index Tree Problem Solutions}


\noindent\textbf{Problem: Tries Operation}
\begin{minted}[
bgcolor=bgcolor3,
frame=lines,
framesep=5mm,
rulecolor=\color{black},
linenos,
numbersep=5pt,
fontsize=\normalsize
]{python}  
class TrieNode:
    def __init__(self):
        self.children = {}
        self.is_end_of_word = False

class Trie:
    def __init__(self):
        self.root = TrieNode()

     def insert(self, word):
        node = self.root
        for char in word:
            if char not in node.children:
                node.children[char] = TrieNode()
            node = node.children[char]
        node.is_end_of_word = True

    def search(self, word):
        node = self.root
        for char in word:
            if char not in node.children:
                return False
            node = node.children[char]
        return node.is_end_of_word

    def delete(self, word):
        def _delete(node, word, depth):
            if depth == len(word):
                if not node.is_end_of_word:
                    return False
                node.is_end_of_word = False
                return len(node.children) == 0
            char = word[depth]
            if char not in node.children:
                return False
            should_delete = _delete(node.children[char], word, depth + 1)
            if should_delete:
                del node.children[char]
                return not node.is_end_of_word and len(node.children) == 0
            return False

        _delete(self.root, word, 0)
\end{minted}
\noindent\textbf{Auto Complete Suggestions}
\begin{minted}[
bgcolor=bgcolor1,
frame=lines,
framesep=5mm,
rulecolor=\color{black},
linenos,
numbersep=5pt,
fontsize=\normalsize
]{python}
# Method to search the Trie for a prefix and return the node where prefix ends
    def _search_prefix_node(self, prefix):
        node = self.root
        for char in prefix:
            if char in node.children:
                node = node.children[char]
            else:
                # If prefix is not in trie, return None
                return None
        return node

    # Recursive helper to collect all words starting from a given node
    def _collect_all_words(self, node, prefix, results):
        if node.is_end_of_word:
            results.append(prefix)
        for char, next_node in node.children.items():
            # Continue the search with extended prefix
            self._collect_all_words(next_node, prefix + char, results)

    # Public method to get autocomplete suggestions for a prefix
    def autocomplete(self, prefix):
        results = []
        # Find the node where the prefix ends
        prefix_node = self._search_prefix_node(prefix)
        if not prefix_node:
            return []  # No suggestions if prefix doesn't exist
        # Collect all words from that node
        self._collect_all_words(prefix_node, prefix, results)
        return results
\end{minted}
\noindent\textbf{Problem: Count Distinct Rows in Binary Matrix}
\begin{minted}[
bgcolor=bgcolor5,
frame=lines,
framesep=5mm,
rulecolor=\color{black},
linenos,
numbersep=5pt,
fontsize=\normalsize
]{python}
class Trie:
    def __init__(self):
        self.root = TrieNode()

    # Insert a row into the Trie
    # Returns True if it's a new row, False if duplicate
    def insert(self, row):
        node = self.root
        is_new = False  # Flag to track if row is unique

        for bit in row:
            # If the child node for current bit doesn't exist, create it
            if node.children[bit] is None:
                node.children[bit] = TrieNode()
                is_new = True  # Since a new node was created, it's a new path
            node = node.children[bit]

        # If end-of-row flag is not yet set, it's a new row
        if not node.is_end_of_row:
            node.is_end_of_row = True
            is_new = True

        return is_new

# Function to count distinct rows using Trie
def count_distinct_rows(matrix):
    trie = Trie()
    count = 0

    for row in matrix:
        if trie.insert(row):
            count += 1  # Row is distinct

    return count
\end{minted}

\noindent\textbf{Problem: Maximum Xor of 2 Elements in Array}
\begin{minted}[
bgcolor=bgcolor6,
frame=lines,
framesep=5mm,
rulecolor=\color{black},
linenos,
numbersep=5pt,
fontsize=\normalsize
]{python}
# Define the number of bits (32 for standard int)
INT_SIZE = 32

# Trie Node for storing binary digits
class TrieNode:
    def __init__(self):
        self.children = [None, None]  # 0 and 1

# Trie structure for inserting and querying
class Trie:
    def __init__(self):
        self.root = TrieNode()

    # Insert number into the Trie
    def insert(self, num):
        node = self.root
        for i in reversed(range(INT_SIZE)):  # From MSB to LSB
            bit = (num >> i) & 1
            if node.children[bit] is None:
                node.children[bit] = TrieNode()
            node = node.children[bit]

    # Find max XOR of num with elements already in Trie
    def max_xor(self, num):
        node = self.root
        max_xor = 0
        for i in reversed(range(INT_SIZE)):
            bit = (num >> i) & 1
            # Prefer opposite bit to maximize XOR
            toggled_bit = 1 - bit
            if node.children[toggled_bit]:
                max_xor |= (1 << i)  # Add this bit to result
                node = node.children[toggled_bit]
            else:
                node = node.children[bit]
        return max_xor

# Main function to find max XOR of any two numbers
def find_max_xor(arr):
    trie = Trie()
    max_result = 0

    # Insert the first number before comparisons
    trie.insert(arr[0])

    for num in arr[1:]:
        # Check current max XOR
        current_xor = trie.max_xor(num)
        max_result = max(max_result, current_xor)
        # Insert current number into Trie
        trie.insert(num)

    return max_result
\end{minted}

\noindent\textbf{Problem: Distinct Substring in a String}\\
Every substring of a string is a prefix of some suffix — but not every substring is a prefix of the original string.
\begin{minted}[
bgcolor=bgcolor10,
frame=lines,
framesep=5mm,
rulecolor=\color{black},
linenos,
numbersep=5pt,
fontsize=\normalsize
]{python}
# Trie Node class
class TrieNode:
    def __init__(self):
        self.children = {}  # Dictionary for character links

# Trie class to insert and count distinct substrings
class Trie:
    def __init__(self):
        self.root = TrieNode()
        self.count = 0  # To count new nodes (distinct substrings)

    # Insert a suffix into the Trie and count new substrings
    def insert_suffix(self, suffix):
        node = self.root
        for char in suffix:
            # If char is not present, it's a new substring
            if char not in node.children:
                node.children[char] = TrieNode()
                self.count += 1  # New substring added
            node = node.children[char]

# Function to count all distinct substrings of a string
def count_distinct_substrings(s):
    trie = Trie()

    # Insert all suffixes into the Trie
    for i in range(len(s)):
        suffix = s[i:]
        trie.insert_suffix(suffix)

    # Total distinct substrings = total nodes created in Trie
    return trie.count
\end{minted}
\noindent\textbf{Problem: Binary Index Tree(tushar)}
\begin{minted}[
bgcolor=bgcolor5,
frame=lines,
framesep=5mm,
rulecolor=\color{black},
linenos,
numbersep=5pt,
fontsize=\normalsize
]{python}

\end{minted}
% \end{document}
