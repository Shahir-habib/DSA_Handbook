% \documentclass[a4paper,10pt]{article}
% \usepackage[utf8]{inputenc}
% \usepackage{geometry}
% \usepackage[table]{xcolor}
% \usepackage{colortbl}
% \usepackage{color,soul}
% \geometry{margin=0.8in}
% \usepackage{xcolor}
% \usepackage{tikz}
% \usepackage{minted}
% \definecolor{bgcolor}{rgb}{0.8, 0.9, 0.5} % 
% \definecolor{bgcolor1}{rgb}{0.95, 0.95, 0.95} % Light Gray
% \definecolor{bgcolor2}{rgb}{0.85, 0.92, 1.0}  % Soft Blue
% \definecolor{bgcolor3}{rgb}{0.9, 0.85, 1.0}   % Light Purple
% \definecolor{bgcolor4}{rgb}{0.95, 0.88, 0.76} % Warm Beige
% \definecolor{bgcolor5}{rgb}{0.8, 0.95, 0.8}   % Gentle Green
% \definecolor{bgcolor6}{rgb}{1.0, 0.87, 0.87}  % Pastel Red
% \definecolor{bgcolor7}{rgb}{0.86, 0.93, 0.83} % Mint Green
% \definecolor{bgcolor8}{rgb}{0.98, 0.85, 0.94} % Soft Pink
% \definecolor{bgcolor9}{rgb}{0.87, 0.94, 0.98} % Sky Blue
% \definecolor{bgcolor10}{rgb}{0.96, 0.96, 0.82} % Pale Yellow
% 
% % Define a custom background colo
% \begin{document}
\section*{Sorting Problem Solutions}
\noindent\textbf{Problem: Insertion Sort}
\begin{minted}[bgcolor=bgcolor1,frame=lines,framesep=5mm,rulecolor=\color{black},linenos,numbersep=5pt,fontsize=\normalsize]{python}
def insertion_sort(arr):
    """
    In-place insertion sort.
    Time Complexity: O(n^2)
    """
    for i in range(1, len(arr)):
        key = arr[i]
        j = i - 1
        while j >= 0 and arr[j] > key:
            arr[j+1] = arr[j]
            j -= 1
        arr[j+1] = key
\end{minted}

\noindent\textbf{Problem: Bubble Sort}
\begin{minted}[bgcolor=bgcolor2,frame=lines,framesep=5mm,rulecolor=\color{black},linenos,numbersep=5pt,fontsize=\normalsize]{python}
def bubble_sort(arr):
    """
    In-place bubble sort with early exit.
    Time Complexity: O(n^2)
    """
    n = len(arr)
    for i in range(n):
        swapped = False
        for j in range(0, n-1-i):
            if arr[j] > arr[j+1]:
                arr[j], arr[j+1] = arr[j+1], arr[j]
                swapped = True
        if not swapped:
            break
\end{minted}

\noindent\textbf{Problem: Selection Sort}
\begin{minted}[bgcolor=bgcolor3,frame=lines,framesep=5mm,rulecolor=\color{black},linenos,numbersep=5pt,fontsize=\normalsize]{python}
def selection_sort(arr):
    """
    In-place selection sort.
    Time Complexity: O(n^2)
    """
    n = len(arr)
    for i in range(n):
        min_idx = i
        for j in range(i+1, n):
            if arr[j] < arr[min_idx]:
                min_idx = j
        arr[i], arr[min_idx] = arr[min_idx], arr[i]
\end{minted}

\noindent\textbf{Problem: Merge Function (Merge Sort)}
\begin{minted}[bgcolor=bgcolor4,frame=lines,framesep=5mm,rulecolor=\color{black},linenos,numbersep=5pt,fontsize=\normalsize]{python}
def merge(left, right):
    """
    Merge two sorted lists.
    Time Complexity: O(n)
    """
    i = j = 0
    merged = []
    while i < len(left) and j < len(right):
        if left[i] <= right[j]:
            merged.append(left[i]); i += 1
        else:
            merged.append(right[j]); j += 1
    merged.extend(left[i:] or right[j:])
    return merged
\end{minted}

\noindent\textbf{Problem: Merge Sort}
\begin{minted}[bgcolor=bgcolor5,frame=lines,framesep=5mm,rulecolor=\color{black},linenos,numbersep=5pt,fontsize=\normalsize]{python}
def merge_sort(arr):
    """
    Recursive merge sort.
    Time Complexity: O(n log n)
    """
    if len(arr) <= 1:
        return arr
    mid = len(arr)//2
    left = merge_sort(arr[:mid])
    right = merge_sort(arr[mid:])
    return merge(left, right)
\end{minted}

\noindent\textbf{Problem: Count Inversions in Array}
\begin{minted}[bgcolor=bgcolor6,frame=lines,framesep=5mm,rulecolor=\color{black},linenos,numbersep=5pt,fontsize=\normalsize]{python}
def count_inversions(arr):
    """
    Count inversions using modified merge sort.
    Time Complexity: O(n log n)
    """
    def _count(a):
        if len(a) <= 1:
            return a, 0
        mid = len(a)//2
        left, inv_l = _count(a[:mid])
        right, inv_r = _count(a[mid:])
        merged, inv_m = [], 0
        i=j=0
        while i < len(left) and j < len(right):
            if left[i] <= right[j]:
                merged.append(left[i]); i+=1
            else:
            # right[j] < left[i], so it is “inverted” with
            # EVERY element left[i], left[i+1], …, left[-1]
                merged.append(right[j]); j+=1
                inv_m += len(left)-i
        merged += left[i:]+right[j:]
        return merged, inv_l + inv_r + inv_m
    return _count(arr)[1]
\end{minted}

\noindent\textbf{Problem: Partitioning of Array (Lomuto / Hoare)}
\begin{minted}[bgcolor=bgcolor7,frame=lines,framesep=5mm,rulecolor=\color{black},linenos,numbersep=5pt,fontsize=\normalsize]{python}
def lomuto_partition(arr, low, high):
    """
    Lomuto partition scheme.
    """
    pivot = arr[high]
    i = low
    for j in range(low, high):
        if arr[j] < pivot:
            arr[i], arr[j] = arr[j], arr[i]
            i += 1
    arr[i], arr[high] = arr[high], arr[i]
    return i

def hoare_partition(arr, low, high):
    """
    Hoare partition scheme.
    """
    pivot = arr[(low+high)//2]
    i, j = low-1, high+1
    while True:
        i += 1
        while arr[i] < pivot:
            i += 1
        j -= 1
        while arr[j] > pivot:
            j -= 1
        if i >= j:
            return j
        arr[i], arr[j] = arr[j], arr[i]
\end{minted}

\noindent\textbf{Problem: Quick Sort}
\begin{minted}[bgcolor=bgcolor8,frame=lines,framesep=5mm,rulecolor=\color{black},linenos,numbersep=5pt,fontsize=\normalsize]{python}
def quick_sort(arr, low=0, high=None):
    """
    In-place quick sort (Lomuto).
    Time Complexity: O(n log n) avg
    """
    if high is None:
        high = len(arr)-1
    if low < high:
        p = lomuto_partition(arr, low, high)
        quick_sort(arr, low, p-1)
        quick_sort(arr, p+1, high)
\end{minted}

\noindent\textbf{Problem: Cycle Sort}
Why it Guarantees Sorted ?  Cycle sort decomposes the permutation of your array into disjoint cycles, then rotates each cycle into place. Every element is guaranteed to land in its unique “rank” index. Once each cycle is done, no further swaps can disturb the sorted prefix. That establishes correctness.
\begin{minted}[bgcolor=bgcolor9,frame=lines,framesep=5mm,rulecolor=\color{black},linenos,numbersep=5pt,fontsize=\normalsize]{python}
def cycle_sort(arr):
    """
    In-place cycle sort, minimizes the number of writes.
    Time Complexity: O(n^2)
    Returns the total number of writes performed.
    """
    writes = 0
    n = len(arr)

    # We will go through each index as the start of a cycle
    for cycle_start in range(n - 1):
        # The item we want to place into its correct position
        item = arr[cycle_start]
        pos = cycle_start

        # 1) Find where to put the item by counting
        #    how many elements smaller than it exist to its right
        for i in range(cycle_start + 1, n):
            if arr[i] < item:
                pos += 1

        # If the item is already in the correct position, move on
        if pos == cycle_start:
            continue

        # 2) Skip over duplicates to find a free slot
        while arr[pos] == item:
            pos += 1

        # 3) Put the item there (this is a write)
        arr[pos], item = item, arr[pos]
        writes += 1

        # 4) Now rotate the rest of the cycle
        #    until we come back to the starting slot
        while pos != cycle_start:
            pos = cycle_start
            # Find the correct position for the new item
            for i in range(cycle_start + 1, n):
                if arr[i] < item:
                    pos += 1
            # Skip duplicates again
            while arr[pos] == item:
                pos += 1
            # Swap and count the write
            arr[pos], item = item, arr[pos]
            writes += 1

    return writes
\end{minted}

\noindent\textbf{Problem: Heap Sort}
\begin{minted}[bgcolor=bgcolor10,frame=lines,framesep=5mm,rulecolor=\color{black},linenos,numbersep=5pt,fontsize=\normalsize]{python}
def heapify(arr, n, i):
    largest = i
    l, r = 2*i+1, 2*i+2
    if l < n and arr[l] > arr[largest]:
        largest = l
    if r < n and arr[r] > arr[largest]:
        largest = r
    if largest != i:
        arr[i], arr[largest] = arr[largest], arr[i]
        heapify(arr, n, largest)

def heap_sort(arr):
    """
    In-place heap sort.
    Time Complexity: O(n log n)
    """
    n = len(arr)
    for i in range(n//2-1, -1, -1):
        heapify(arr, n, i)
    for i in range(n-1, 0, -1):
        arr[0], arr[i] = arr[i], arr[0]
        heapify(arr, i, 0)
\end{minted}

\noindent\textbf{Problem: Counting Sort}
\begin{minted}[bgcolor=bgcolor1,frame=lines,framesep=5mm,rulecolor=\color{black},linenos,numbersep=5pt,fontsize=\normalsize]{python}
def counting_sort(arr, k=None):
    """
    Non-comparison sort for integers 0..k.
    Time Complexity: O(n+k)
    """
    if not arr:
        return []
    if k is None:
        k = max(arr)
    count = [0]*(k+1)
    for x in arr:
        count[x] += 1
    for i in range(1, k+1):
        count[i] += count[i-1]
    output = [0]*len(arr)
    for x in reversed(arr):     # Traversing right to left for stable sorting
        count[x] -= 1
        output[count[x]] = x
    return output
\end{minted}

\noindent\textbf{Problem: Radix Sort}
\begin{minted}[bgcolor=bgcolor2,frame=lines,framesep=5mm,rulecolor=\color{black},linenos,numbersep=5pt,fontsize=\normalsize]{python}
def radix_sort(arr):
    """
    Least significant digit radix sort for non-negative ints.
    Time Complexity: O(d*(n+ b))
    """
    if not arr:
        return []
    max_val = max(arr)
    exp = 1
    while exp <= max_val:
        buckets = [[] for _ in range(10)]
        for x in arr:
            buckets[(x//exp) % 10].append(x)
        arr = [y for bucket in buckets for y in bucket]
        exp *= 10
    return arr
\end{minted}

\noindent\textbf{Problem: Bucket Sort}
\begin{minted}[bgcolor=bgcolor3,frame=lines,framesep=5mm,rulecolor=\color{black},linenos,numbersep=5pt,fontsize=\normalsize]{python}
def bucket_sort(arr, bucket_size=5):
    """
    Bucket sort for uniformly distributed values.
    Time Complexity: O(n + k log k)
    """
    if not arr:
        return []
    min_val, max_val = min(arr), max(arr)
    bucket_count = (max_val - min_val)//bucket_size + 1
    buckets = [[] for _ in range(bucket_count)]
    for x in arr:
        buckets[(x - min_val)//bucket_size].append(x)
    arr = []
    for b in buckets:
        arr.extend(sorted(b))
    return arr
\end{minted}

\noindent\textbf{Problem: Kth Smallest Element in Array}
\begin{minted}[bgcolor=bgcolor4,frame=lines,framesep=5mm,rulecolor=\color{black},linenos,numbersep=5pt,fontsize=\normalsize]{python}
import random
def kth_smallest(arr, k):
    """
    Quickselect for k-th smallest (1-based).
    Time Complexity: O(n) avg
    """
    def select(lo, hi, k):
        if lo == hi:
            return arr[lo]
        pivot = arr[random.randint(lo, hi)]
        lows = [x for x in arr[lo:hi+1] if x < pivot]
        highs = [x for x in arr[lo:hi+1] if x > pivot]
        mids  = [x for x in arr[lo:hi+1] if x == pivot]
        if k <= len(lows):
            return select(lo, lo+len(lows)-1, k)
        elif k > len(lows) + len(mids):
            return select(lo+len(lows)+len(mids), hi, k-len(lows)-len(mids))
        else:
            return pivot
    return select(0, len(arr)-1, k)
\end{minted}

\noindent\textbf{Problem: Chocolate Distribution (Min Diff)}
\begin{minted}[bgcolor=bgcolor5,frame=lines,framesep=5mm,rulecolor=\color{black},linenos,numbersep=5pt,fontsize=\normalsize]{python}
def min_diff_chocolate(packets, m):
    """
    Find min diff between max and min of any m packets.
    Time Complexity: O(n log n)
    """
    if m == 0 or len(packets) < m:
        return 0
    packets.sort()
    return min(packets[i+m-1] - packets[i] for i in range(len(packets)-m+1))
\end{minted}

\noindent\textbf{Problem: Sort 3 Types of Elements (Dutch National Flag)}
\begin{minted}[bgcolor=bgcolor6,frame=lines,framesep=5mm,rulecolor=\color{black},linenos,numbersep=5pt,fontsize=\normalsize]{python}
def dutch_national_flag(arr):
    """
    Sort array of 0s,1s,2s in one pass.
    Time Complexity: O(n)
    """
    low = mid = 0
    high = len(arr)-1
    while mid <= high:
        if arr[mid] == 0:
            arr[low], arr[mid] = arr[mid], arr[low]
            low += 1; mid += 1
        elif arr[mid] == 1:
            mid += 1
        else:
            arr[mid], arr[high] = arr[high], arr[mid]
            high -= 1
\end{minted}

\noindent\textbf{Problem: Merge Overlapping Intervals}
\begin{minted}[bgcolor=bgcolor7,frame=lines,framesep=5mm,rulecolor=\color{black},linenos,numbersep=5pt,fontsize=\normalsize]{python}
def merge_intervals(intervals):
    """
    Merge all overlapping intervals.
    Time Complexity: O(n log n)
    """
    if not intervals:
        return []
    intervals.sort(key=lambda x: x[0])
    merged = [intervals[0]]
    for start, end in intervals[1:]:
        last_end = merged[-1][1]
        if start <= last_end:
            merged[-1][1] = max(last_end, end)
        else:
            merged.append([start, end])
    return merged
\end{minted}

\noindent\textbf{Problem: Meeting Maximum Guests}
\begin{minted}[bgcolor=bgcolor8,frame=lines,framesep=5mm,rulecolor=\color{black},linenos,numbersep=5pt,fontsize=\normalsize]{python}
def max_guests(arrivals: List[int], departures: List[int]) -> int:
    """
    Finds the maximum number of guests present at the same time
    given lists of arrival and departure times.
    Time Complexity: O(n log n)
    """
    # 1) Build an “events” list: +1 for each arrival, -1 for each departure
    events: List[Tuple[int,int]] = []
    for t in arrivals:
        events.append((t, 1))    # when a guest arrives at t, delta = +1
    for t in departures:
        events.append((t, -1))   # when a guest leaves at t, delta = -1

    # 2) Sort by time; if two events share the same time,
    #    process arrivals (+1) before departures (–1) so we count correctly
    events.sort(key=lambda x: (x[0], -x[1]))

    curr = 0   # current number of guests at the sweep line
    maxg = 0   # record of the maximum seen

    # 3) Sweep through sorted events, updating the count
    for time, delta in events:
        curr += delta            # add or remove a guest
        maxg = max(maxg, curr)   # update peak if needed

    return maxg
\end{minted}
% \end{document}
