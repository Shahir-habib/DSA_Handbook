% \documentclass[a4paper,10pt]{article}
% \usepackage[utf8]{inputenc}
% \usepackage{geometry}
% \usepackage[table]{xcolor}
% \usepackage{colortbl}
% \usepackage{color,soul}
% \geometry{margin=0.8in}
% \usepackage{xcolor}
% \usepackage{tikz}
% \usepackage{minted}
% \definecolor{bgcolor}{rgb}{0.8, 0.9, 0.5} % 
% \definecolor{bgcolor1}{rgb}{0.95, 0.95, 0.95} % Light Gray
% \definecolor{bgcolor2}{rgb}{0.85, 0.92, 1.0}  % Soft Blue
% \definecolor{bgcolor3}{rgb}{0.9, 0.85, 1.0}   % Light Purple
% \definecolor{bgcolor4}{rgb}{0.95, 0.88, 0.76} % Warm Beige
% \definecolor{bgcolor5}{rgb}{0.8, 0.95, 0.8}   % Gentle Green
% \definecolor{bgcolor6}{rgb}{1.0, 0.87, 0.87}  % Pastel Red
% \definecolor{bgcolor7}{rgb}{0.86, 0.93, 0.83} % Mint Green
% \definecolor{bgcolor8}{rgb}{0.98, 0.85, 0.94} % Soft Pink
% \definecolor{bgcolor9}{rgb}{0.87, 0.94, 0.98} % Sky Blue
% \definecolor{bgcolor10}{rgb}{0.96, 0.96, 0.82} % Pale Yellow
% 
% \begin{document}
% Rat in a Maze
\noindent\textbf{Problem: Rat in a Maze}
\begin{minted}[
bgcolor=bgcolor10,
frame=lines,
framesep=5mm,
rulecolor=\color{black},
linenos,
numbersep=5pt,
fontsize=\normalsize
]{python}
def rat_in_maze(maze: List[List[int]]) -> List[List[int]]:
    """
    Finds path for rat from (0,0) to (n-1,n-1) in maze.
    Time Complexity: O(4^(n^2)), Space Complexity: O(n^2)
    """
    n = len(maze)
    path = [[0]*n for _ in range(n)]
    
    def is_safe(x, y):
        return 0 <= x < n and 0 <= y < n and maze[x][y] == 1
    
    def backtrack(x, y):
        if x == n-1 and y == n-1:
            path[x][y] = 1
            return True
        if is_safe(x, y):
            path[x][y] = 1
            # Move right
            if backtrack(x, y+1):
                return True
            # Move down
            if backtrack(x+1, y):
                return True
            path[x][y] = 0  # Backtrack
        return False
        
    backtrack(0, 0)
    return path
\end{minted}

% N-Queen Problem
\noindent\textbf{Problem: N-Queen Problem}
\begin{minted}[
bgcolor=bgcolor2,
frame=lines,
framesep=5mm,
rulecolor=\color{black},
linenos,
numbersep=5pt,
fontsize=\normalsize
]{python}
def solve_n_queens(n: int) -> List[List[str]]:
    """
    Places n queens on nxn chessboard such that no two queens attack each other.
    Time Complexity: O(n!), Space Complexity: O(n^2)
    """
    def is_safe(row, col):
        # Check column
        for i in range(row):
            if board[i][col] == 'Q':
                return False
        # Check upper left diagonal
        for i, j in zip(range(row-1, -1, -1), range(col-1, -1, -1)):
            if board[i][j] == 'Q':
                return False
        # Check upper right diagonal
        for i, j in zip(range(row-1, -1, -1), range(col+1, n)):
            if board[i][j] == 'Q':
                return False
        return True
        
    def backtrack(row):
        if row == n:
            result.append(["".join(r) for r in board])
            return
        for col in range(n):
            if is_safe(row, col):
                board[row][col] = 'Q'
                backtrack(row+1)
                board[row][col] = '.'
                
    board = [['.']*n for _ in range(n)]
    result = []
    backtrack(0)
    return result
\end{minted}

% Permutation without Forbidden Substring
\noindent\textbf{Problem: Permutation without Forbidden Substring}
\begin{minted}[
bgcolor=bgcolor7,
frame=lines,
framesep=5mm,
rulecolor=\color{black},
linenos,
numbersep=5pt,
fontsize=\normalsize
]{python}
def valid_permutations(s: str, forbidden: str) -> List[str]:
    """
    Generates permutations of s that don't contain forbidden substring.
    Time Complexity: O(n! * n), Space Complexity: O(n)
    """
    def backtrack(start):
        if start == len(chars):
            perm = "".join(chars)
            if forbidden not in perm:
                result.append(perm)
            return
        for i in range(start, len(chars)):
            chars[start], chars[i] = chars[i], chars[start]
            backtrack(start+1)
            chars[start], chars[i] = chars[i], chars[start]
            
    result = []
    chars = list(s)
    backtrack(0)
    return result
\end{minted}

% Sudoku Solver
\noindent\textbf{Problem: Sudoku Solver}
\begin{minted}[
bgcolor=bgcolor5,
frame=lines,
framesep=5mm,
rulecolor=\color{black},
linenos,
numbersep=5pt,
fontsize=\normalsize
]{python}
def solve_sudoku(board: List[List[str]]) -> None:
    """
    Solves 9x9 Sudoku puzzle in-place using backtracking.
    Time Complexity: O(9^(n^2)), Space Complexity: O(1)
    """
    def is_valid(row, col, num):
        # Check row
        for i in range(9):
            if board[row][i] == num:
                return False
        # Check column
        for i in range(9):
            if board[i][col] == num:
                return False
        # Check 3x3 box
        start_row, start_col = 3 * (row // 3), 3 * (col // 3)
        for i in range(3):
            for j in range(3):
                if board[start_row+i][start_col+j] == num:
                    return False
        return True
        
    def backtrack():
        for i in range(9):
            for j in range(9):
                if board[i][j] == '.':
                    for num in map(str, range(1, 10)):
                        if is_valid(i, j, num):
                            board[i][j] = num
                            if backtrack():
                                return True
                            board[i][j] = '.'  # Backtrack
                    return False
        return True  # All cells filled
        
    backtrack()
\end{minted}

% Combination Sum
\noindent\textbf{Problem: Combination Sum}
\begin{minted}[
bgcolor=bgcolor8,
frame=lines,
framesep=5mm,
rulecolor=\color{black},
linenos,
numbersep=5pt,
fontsize=\normalsize
]{python}
def combination_sum(candidates: List[int], target: int) -> List[List[int]]:
    """
    Finds all combinations that sum to target (reuse allowed).
    Time Complexity: O(2^target), Space Complexity: O(target)
    """
    def backtrack(start, path, current_sum):
        if current_sum == target:
            result.append(path[:])
            return
        if current_sum > target:
            return
        for i in range(start, len(candidates)):
            path.append(candidates[i])
            backtrack(i, path, current_sum + candidates[i])
            path.pop()
            
    result = []
    backtrack(0, [], 0)
    return result
\end{minted}

% Generate Parentheses
\noindent\textbf{Problem: Generate Parentheses}
\begin{minted}[
bgcolor=bgcolor1,
frame=lines,
framesep=5mm,
rulecolor=\color{black},
linenos,
numbersep=5pt,
fontsize=\normalsize
]{python}
def generate_parenthesis(n: int) -> List[str]:
    """
    Generates all valid n pairs of parentheses.
    Time Complexity: O(4^n/root(N)), Space Complexity: O(n)
    """
    def backtrack(s, left, right):
        if len(s) == 2*n:
            result.append(s)
            return
        if left < n:
            backtrack(s+'(', left+1, right)
        if right < left:
            backtrack(s+')', left, right+1)
            
    result = []
    backtrack('', 0, 0)
    return result
\end{minted}

% Subset Sum = K (Print All Subsets)
\noindent\textbf{Problem: Subset Sum = K (Print All Subsets)}
\begin{minted}[
bgcolor=bgcolor9,
frame=lines,
framesep=5mm,
rulecolor=\color{black},
linenos,
numbersep=5pt,
fontsize=\normalsize
]{python}
def subset_sum(nums: List[int], target: int) -> List[List[int]]:
    """
    Finds all subsets that sum to target.
    Time Complexity: O(2^n), Space Complexity: O(n)
    """
    def backtrack(start, path, current_sum):
        if current_sum == target:
            result.append(path[:])
            return
        if start >= len(nums) or current_sum > target:
            return
        for i in range(start, len(nums)):
            path.append(nums[i])
            backtrack(i+1, path, current_sum + nums[i])
            path.pop()
            
    result = []
    nums.sort()
    backtrack(0, [], 0)
    return result
\end{minted}

% Combination Sum II
\noindent\textbf{Problem: Combination Sum II}
\begin{minted}[
bgcolor=bgcolor6,
frame=lines,
framesep=5mm,
rulecolor=\color{black},
linenos,
numbersep=5pt,
fontsize=\normalsize
]{python}
def combination_sum2(candidates: List[int], target: int) -> List[List[int]]:
    """
    Finds unique combinations that sum to target (each used once).
    Time Complexity: O(2^n), Space Complexity: O(n)
    """
    def backtrack(start, path, current_sum):
        if current_sum == target:
            result.append(path[:])
            return
        if current_sum > target:
            return
        for i in range(start, len(candidates)):
            # Skip duplicates
            if i > start and candidates[i] == candidates[i-1]:
                continue
            path.append(candidates[i])
            backtrack(i+1, path, current_sum + candidates[i])
            path.pop()
            
    result = []
    candidates.sort()
    backtrack(0, [], 0)
    return result
\end{minted}

% Subset Sum II (All Unique Subsets)
\noindent\textbf{Problem: Subset Sum II (All Unique Subsets)}
\begin{minted}[
bgcolor=bgcolor4,
frame=lines,
framesep=5mm,
rulecolor=\color{black},
linenos,
numbersep=5pt,
fontsize=\normalsize
]{python}
def subsets_with_dup(nums: List[int]) -> List[List[int]]:
    """
    Generates all unique subsets (with duplicates in input).
    Time Complexity: O(2^n), Space Complexity: O(n)
    """
    def backtrack(start, path):
        result.append(path[:])
        for i in range(start, len(nums)):
            # Skip duplicates
            if i > start and nums[i] == nums[i-1]:
                continue
            path.append(nums[i])
            backtrack(i+1, path)
            path.pop()
            
    nums.sort()
    result = []
    backtrack(0, [])
    return result
\end{minted}

% Palindrome Partitioning
\noindent\textbf{Problem: Palindrome Partitioning}
\begin{minted}[
bgcolor=bgcolor10,
frame=lines,
framesep=5mm,
rulecolor=\color{black},
linenos,
numbersep=5pt,
fontsize=\normalsize
]{python}
def partition_palindrome(s: str) -> List[List[str]]:
    """
    Partitions string such that every substring is palindrome.
    Time Complexity: O(n*2^n), Space Complexity: O(n^2)
    """
    def is_palindrome(sub):
        return sub == sub[::-1]
        
    def backtrack(start, path):
        if start == len(s):
            result.append(path[:])
            return
        for end in range(start+1, len(s)+1):
            substr = s[start:end]
            if is_palindrome(substr):
                path.append(substr)
                backtrack(end, path)
                path.pop()
                
    result = []
    backtrack(0, [])
    return result
\end{minted}

% M-Coloring of Graph
\noindent\textbf{Problem: M-Coloring of Graph}
\begin{minted}[
bgcolor=bgcolor2,
frame=lines,
framesep=5mm,
rulecolor=\color{black},
linenos,
numbersep=5pt,
fontsize=\normalsize
]{python}
def graph_coloring(graph: List[List[int]], m: int) -> bool:
    """
    Checks if graph can be colored with m colors (adjacent nodes different colors).
    Time Complexity: O(m^n), Space Complexity: O(n)
    """
    n = len(graph)
    colors = [-1] * n
    
    def is_safe(node, color):
        for neighbor in range(n):
            if graph[node][neighbor] == 1 and colors[neighbor] == color:
                return False
        return True
        
    def backtrack(node):
        if node == n:
            return True
        for color in range(m):
            if is_safe(node, color):
                colors[node] = color
                if backtrack(node+1):
                    return True
                colors[node] = -1  # Backtrack
        return False
        
    return backtrack(0)
\end{minted}

% \end{document}
