% \documentclass[a4paper,10pt]{article}
% \usepackage[utf8]{inputenc}
% \usepackage{geometry}
% \usepackage[table]{xcolor}
% \usepackage{colortbl}
% \usepackage{color,soul}
% \geometry{margin=0.8in}
% \usepackage{xcolor}
% \usepackage{tikz}
% \usepackage{minted}
% \definecolor{bgcolor}{rgb}{0.8, 0.9, 0.5} % 
% \definecolor{bgcolor1}{rgb}{0.95, 0.95, 0.95} % Light Gray
% \definecolor{bgcolor2}{rgb}{0.85, 0.92, 1.0}  % Soft Blue
% \definecolor{bgcolor3}{rgb}{0.9, 0.85, 1.0}   % Light Purple
% \definecolor{bgcolor4}{rgb}{0.95, 0.88, 0.76} % Warm Beige
% \definecolor{bgcolor5}{rgb}{0.8, 0.95, 0.8}   % Gentle Green
% \definecolor{bgcolor6}{rgb}{1.0, 0.87, 0.87}  % Pastel Red
% \definecolor{bgcolor7}{rgb}{0.86, 0.93, 0.83} % Mint Green
% \definecolor{bgcolor8}{rgb}{0.98, 0.85, 0.94} % Soft Pink
% \definecolor{bgcolor9}{rgb}{0.87, 0.94, 0.98} % Sky Blue
% \definecolor{bgcolor10}{rgb}{0.96, 0.96, 0.82} % Pale Yellow
% 
% % Define a custom background colo
% \begin{document}
\section*{String Problem Solutions}
\noindent\textbf{Problem: Is Subsequence of Other}
\begin{minted}[
bgcolor=bgcolor5,
frame=lines,
framesep=5mm,
rulecolor=\color{black},
linenos,
numbersep=5pt,
fontsize=\normalsize
]{python}
def is_subsequence(s: str, t: str) -> bool:
    """
    Return True if `s` is a subsequence of `t`, i.e. all characters of s
    appear in order (but not necessarily contiguously) in t.
    """
    i, j = 0, 0
    # i scans s, j scans t
    while i < len(s) and j < len(t):
        if s[i] == t[j]:
            # match: consume s[i]
            i += 1
        # always advance t’s pointer
        j += 1
    # if we’ve consumed all of s, it’s a subsequence
    return i == len(s)
\end{minted}
\noindent\textbf{Problem: Group the Anagrams}
\begin{minted}[
bgcolor=bgcolor6,
frame=lines,
framesep=5mm,
rulecolor=\color{black},
linenos,
numbersep=5pt,
fontsize=\normalsize
]{python}
from collections import defaultdict
from typing import List

def group_anagrams(words: List[str]) -> List[List[str]]:
    """
    Group a list of words into anagrams.
    """
    buckets = defaultdict(list)
    
    for w in words:
        # sort the word’s letters to form the key
        key = tuple(sorted(w))
        buckets[key].append(w)
    
    # return the grouped anagrams
    return list(buckets.values())
\end{minted}
\noindent\textbf{Problem:} Naive Pattern Searching
\begin{minted}[
bgcolor=bgcolor6,
frame=lines,
framesep=5mm,
rulecolor=\color{black},
linenos,
numbersep=5pt,
fontsize=\normalsize
]{python}
def naive_pattern_search(text: str, pat: str) -> List[int]:
    """
    Return start-indices where pat occurs in text by brute force.
    O(n*m) time.
    """
    n, m = len(text), len(pat)
    matches = []
    for i in range(n - m + 1):
        # check match at offset i
        for j in range(m):
            if text[i + j] != pat[j]:
                break
        else:
            matches.append(i)
    return matches
\end{minted}
\noindent\textbf{Problem: Rabin–Karp Algorithm}
\begin{minted}[
bgcolor=bgcolor7,
frame=lines,
framesep=5mm,
rulecolor=\color{black},
linenos,
numbersep=5pt,
fontsize=\normalsize
]{python}
def rabin_karp_search(text: str, pat: str,
                      base: int = 256, mod: int = 101) -> List[int]:
    """
    Find all occurrences of pat in text using rolling‐hash.
    Average O(n+m), worst O(n*m) if many hash collisions.
    """
    n, m = len(text), len(pat)
    if m > n:
        return []

    # precompute high-order base^(m-1) % mod
    h = pow(base, m - 1, mod)
    p_hash = t_hash = 0
    res = []

    # initial hash for pattern & first window
    for i in range(m):
        p_hash = (p_hash * base + ord(pat[i])) % mod
        t_hash = (t_hash * base + ord(text[i])) % mod

    for i in range(n - m + 1):
        # if hashes match, verify actual substring
        if p_hash == t_hash and text[i:i+m] == pat:
            res.append(i)
        # roll the window
        if i < n - m:
            t_hash = (t_hash - ord(text[i]) * h) % mod
            t_hash = (t_hash * base + ord(text[i + m])) % mod
            t_hash = (t_hash + mod) % mod  # ensure positive
    return res
\end{minted}
\noindent\textbf{Problem:} Knuth–Morris–Pratt (KMP) Algorithm
\begin{minted}[
bgcolor=bgcolor3,
frame=lines,
framesep=5mm,
rulecolor=\color{black},
linenos,
numbersep=5pt,
fontsize=\normalsize
]{python}
def kmp_search(text: str, pat: str) -> List[int]:
    """
    Return all start-indices of pat in text using KMP in O(n+m).
    """
    # build lps (longest proper‐prefix‐that‐is‐suffix) table
    m = len(pat)
    lps = [0] * m
    length = 0
    i = 1
    while i < m:
        if pat[i] == pat[length]:
            length += 1
            lps[i] = length
            i += 1
        elif length:
            length = lps[length - 1]
        else:
            lps[i] = 0
            i += 1

    # search
    res = []
    i = j = 0  # i→text, j→pat
    n = len(text)
    while i < n:
        if text[i] == pat[j]:
            i += 1
            j += 1
            if j == m:
                res.append(i - m)
                j = lps[j - 1]
        else:
            if j:
                j = lps[j - 1]
            else:
                i += 1
    return res
\end{minted}
\noindent\textbf{Problem:} Check if Strings are Rotations
\begin{minted}[
bgcolor=bgcolor8,
frame=lines,
framesep=5mm,
rulecolor=\color{black},
linenos,
numbersep=5pt,
fontsize=\normalsize
]{python}
def are_rotations(s1: str, s2: str) -> bool:
    """
    Return True if s2 is a rotation of s1 by checking s2 in s1+s1.
    O(n) time.
    """
    return len(s1) == len(s2) and s2 in (s1 + s1)
\end{minted}
\noindent\textbf{Problem:} Longest Substring with Distinct Characters
\begin{minted}[
bgcolor=bgcolor2,
frame=lines,
framesep=5mm,
rulecolor=\color{black},
linenos,
numbersep=5pt,
fontsize=\normalsize
]{python}
def longest_unique_substring(s: str) -> Tuple[int, str]:
    """
    Return (length, substring) of the longest substring without repeating chars.
    Uses sliding window + hashmap in O(n).
    """
    start = max_len = 0
    max_sub = ""
    last_seen = {}

    for i, c in enumerate(s):
        if c in last_seen and last_seen[c] >= start:
            # move window past previous occurrence
            start = last_seen[c] + 1
        last_seen[c] = i
        curr_len = i - start + 1
        if curr_len > max_len:
            max_len = curr_len
            max_sub = s[start:i+1]

    return max_len, max_sub
\end{minted}
\noindent\textbf{Problem:} Lexicographical rank of a String
\begin{minted}[
bgcolor=bgcolor4,
frame=lines,
framesep=5mm,
rulecolor=\color{black},
linenos,
numbersep=5pt,
fontsize=\normalsize
]{python}

def lexicographic_rank(s: str) -> int:
    """
    Return 1-based rank of s among its permutations (all chars unique).
    O(n^2) due to counting smaller chars for each position.
    """
    n = len(s)
    rank = 1
    for i in range(n):
        # count chars smaller than s[i] to its right
        smaller = sum(1 for j in range(i+1, n) if s[j] < s[i])
        rank += smaller * factorial(n - i - 1)
    return rank
\end{minted}
\noindent\textbf{Problem:} Longest Palindromic Substring
\begin{minted}[
bgcolor=bgcolor1,
frame=lines,
framesep=5mm,
rulecolor=\color{black},
linenos,
numbersep=5pt,
fontsize=\normalsize
]{python}
def longest_palindromic_substring(s: str) -> str:
    """
    Return the longest palindromic substring via expand‐around‐center in O(n^2).
    """
    if not s:
        return ""

    def expand(l: int, r: int) -> Tuple[int,int]:
        # expand as long as s[l]==s[r]
        while l >= 0 and r < len(s) and s[l] == s[r]:
            l -= 1
            r += 1
        return l+1, r-1  # back up one step

    start = end = 0
    for i in range(len(s)):
        # odd-length palindrome
        l1, r1 = expand(i, i)
        # even-length palindrome
        l2, r2 = expand(i, i+1)
        # pick the longer of the two
        if r1 - l1 > end - start:
            start, end = l1, r1
        if r2 - l2 > end - start:
            start, end = l2, r2

    return s[start:end+1]

\end{minted}
\noindent\textbf{Problem:} Z-function String Matching
\begin{minted}[
bgcolor=bgcolor10,
frame=lines,
framesep=5mm,
rulecolor=\color{black},
linenos,
numbersep=5pt,
fontsize=\normalsize
]{python}
def z_function(s: str) -> List[int]:
    """
    Compute the Z-array for s where Z[i] = max length of prefix
    starting at s[i]. Runs in O(len(s)).
    """
    n = len(s)
    Z = [0]*n
    l = r = 0
    for i in range(1, n):
        if i <= r:
            Z[i] = min(r - i + 1, Z[i-l])
        while i + Z[i] < n and s[Z[i]] == s[i + Z[i]]:
            Z[i] += 1
        if i + Z[i] - 1 > r:
            l, r = i, i + Z[i] - 1
    return Z

def z_search(text: str, pat: str) -> List[int]:
    """
    Find all occurrences of pat in text in O(n+m) via Z-function.
    """
    concat = pat + "$" + text
    Z = z_function(concat)
    m = len(pat)
    res = []
    for i in range(m+1, len(concat)):
        if Z[i] == m:
            # match starts at i-(m+1)
            res.append(i - m - 1)
    return res
\end{minted}
\noindent\textbf{Problem:} Anagram Search
\begin{minted}[
bgcolor=bgcolor,
frame=lines,
framesep=5mm,
rulecolor=\color{black},
linenos,
numbersep=5pt,
fontsize=\normalsize
]{python}
def find_anagrams(text: str, pat: str) -> List[int]:
    """
    Return all start-indices where text[i:i+len(pat)] is an anagram of pat.
    Uses sliding window + counters in O(n).
    """
    n, m = len(text), len(pat)
    if m > n:
        return []

    pat_count = Counter(pat)
    win_count = Counter()
    res = []

    for i, c in enumerate(text):
        win_count[c] += 1
        # shrink window when size > m
        if i >= m:
            left_char = text[i-m]
            if win_count[left_char] == 1:
                del win_count[left_char]
            else:
                win_count[left_char] -= 1
        # compare when we have a full-size window
        if i >= m-1 and win_count == pat_count:
            res.append(i - m + 1)

    return res
\end{minted}
% \end{document}
