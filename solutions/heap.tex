% \documentclass[a4paper,10pt]{article}
% \usepackage[utf8]{inputenc}
% \usepackage{geometry}
% \usepackage[table]{xcolor}
% \usepackage{colortbl}
% \usepackage{color,soul}
% \geometry{margin=0.8in}
% \usepackage{xcolor}
% \usepackage{tikz}
% \usepackage{minted}
% \definecolor{bgcolor}{rgb}{0.8, 0.9, 0.5} % 
% \definecolor{bgcolor1}{rgb}{0.95, 0.95, 0.95} % Light Gray
% \definecolor{bgcolor2}{rgb}{0.85, 0.92, 1.0}  % Soft Blue
% \definecolor{bgcolor3}{rgb}{0.9, 0.85, 1.0}   % Light Purple
% \definecolor{bgcolor4}{rgb}{0.95, 0.88, 0.76} % Warm Beige
% \definecolor{bgcolor5}{rgb}{0.8, 0.95, 0.8}   % Gentle Green
% \definecolor{bgcolor6}{rgb}{1.0, 0.87, 0.87}  % Pastel Red
% \definecolor{bgcolor7}{rgb}{0.86, 0.93, 0.83} % Mint Green
% \definecolor{bgcolor8}{rgb}{0.98, 0.85, 0.94} % Soft Pink
% \definecolor{bgcolor9}{rgb}{0.87, 0.94, 0.98} % Sky Blue
% \definecolor{bgcolor10}{rgb}{0.96, 0.96, 0.82} % Pale Yellow
% 
% \begin{document}
% Heapify
\noindent\textbf{Problem: Heapify (min-heap)}
\begin{minted}[
bgcolor=bgcolor3,
frame=lines,
framesep=5mm,
rulecolor=\color{black},
linenos,
numbersep=5pt,
fontsize=\normalsize
]{python}
def min_heapify(arr, i, n):
    """
    Maintains min-heap property from index i downward.
    Time Complexity: O(log n), Space Complexity: O(1)
    """
    smallest = i
    left = 2 * i + 1
    right = 2 * i + 2
    
    if left < n and arr[left] < arr[smallest]:
        smallest = left
        
    if right < n and arr[right] < arr[smallest]:
        smallest = right
        
    if smallest != i:
        arr[i], arr[smallest] = arr[smallest], arr[i]
        min_heapify(arr, smallest, n)
\end{minted}

% Build Heap
\noindent\textbf{Problem: Build Heap}
\begin{minted}[
bgcolor=bgcolor7,
frame=lines,
framesep=5mm,
rulecolor=\color{black},
linenos,
numbersep=5pt,
fontsize=\normalsize
]{python}
def build_min_heap(arr):
    """
    Converts array into min-heap from bottom-up.
    Time Complexity: O(n), Space Complexity: O(1)
    """
    n = len(arr)
    for i in range(n // 2 - 1, -1, -1):
        min_heapify(arr, i, n)
\end{minted}

% Insertion in Heap
\noindent\textbf{Problem: Insertion in Heap}
\begin{minted}[
bgcolor=bgcolor5,
frame=lines,
framesep=5mm,
rulecolor=\color{black},
linenos,
numbersep=5pt,
fontsize=\normalsize
]{python}
def heap_insert(heap, value):
    """
    Inserts value into min-heap and maintains heap property.
    Time Complexity: O(log n), Space Complexity: O(1)
    """
    heap.append(value)
    i = len(heap) - 1
    # Bubble up
    while i > 0:
        parent = (i - 1) // 2
        if heap[parent] <= heap[i]:
            break
        heap[i], heap[parent] = heap[parent], heap[i]
        i = parent
\end{minted}

% Decrease Key in Min Heap
\noindent\textbf{Problem: Decrease Key in Min Heap}
\begin{minted}[
bgcolor=bgcolor8,
frame=lines,
framesep=5mm,
rulecolor=\color{black},
linenos,
numbersep=5pt,
fontsize=\normalsize
]{python}
def decrease_key(heap, i, new_val):
    """
    Decreases value at index i and maintains heap property.
    Time Complexity: O(log n), Space Complexity: O(1)
    """
    if new_val > heap[i]:
        raise ValueError("New value larger than current")
    
    heap[i] = new_val
    # Bubble up
    while i > 0:
        parent = (i - 1) // 2
        if heap[parent] <= heap[i]:
            break
        heap[i], heap[parent] = heap[parent], heap[i]
        i = parent
\end{minted}

% Extract Min from Min Heap
\noindent\textbf{Problem: Extract Min from Min Heap}
\begin{minted}[
bgcolor=bgcolor1,
frame=lines,
framesep=5mm,
rulecolor=\color{black},
linenos,
numbersep=5pt,
fontsize=\normalsize
]{python}
def extract_min(heap):
    """
    Extracts minimum element from min-heap.
    Time Complexity: O(log n), Space Complexity: O(1)
    """
    if not heap:
        return None
    
    min_val = heap[0]
    heap[0] = heap[-1]
    heap.pop()
    min_heapify(heap, 0, len(heap))
    return min_val
\end{minted}

% Heap Sort
\noindent\textbf{Problem: Heap Sort}
\begin{minted}[
bgcolor=bgcolor9,
frame=lines,
framesep=5mm,
rulecolor=\color{black},
linenos,
numbersep=5pt,
fontsize=\normalsize
]{python}
def heap_sort(arr):
    """
    Sorts array using heap sort algorithm.
    Time Complexity: O(n log n), Space Complexity: O(1)
    """
    n = len(arr)
    # Build max-heap (using min_heapify with sign flip)
    for i in range(n // 2 - 1, -1, -1):
        min_heapify(arr, i, n)
    
    # Extract elements one by one
    for i in range(n - 1, 0, -1):
        arr[0], arr[i] = arr[i], arr[0]
        min_heapify(arr, 0, i)
\end{minted}

% Sort K-Sorted Array
\noindent\textbf{Problem: Sort K-Sorted Array}
\begin{minted}[
bgcolor=bgcolor6,
frame=lines,
framesep=5mm,
rulecolor=\color{black},
linenos,
numbersep=5pt,
fontsize=\normalsize
]{python}
import heapq

def sort_k_sorted(arr, k):
    """
    Sorts array where each element is at most k positions away.
    Time Complexity: O(n log k), Space Complexity: O(k)
    """
    min_heap = []
    sorted_arr = []
    
    # Add first k+1 elements
    for i in range(min(k + 1, len(arr))):
        heapq.heappush(min_heap, arr[i])
    
    # Process remaining elements
    for i in range(k + 1, len(arr)):
        sorted_arr.append(heapq.heappop(min_heap))
        heapq.heappush(min_heap, arr[i])
    
    # Empty heap
    while min_heap:
        sorted_arr.append(heapq.heappop(min_heap))
    
    return sorted_arr
\end{minted}

% Buy Maximum Items
\noindent\textbf{Problem: Buy Maximum Items with Given Sum}
\begin{minted}[
bgcolor=bgcolor4,
frame=lines,
framesep=5mm,
rulecolor=\color{black},
linenos,
numbersep=5pt,
fontsize=\normalsize
]{python}
import heapq

def max_items(costs, budget):
    """
    Calculates maximum items that can be bought with given budget.
    Time Complexity: O(n log n), Space Complexity: O(1)
    """
    heapq.heapify(costs)
    count = 0
    
    while costs and budget >= costs[0]:
        budget -= heapq.heappop(costs)
        count += 1
    
    return count
\end{minted}

% K Largest Elements
\noindent\textbf{Problem: K Largest Elements}
\begin{minted}[
bgcolor=bgcolor10,
frame=lines,
framesep=5mm,
rulecolor=\color{black},
linenos,
numbersep=5pt,
fontsize=\normalsize
]{python}
import heapq

def k_largest(arr, k):
    """
    Finds k largest elements in array using min-heap.
    Time Complexity: O(n log k), Space Complexity: O(k)
    """
    min_heap = []
    for num in arr:
        heapq.heappush(min_heap, num)
        if len(min_heap) > k:
            heapq.heappop(min_heap)
    return min_heap
\end{minted}

% K Closest Elements
\noindent\textbf{Problem: K Closest Elements}
\begin{minted}[
bgcolor=bgcolor2,
frame=lines,
framesep=5mm,
rulecolor=\color{black},
linenos,
numbersep=5pt,
fontsize=\normalsize
]{python}
import heapq

def k_closest(points, k, origin=(0,0)):
    """
    Finds k closest points to origin using max-heap.
    Time Complexity: O(n log k), Space Complexity: O(k)
    """
    def distance(p):
        return (p[0]-origin[0])**2 + (p[1]-origin[1])**2
    
    max_heap = []
    for point in points:
        dist = -distance(point)  # Use negative for max-heap
        if len(max_heap) < k:
            heapq.heappush(max_heap, (dist, point))
        else:
            if dist > max_heap[0][0]:
                heapq.heapreplace(max_heap, (dist, point))
    
    return [point for (_, point) in max_heap]
\end{minted}

% Merge K Sorted Arrays
\noindent\textbf{Problem: Merge K Sorted Arrays}
\begin{minted}[
bgcolor=bgcolor7,
frame=lines,
framesep=5mm,
rulecolor=\color{black},
linenos,
numbersep=5pt,
fontsize=\normalsize
]{python}
import heapq

def merge_k_sorted(arrays):
    """
    Merges k sorted arrays into single sorted array.
    Time Complexity: O(n log k), Space Complexity: O(k)
    """
    min_heap = []
    result = []
    
    # Initialize heap with first element of each array
    for i, arr in enumerate(arrays):
        if arr:
            heapq.heappush(min_heap, (arr[0], i, 0))
    
    # Merge process
    while min_heap:
        val, arr_idx, idx = heapq.heappop(min_heap)
        result.append(val)
        if idx + 1 < len(arrays[arr_idx]):
            next_val = arrays[arr_idx][idx+1]
            heapq.heappush(min_heap, (next_val, arr_idx, idx+1))
    
    return result
\end{minted}

% Median of a Stream
\noindent\textbf{Problem: Median of a Stream}
\begin{minted}[
bgcolor=bgcolor5,
frame=lines,
framesep=5mm,
rulecolor=\color{black},
linenos,
numbersep=5pt,
fontsize=\normalsize
]{python}
import heapq

class MedianFinder:
    """
    Maintains median of streaming data using two heaps.
    Time Complexity: O(log n) per insertion, O(1) for median
    Space Complexity: O(n)
    """
    def __init__(self):
        self.small = []  # max-heap (store negative values)
        self.large = []  # min-heap

    def add_num(self, num: int) -> None:
        # Add to appropriate heap
        if not self.small or num <= -self.small[0]:
            heapq.heappush(self.small, -num)
        else:
            heapq.heappush(self.large, num)
        
        # Balance heaps
        if len(self.small) > len(self.large) + 1:
            heapq.heappush(self.large, -heapq.heappop(self.small))
        elif len(self.large) > len(self.small):
            heapq.heappush(self.small, -heapq.heappop(self.large))

    def find_median(self) -> float:
        if len(self.small) == len(self.large):
            return (-self.small[0] + self.large[0]) / 2
        return -self.small[0]
\end{minted}

% \end{document}
