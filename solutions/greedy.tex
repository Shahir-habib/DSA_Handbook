% \documentclass[a4paper,10pt]{article}
% \usepackage[utf8]{inputenc}
% \usepackage{geometry}
% \usepackage[table]{xcolor}
% \usepackage{colortbl}
% \usepackage{color,soul}
% \geometry{margin=0.8in}
% \usepackage{xcolor}
% \usepackage{tikz}
% \usepackage{minted}
% \definecolor{bgcolor}{rgb}{0.8, 0.9, 0.5} % 
% \definecolor{bgcolor1}{rgb}{0.95, 0.95, 0.95} % Light Gray
% \definecolor{bgcolor2}{rgb}{0.85, 0.92, 1.0}  % Soft Blue
% \definecolor{bgcolor3}{rgb}{0.9, 0.85, 1.0}   % Light Purple
% \definecolor{bgcolor4}{rgb}{0.95, 0.88, 0.76} % Warm Beige
% \definecolor{bgcolor5}{rgb}{0.8, 0.95, 0.8}   % Gentle Green
% \definecolor{bgcolor6}{rgb}{1.0, 0.87, 0.87}  % Pastel Red
% \definecolor{bgcolor7}{rgb}{0.86, 0.93, 0.83} % Mint Green
% \definecolor{bgcolor8}{rgb}{0.98, 0.85, 0.94} % Soft Pink
% \definecolor{bgcolor9}{rgb}{0.87, 0.94, 0.98} % Sky Blue
% \definecolor{bgcolor10}{rgb}{0.96, 0.96, 0.82} % Pale Yellow
% 
% \begin{document}
% Activity Selection Problem
\noindent\textbf{Problem: Activity Selection Problem}
\begin{minted}[
bgcolor=bgcolor3,
frame=lines,
framesep=5mm,
rulecolor=\color{black},
linenos,
numbersep=5pt,
fontsize=\normalsize
]{python}
def activity_selection(start, finish):
    """
    Selects maximum number of non-overlapping activities.
    Time Complexity: O(n log n), Space Complexity: O(n)
    """
    # Sort activities by finish time
    activities = sorted(zip(start, finish), key=lambda x: x[1])
    count = 1
    last_finish = activities[0][1]
    
    # Select activities
    for i in range(1, len(activities)):
        s, f = activities[i]
        if s >= last_finish:
            count += 1
            last_finish = f
    
    return count
\end{minted}

% Fractional Knapsack Problem
\noindent\textbf{Problem: Fractional Knapsack Problem}
\begin{minted}[
bgcolor=bgcolor7,
frame=lines,
framesep=5mm,
rulecolor=\color{black},
linenos,
numbersep=5pt,
fontsize=\normalsize
]{python}
def fractional_knapsack(weights, values, capacity):
    """
    Maximizes value in knapsack using fractional items.
    Time Complexity: O(n log n), Space Complexity: O(n)
    """
    # Calculate value per unit weight
    items = [(v/w, w, v) for w, v in zip(weights, values)]
    items.sort(reverse=True)  # Sort by value/weight descending
    
    total_value = 0.0
    for ratio, weight, value in items:
        if capacity >= weight:
            # Take whole item
            total_value += value
            capacity -= weight
        else:
            # Take fraction
            total_value += ratio * capacity
            break
    
    return total_value
\end{minted}

% Job Sequencing Problem
\noindent\textbf{Problem: Job Sequencing Problem}
\begin{minted}[
bgcolor=bgcolor5,
frame=lines,
framesep=5mm,
rulecolor=\color{black},
linenos,
numbersep=5pt,
fontsize=\normalsize
]{python}
def job_sequencing(jobs):
    """
    Maximizes profit by scheduling jobs within deadlines.
    Time Complexity: O(n²), Space Complexity: O(n)
    """
    # Sort jobs by profit descending
    jobs.sort(key=lambda x: x[2], reverse=True)
    
    # Find maximum deadline
    max_deadline = max(job[1] for job in jobs)
    
    # Initialize result array
    result = [None] * (max_deadline + 1)
    total_profit = 0
    
    for job in jobs:
        # Find latest available slot before deadline
        for j in range(job[1], 0, -1):
            if j <= max_deadline and result[j] is None:
                result[j] = job[0]  # Job ID
                total_profit += job[2]  # Profit
                break
    
    # Return sequence and profit
    sequence = [job_id for job_id in result if job_id is not None]
    return sequence, total_profit
\end{minted}

% Huffman Coding
\noindent\textbf{Problem: Huffman Coding}
\begin{minted}[
bgcolor=bgcolor8,
frame=lines,
framesep=5mm,
rulecolor=\color{black},
linenos,
numbersep=5pt,
fontsize=\normalsize
]{python}
import heapq
from collections import defaultdict

class Node:
    def __init__(self, char=None, freq=0):
        self.char = char
        self.freq = freq
        self.left = None
        self.right = None
        
    # For heap comparison
    def __lt__(self, other):
        return self.freq < other.freq

def build_huffman_tree(text):
    """
    Builds Huffman tree from character frequencies.
    Time Complexity: O(n log n), Space Complexity: O(n)
    """
    # Count character frequencies
    freq = defaultdict(int)
    for char in text:
        freq[char] += 1
    
    # Create leaf nodes and push to min-heap
    min_heap = []
    for char, count in freq.items():
        heapq.heappush(min_heap, Node(char, count))
    
    # Build Huffman tree
    while len(min_heap) > 1:
        left = heapq.heappop(min_heap)
        right = heapq.heappop(min_heap)
        
        merged = Node(freq=left.freq + right.freq)
        merged.left = left
        merged.right = right
        heapq.heappush(min_heap, merged)
    
    return heapq.heappop(min_heap)

def generate_codes(root, prefix="", code_map=None):
    """
    Generates Huffman codes from tree traversal.
    Time Complexity: O(n), Space Complexity: O(n)
    """
    if code_map is None:
        code_map = {}
    
    if root.char is not None:
        code_map[root.char] = prefix
    else:
        generate_codes(root.left, prefix + "0", code_map)
        generate_codes(root.right, prefix + "1", code_map)
    
    return code_map

def huffman_encode(text):
    """
    Encodes text using Huffman coding.
    Returns encoded text and code mapping.
    """
    if not text:
        return "", {}
    
    root = build_huffman_tree(text)
    codes = generate_codes(root)
    encoded = "".join(codes[char] for char in text)
    return encoded, codes
\end{minted}

% \end{document}
