% \documentclass[a4paper,10pt]{article}
% \usepackage[utf8]{inputenc}
% \usepackage{geometry}
% \usepackage[table]{xcolor}
% \usepackage{colortbl}
% \usepackage{color,soul}
% \geometry{margin=0.8in}
% \usepackage{xcolor}
% \usepackage{tikz}
% \usepackage{minted}
% \definecolor{bgcolor}{rgb}{0.8, 0.9, 0.5} % 
% \definecolor{bgcolor1}{rgb}{0.95, 0.95, 0.95} % Light Gray
% \definecolor{bgcolor2}{rgb}{0.85, 0.92, 1.0}  % Soft Blue
% \definecolor{bgcolor3}{rgb}{0.9, 0.85, 1.0}   % Light Purple
% \definecolor{bgcolor4}{rgb}{0.95, 0.88, 0.76} % Warm Beige
% \definecolor{bgcolor5}{rgb}{0.8, 0.95, 0.8}   % Gentle Green
% \definecolor{bgcolor6}{rgb}{1.0, 0.87, 0.87}  % Pastel Red
% \definecolor{bgcolor7}{rgb}{0.86, 0.93, 0.83} % Mint Green
% \definecolor{bgcolor8}{rgb}{0.98, 0.85, 0.94} % Soft Pink
% \definecolor{bgcolor9}{rgb}{0.87, 0.94, 0.98} % Sky Blue
% \definecolor{bgcolor10}{rgb}{0.96, 0.96, 0.82} % Pale Yellow
% 
% % Define a custom background colo
% \begin{document}
\section*{Hashing Problem Solutions}
\noindent\textbf{Problem: Intersection and Union of Arrays}
\begin{minted}[bgcolor=bgcolor3,frame=lines,framesep=5mm,rulecolor=\color{black},linenos,numbersep=5pt,fontsize=\normalsize]{python}
def intersection_union(a, b):
    """
    Return the intersection and union of two arrays.
    Time Complexity: O(n + m)
    """
    sa, sb = set(a), set(b)
    # Intersection: elements common to both
    inter = sa & sb
    # Union: all elements from both
    uni   = sa | sb
    return list(inter), list(uni)
\end{minted}

\noindent\textbf{Problem: Subarray with Sum = 0}
\begin{minted}[bgcolor=bgcolor4,frame=lines,framesep=5mm,rulecolor=\color{black},linenos,numbersep=5pt,fontsize=\normalsize]{python}
def has_zero_sum_subarray(arr):
    """
    Check if any subarray sums to zero.
    Time Complexity: O(n)
    """
    seen = set([0])
    prefix = 0
    for x in arr:
        prefix += x
        # If prefix repeats or is zero, subarray sum = 0 exists
        if prefix in seen:
            return True
        seen.add(prefix)
    return False
\end{minted}

\noindent\textbf{Problem: Subarray with Xor = k}
\begin{minted}[bgcolor=bgcolor5,frame=lines,framesep=5mm,rulecolor=\color{black},linenos,numbersep=5pt,fontsize=\normalsize]{python}
def has_xor_subarray(arr, k):
    """
    Check if any subarray's xor equals k.
    Time Complexity: O(n)
    """
    seen = set([0])
    prefix = 0
    for x in arr:
        prefix ^= x
        # If prefix^k seen before, subarray xor = k exists
        if (prefix ^ k) in seen:
            return True
        seen.add(prefix)
    return False
\end{minted}

\noindent\textbf{Problem: Longest Subarray with Given Sum k}
\begin{minted}[bgcolor=bgcolor6,frame=lines,framesep=5mm,rulecolor=\color{black},linenos,numbersep=5pt,fontsize=\normalsize]{python}
def longest_subarray_sum_k(arr, k):
    """
    Return length of longest subarray summing to k.
    Time Complexity: O(n)
    """
    first_occ = {0: -1}  # prefix_sum -> first index
    prefix = 0
    max_len = 0
    for i, x in enumerate(arr):
        prefix += x
        # Record first occurrence of this prefix
        if prefix not in first_occ:
            first_occ[prefix] = i
        # Check if a subarray summing to k ends here
        if (prefix - k) in first_occ:
            length = i - first_occ[prefix - k]
            max_len = max(max_len, length)
    return max_len
\end{minted}

\noindent\textbf{Problem: Longest Subarray with Equal 0s and 1s}
\begin{minted}[bgcolor=bgcolor7,frame=lines,framesep=5mm,rulecolor=\color{black},linenos,numbersep=5pt,fontsize=\normalsize]{python}
def longest_equal_01(arr):
    """
    Longest subarray with equal number of 0s and 1s.
    Time Complexity: O(n)
    """
    # Replace 0 with -1, then find longest zero-sum subarray
    mapped = [1 if x==1 else -1 for x in arr]
    first_occ = {0: -1}
    prefix = 0
    max_len = 0
    for i, x in enumerate(mapped):
        prefix += x
        if prefix not in first_occ:
            first_occ[prefix] = i
        max_len = max(max_len, i - first_occ[prefix])
    return max_len
\end{minted}

\noindent\textbf{Problem: Longest Common Binary Subarray with Given Sum}
\begin{minted}[bgcolor=bgcolor8,frame=lines,framesep=5mm,rulecolor=\color{black},linenos,numbersep=5pt,fontsize=\normalsize]{python}
from typing import List, Tuple

def longest_common_sum_subarray(a: List[int], b: List[int]) -> Tuple[int, List[int]]:
    """
    Finds the longest span over which the prefix sums of a and b are equal.
    Returns (length, subarray_of_a).
    """
    n = min(len(a), len(b))
    diff_to_idx = {0: -1}    # maps prefix‐sum diff → earliest index
    diff = 0
    best_len = 0
    best_start = 0

    for i in range(n):
        # update running difference
        diff += a[i] - b[i]

        if diff in diff_to_idx:
            # we’ve seen this diff before at index prev
            prev = diff_to_idx[diff]
            curr_len = i - prev
            if curr_len > best_len:
                best_len = curr_len
                best_start = prev + 1
        else:
            # first time seeing this diff
            diff_to_idx[diff] = i

    # slice out the winner from a
    return best_len, a[best_start : best_start + best_len]

\end{minted}

\noindent\textbf{Problem: Longest Consecutive Subsequence}
\begin{minted}[bgcolor=bgcolor9,frame=lines,framesep=5mm,rulecolor=\color{black},linenos,numbersep=5pt,fontsize=\normalsize]{python}
def longest_consecutive(nums):
    """
    Length of the longest run of consecutive integers.
    Time Complexity: O(n)
    """
    num_set = set(nums)
    max_len = 0
    for x in num_set:
        # Only start at the beginning of a sequence
        if x - 1 not in num_set:
            curr = x
            length = 1
            while curr + 1 in num_set:
                curr += 1
                length += 1
            max_len = max(max_len, length)
    return max_len
\end{minted}

\noindent\textbf{Problem: Count Distinct Elements in Every Window}
\begin{minted}[bgcolor=bgcolor10,frame=lines,framesep=5mm,rulecolor=\color{black},linenos,numbersep=5pt,fontsize=\normalsize]{python}
def distinct_in_windows(arr, k):
    """
    Return list of distinct counts for every window of size k.
    Time Complexity: O(n)
    """
    if k > len(arr): return []
    freq = {}
    distinct = []
    # Initialize first window
    for x in arr[:k]:
        freq[x] = freq.get(x, 0) + 1
    distinct.append(len(freq))
    # Slide window
    for i in range(k, len(arr)):
        # Remove outgoing
        out = arr[i-k]
        freq[out] -= 1
        if freq[out] == 0:
            del freq[out]
        # Add incoming
        in_ = arr[i]
        freq[in_] = freq.get(in_, 0) + 1
        distinct.append(len(freq))
    return distinct
\end{minted}

\noindent\textbf{Problem: More than n/k Occurrences in Array}
\begin{minted}[bgcolor=bgcolor5,frame=lines,framesep=5mm,rulecolor=\color{black},linenos,numbersep=5pt,fontsize=\normalsize]{python}
from typing import List, Any

def more_than_n_over_k(arr: List[Any], k: int) -> List[Any]:
    """
    Return all elements in `arr` that appear more than len(arr)//k times.
    Uses a Boyer–Moore–style majority‐vote generalization with at most k−1 candidates.
    Time Complexity: O(n), Space Complexity: O(k)
    """
    n = len(arr)
    if k < 2:
        raise ValueError("k must be at least 2")

    # 1) First pass: find up to (k−1) potential candidates
    #    Maintain a map of candidate -> “vote count”
    candidates = {}  # type: dict[Any, int]
    for x in arr:
        if x in candidates:
            # increment count if x is already a candidate
            candidates[x] += 1
        elif len(candidates) < k - 1:
            # room to add a new candidate
            candidates[x] = 1
        else:
            # no room: decrement every candidate’s count
            # if any count drops to zero, remove it
            to_delete = []
            for c in candidates:
                candidates[c] -= 1
                if candidates[c] == 0:
                    to_delete.append(c)
            for c in to_delete:
                del candidates[c]

    # 2) Second pass: verify actual frequencies of the remaining candidates
    counts = {c: 0 for c in candidates}
    for x in arr:
        if x in counts:
            counts[x] += 1

    # 3) Collect those whose true count exceeds n//k
    result = []
    threshold = n // k
    for c, cnt in counts.items():
        if cnt > threshold:
            result.append(c)

    return result
\end{minted}
% \end{document}
