% \documentclass[a4paper,10pt]{article}
% \usepackage[utf8]{inputenc}
% \usepackage{geometry}
% \usepackage[table]{xcolor}
% \usepackage{colortbl}
% \usepackage{color,soul}
% \geometry{margin=0.8in}
% \usepackage{xcolor}
% \usepackage{tikz}
% \usepackage{minted}
% \definecolor{bgcolor}{rgb}{0.8, 0.9, 0.5} % 
% \definecolor{bgcolor1}{rgb}{0.95, 0.95, 0.95} % Light Gray
% \definecolor{bgcolor2}{rgb}{0.85, 0.92, 1.0}  % Soft Blue
% \definecolor{bgcolor3}{rgb}{0.9, 0.85, 1.0}   % Light Purple
% \definecolor{bgcolor4}{rgb}{0.95, 0.88, 0.76} % Warm Beige
% \definecolor{bgcolor5}{rgb}{0.8, 0.95, 0.8}   % Gentle Green
% \definecolor{bgcolor6}{rgb}{1.0, 0.87, 0.87}  % Pastel Red
% \definecolor{bgcolor7}{rgb}{0.86, 0.93, 0.83} % Mint Green
% \definecolor{bgcolor8}{rgb}{0.98, 0.85, 0.94} % Soft Pink
% \definecolor{bgcolor9}{rgb}{0.87, 0.94, 0.98} % Sky Blue
% \definecolor{bgcolor10}{rgb}{0.96, 0.96, 0.82} % Pale Yellow
% 
% \begin{document}
% Balanced Parentheses
\noindent\textbf{Problem: Balanced Parentheses}
\begin{minted}[
bgcolor=bgcolor2,
frame=lines,
framesep=5mm,
rulecolor=\color{black},
linenos,
numbersep=5pt,
fontsize=\normalsize
]{python}
def is_valid(s: str) -> bool:
    """
    Checks if parentheses string is balanced.
    Time Complexity: O(n), Space Complexity: O(n)
    """
    stack = []
    mapping = {')': '(', '}': '{', ']': '['}
    
    for char in s:
        if char in mapping:
            top = stack.pop() if stack else '#'
            if mapping[char] != top:
                return False
        else:
            stack.append(char)
    
    return not stack
\end{minted}

\noindent\textbf{Problem: Sort A Stack}
\begin{minted}[
bgcolor=bgcolor4,
frame=lines,
framesep=5mm,
rulecolor=\color{black},
linenos,
numbersep=5pt,
fontsize=\normalsize
]{python}
def sorted_insert(stack, element):
    """Helper function to insert an element into a sorted stack."""
    if not stack or element > stack[-1]:
        stack.append(element)
    else:
        temp = stack.pop()
        sorted_insert(stack, element)
        stack.append(temp)

def sort_stack(stack):
    """Sorts the stack in ascending order using recursion."""
    if stack:
        temp = stack.pop()
        sort_stack(stack)
        sorted_insert(stack, temp)
\end{minted}
% Implement K Stacks in an Array
\noindent\textbf{Problem: Implement K Stacks in an Array}
\begin{minted}[
bgcolor=bgcolor7,
frame=lines,
framesep=5mm,
rulecolor=\color{black},
linenos,
numbersep=5pt,
fontsize=\normalsize
]{python}
class KStacks:
    """
    Efficiently implements K stacks in single array.
    Time Complexity: O(1) for push/pop, Space Complexity: O(n + k)
    """
    def __init__(self, k, n):
        self.k = k  # Number of stacks
        self.n = n  # Total array size
        self.arr = [0] * n  # Storage array
        self.top = [-1] * k  # Top indices for each stack
        self.next_idx = list(range(1, n)) + [-1]  # Next free index
        self.free = 0  # Starting free index

    def push(self, sn, item):
        if self.free == -1:
            raise Exception("Stack overflow")
        
        # Store at current free position
        insert_at = self.free
        self.free = self.next_idx[insert_at]
        self.arr[insert_at] = item
        
        # Update stack pointers
        self.next_idx[insert_at] = self.top[sn]
        self.top[sn] = insert_at

    def pop(self, sn):
        if self.top[sn] == -1:
            raise Exception("Stack underflow")
        
        # Get top element and update pointers
        top_idx = self.top[sn]
        item = self.arr[top_idx]
        self.top[sn] = self.next_idx[top_idx]
        
        # Update free list
        self.next_idx[top_idx] = self.free
        self.free = top_idx
        
        return item
\end{minted}

% Stock Span Problem
\noindent\textbf{Problem: Stock Span Problem}
\begin{minted}[
bgcolor=bgcolor5,
frame=lines,
framesep=5mm,
rulecolor=\color{black},
linenos,
numbersep=5pt,
fontsize=\normalsize
]{python}
def stock_span(prices):
    """
    Calculates span for each day where span = consecutive days before
    where price <= current day.
    Time Complexity: O(n), Space Complexity: O(n)
    """
    n = len(prices)
    stack = []
    span = [1] * n
    
    for i in range(n):
        # Pop smaller elements
        while stack and prices[stack[-1]] <= prices[i]:
            stack.pop()
        
        # Calculate span
        span[i] = i - stack[-1] if stack else i + 1
        stack.append(i)
    
    return span
\end{minted}

% Previous Greater Element
\noindent\textbf{Problem: Previous Greater Element}
\begin{minted}[
bgcolor=bgcolor3,
frame=lines,
framesep=5mm,
rulecolor=\color{black},
linenos,
numbersep=5pt,
fontsize=\normalsize
]{python}
def prev_greater(arr):
    """
    Finds previous greater element for each array element.
    Time Complexity: O(n), Space Complexity: O(n)
    """
    stack = []
    result = [-1] * len(arr)
    
    for i in range(len(arr)):
        while stack and arr[stack[-1]] <= arr[i]:
            stack.pop()
        
        result[i] = arr[stack[-1]] if stack else -1
        stack.append(i)
    
    return result
\end{minted}

% Next Greater Element
\noindent\textbf{Problem: Next Greater Element}
\begin{minted}[
bgcolor=bgcolor8,
frame=lines,
framesep=5mm,
rulecolor=\color{black},
linenos,
numbersep=5pt,
fontsize=\normalsize
]{python}
def next_greater(arr):
    """
    Finds next greater element for each array element.
    Time Complexity: O(n), Space Complexity: O(n)
    """
    stack = []
    result = [-1] * len(arr)
    
    for i in range(len(arr)):
        while stack and arr[stack[-1]] < arr[i]:
            result[stack.pop()] = arr[i]
        stack.append(i)
    
    return result
####################################ALTERNATE#####################################
    def next_greater_from_right(arr):
    """
    Finds next greater element for each array element (right to left traversal).
    Time: O(n), Space: O(n)
    """
    stack = []
    n = len(arr)
    result = [-1] * n

    for i in range(n - 1, -1, -1):
        # Pop smaller or equal elements
        while stack and stack[-1] <= arr[i]:
            stack.pop()

        # Top of stack is the next greater
        if stack:
            result[i] = stack[-1]

        # Push current element for upcoming comparisons
        stack.append(arr[i])

    return result

\end{minted}

% Largest Rectangular Area in Histogram
\noindent\textbf{Problem: Largest Rectangular Area in Histogram}\\
Keep a monotonic increasing stack of indices.When you see a new bar at index i that’s shorter than the bar at the top of the stack, you know:

The popped bar (call its index j) has its right‐first‐smaller at i.

Its left‐first‐smaller is the new top of the stack (after popping).
\begin{minted}[
bgcolor=bgcolor1,
frame=lines,
framesep=5mm,
rulecolor=\color{black},
linenos,
numbersep=5pt,
fontsize=\normalsize
]{python}
def largest_rectangle_area(heights):
    """
    Calculates largest rectangle area in histogram using monotonic stack.
    Time Complexity: O(n), Space Complexity: O(n)
    """
    stack = []
    max_area = 0
    heights.append(0)  # Sentinel value
    
    for i in range(len(heights)):
        while stack and heights[i] < heights[stack[-1]]:
            h = heights[stack.pop()]
            w = i - stack[-1] - 1 if stack else i
            max_area = max(max_area, h * w)
        stack.append(i)
    
    return max_area
\end{minted}

% Stack with getMin() in O(1)
\noindent\textbf{Problem: Stack with getMin() in O(1)}
\begin{minted}[
bgcolor=bgcolor9,
frame=lines,
framesep=5mm,
rulecolor=\color{black},
linenos,
numbersep=5pt,
fontsize=\normalsize
]{python}
class MinStack:
    """
    Implements stack that supports push, pop, top, and getMin in O(1) time.
    Space Complexity: O(n)
    """
    def __init__(self):
        self.stack = []
        self.min_stack = []

    def push(self, val: int) -> None:
        self.stack.append(val)
        if not self.min_stack or val <= self.min_stack[-1]:
            self.min_stack.append(val)

    def pop(self) -> None:
        if self.stack.pop() == self.min_stack[-1]:
            self.min_stack.pop()

    def top(self) -> int:
        return self.stack[-1]

    def getMin(self) -> int:
        return self.min_stack[-1]

################################CONSTANT SPACE #############################

class MinStack:
    def __init__(self):
        self.stack = []
        self.min = None

    def push(self, val: int) -> None:
        if not self.stack:
            self.stack.append(val)
            self.min = val
        elif val >= self.min:
            self.stack.append(val)
        else:
            # Encode the new min in the stack
            self.stack.append(2 * val - self.min)
            self.min = val

    def pop(self) -> None:
        if not self.stack:
            return
        top = self.stack.pop()
        if top < self.min:
            # Decoding previous min
            self.min = 2 * self.min - top

    def top(self) -> int:
        top = self.stack[-1]
        if top >= self.min:
            return top
        else:
            # It’s an encoded value, so actual top is current min
            return self.min

    def getMin(self) -> int:
        return self.min
\end{minted}

% Sum of Subarray Minimums
\noindent\textbf{Problem: Sum of Subarray Minimums}
\begin{minted}[
bgcolor=bgcolor6,
frame=lines,
framesep=5mm,
rulecolor=\color{black},
linenos,
numbersep=5pt,
fontsize=\normalsize
]{python}
def sum_subarray_mins(arr):
    """
    Calculates sum of minimums of all contiguous subarrays.
    Time Complexity: O(n), Space Complexity: O(n)
    """
    MOD = 10**9 + 7
    stack = []
    left = [0] * len(arr)  # Left boundary
    right = [0] * len(arr)  # Right boundary

    # Left boundaries
    for i in range(len(arr)):
        while stack and arr[stack[-1]] > arr[i]:
            stack.pop()
        left[i] = i - stack[-1] if stack else i + 1
        stack.append(i)
    
    stack.clear()
    
    # Right boundaries
    for i in range(len(arr)-1, -1, -1):
        while stack and arr[stack[-1]] >= arr[i]:
            stack.pop()
        right[i] = stack[-1] - i if stack else len(arr) - i
        stack.append(i)
    
    # Calculate total sum
    total = 0
    for i in range(len(arr)):
        total = (total + arr[i] * left[i] * right[i]) % MOD
        
    return total
\end{minted}

% Remove K Digits to Make Minimum
\noindent\textbf{Problem: Remove K Digits to Make Minimum}
\begin{minted}[
bgcolor=bgcolor4,
frame=lines,
framesep=5mm,
rulecolor=\color{black},
linenos,
numbersep=5pt,
fontsize=\normalsize
]{python}
def remove_k_digits(num: str, k: int) -> str:
    """
    Removes k digits to form smallest possible number.
    Time Complexity: O(n), Space Complexity: O(n)
    """
    stack = []
    remain = len(num) - k
    
    for digit in num:
        while k and stack and stack[-1] > digit:
            stack.pop()
            k -= 1
        stack.append(digit)
    
    # If we still have digits to remove (monotonically increasing case)
    while k > 0:
        stack.pop()
        k -= 1
    
    # Remove leading zeros
    res = "".join(stack).lstrip('0')
    return res if res else "0"
\end{minted}

% Infix to Postfix Conversion
\noindent\textbf{Problem: Infix to Postfix Conversion}
\begin{minted}[
bgcolor=bgcolor10,
frame=lines,
framesep=5mm,
rulecolor=\color{black},
linenos,
numbersep=5pt,
fontsize=\normalsize
]{python}
def infix_to_postfix(infix):
    """
    Converts infix expression to postfix using Shunting Yard algorithm.
    Time Complexity: O(n), Space Complexity: O(n)
    """
    precedence = {'+':1, '-':1, '*':2, '/':2, '^':3}
    stack = []
    output = []
    
    for token in infix:
        if token.isalnum():
            output.append(token)
        elif token == '(':
            stack.append(token)
        elif token == ')':
            while stack and stack[-1] != '(':
                output.append(stack.pop())
            stack.pop()  # Remove '('
        else:  # Operator
            while (stack and stack[-1] != '(' and 
                   precedence[token] <= precedence.get(stack[-1], 0)):
                output.append(stack.pop())
            stack.append(token)
    
    while stack:
        output.append(stack.pop())
    
    return ''.join(output)
\end{minted}

% Evaluation of Postfix Expression
\noindent\textbf{Problem: Evaluation of Postfix Expression}
\begin{minted}[
bgcolor=bgcolor2,
frame=lines,
framesep=5mm,
rulecolor=\color{black},
linenos,
numbersep=5pt,
fontsize=\normalsize
]{python}
def eval_postfix(expression):
    """
    Evaluates postfix expression using stack.
    Time Complexity: O(n), Space Complexity: O(n)
    """
    stack = []
    
    for token in expression:
        if token.isdigit():
            stack.append(int(token))
        else:
            b = stack.pop()
            a = stack.pop()
            if token == '+': stack.append(a + b)
            elif token == '-': stack.append(a - b)
            elif token == '*': stack.append(a * b)
            elif token == '/': stack.append(int(a / b))  # Integer division
            elif token == '^': stack.append(a ** b)
    
    return stack[0]
\end{minted}

% Infix to Prefix Conversion
\noindent\textbf{Problem: Infix to Prefix Conversion}
\begin{minted}[
bgcolor=bgcolor7,
frame=lines,
framesep=5mm,
rulecolor=\color{black},
linenos,
numbersep=5pt,
fontsize=\normalsize
]{python}
def infix_to_prefix(infix):
    """
    Converts infix expression to prefix using modified Shunting Yard.
    Time Complexity: O(n), Space Complexity: O(n)
    """
    infix = infix[::-1]  # Reverse infix
    # Swap parentheses
    infix = ''.join([')' if c == '(' else '(' if c == ')' else c for c in infix])
    
    postfix = infix_to_postfix(infix)  # Use our previous function
    return postfix[::-1]  # Reverse postfix to get prefix
\end{minted}

% Evaluation of Prefix Expression
\noindent\textbf{Problem: Evaluation of Prefix Expression}
\begin{minted}[
bgcolor=bgcolor5,
frame=lines,
framesep=5mm,
rulecolor=\color{black},
linenos,
numbersep=5pt,
fontsize=\normalsize
]{python}
def eval_prefix(expression):
    """
    Evaluates prefix expression using stack.
    Time Complexity: O(n), Space Complexity: O(n)
    """
    stack = []
    # Process from right to left
    for token in reversed(expression):
        if token.isdigit():
            stack.append(int(token))
        else:
            a = stack.pop()
            b = stack.pop()
            if token == '+': stack.append(a + b)
            elif token == '-': stack.append(a - b)
            elif token == '*': stack.append(a * b)
            elif token == '/': stack.append(int(a / b))
            elif token == '^': stack.append(a ** b)
    
    return stack[0]
\end{minted}

% Largest Rectangle with All 1's (Binary Matrix)
\noindent\textbf{Problem: Largest Rectangle with All 1's (Binary Matrix)}
\begin{minted}[
bgcolor=bgcolor3,
frame=lines,
framesep=5mm,
rulecolor=\color{black},
linenos,
numbersep=5pt,
fontsize=\normalsize
]{python}
def maximal_rectangle(matrix):
    """
    Finds largest rectangle containing only 1's in binary matrix.
    Time Complexity: O(rows * cols), Space Complexity: O(cols)
    """
    if not matrix:
        return 0
    
    rows, cols = len(matrix), len(matrix[0])
    heights = [0] * cols
    max_area = 0
    
    for i in range(rows):
        # Update heights for current row
        for j in range(cols):
            heights[j] = heights[j] + 1 if matrix[i][j] == '1' else 0
        
        # Calculate max area in histogram
        stack = []
        for j in range(cols + 1):
            while stack and (j == cols or heights[j] < heights[stack[-1]]):
                h = heights[stack.pop()]
                w = j - stack[-1] - 1 if stack else j
                max_area = max(max_area, h * w)
            stack.append(j)
    
    return max_area
\end{minted}

% \end{document}
