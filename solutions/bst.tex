% \documentclass[a4paper,10pt]{article}
% \usepackage[utf8]{inputenc}
% \usepackage{geometry}
% \usepackage[table]{xcolor}
% \usepackage{colortbl}
% \usepackage{color,soul}
% \geometry{margin=0.8in}
% \usepackage{xcolor}
% \usepackage{tikz}
% \usepackage{minted}
% \definecolor{bgcolor}{rgb}{0.8, 0.9, 0.5} % 
% \definecolor{bgcolor1}{rgb}{0.95, 0.95, 0.95} % Light Gray
% \definecolor{bgcolor2}{rgb}{0.85, 0.92, 1.0}  % Soft Blue
% \definecolor{bgcolor3}{rgb}{0.9, 0.85, 1.0}   % Light Purple
% \definecolor{bgcolor4}{rgb}{0.95, 0.88, 0.76} % Warm Beige
% \definecolor{bgcolor5}{rgb}{0.8, 0.95, 0.8}   % Gentle Green
% \definecolor{bgcolor6}{rgb}{1.0, 0.87, 0.87}  % Pastel Red
% \definecolor{bgcolor7}{rgb}{0.86, 0.93, 0.83} % Mint Green
% \definecolor{bgcolor8}{rgb}{0.98, 0.85, 0.94} % Soft Pink
% \definecolor{bgcolor9}{rgb}{0.87, 0.94, 0.98} % Sky Blue
% \definecolor{bgcolor10}{rgb}{0.96, 0.96, 0.82} % Pale Yellow
% 
% % Define a custom background colo
% \begin{document}
\section*{BST Solved Problems }
\noindent\textbf{Problem: Deletion in BST}
\begin{minted}[
bgcolor=bgcolor6,
frame=lines,
framesep=5mm,
rulecolor=\color{black},
linenos,
numbersep=5pt,
fontsize=\normalsize
]{python}
def deleteNode(root: TreeNode, key: int) -> TreeNode:
    """
    Deletes node with given key in BST.
    Time Complexity: O(h)     Space Complexity: O(h)
    """
    if not root:
        return None
    if key < root.val:
        root.left = deleteNode(root.left, key)
    elif key > root.val:
        root.right = deleteNode(root.right, key)
    else:
        if not root.left:
            return root.right
        elif not root.right:
            return root.left
        # Node with two children: get inorder successor
        temp = root.right
        while temp.left:
            temp = temp.left
        root.val = temp.val
        root.right = deleteNode(root.right, temp.val)
    return root
\end{minted}


\noindent\textbf{Problem: Floor/Ceil in BST}
\begin{minted}[
bgcolor=bgcolor4,
frame=lines,
framesep=5mm,
rulecolor=\color{black},
linenos,
numbersep=5pt,
fontsize=\normalsize
]{python}
def floor_ceil(root: TreeNode, key: int) -> tuple:
    """
    Finds floor and ceil values for given key in BST.
    Time Complexity: O(h)     Space Complexity: O(1)
    """
    floor = -10**9
    ceil = 10**9
    curr = root
    
    while curr:
        if curr.val == key:
            return (curr.val, curr.val)
        elif curr.val < key:
            floor = max(floor, curr.val)
            curr = curr.right
        else:
            ceil = min(ceil, curr.val)
            curr = curr.left
    return (floor if floor != -10**9 else None, ceil if ceil != 10**9 else None)
\end{minted}
\noindent\textbf{Inorder Successor and Predecessor in BST}
\begin{minted}[
bgcolor=bgcolor6,
frame=lines,
framesep=5mm,
rulecolor=\color{black},
linenos,
numbersep=5pt,
fontsize=\normalsize
]{python}
class TreeNode:
    def __init__(self, key, left=None, right=None, parent=None):
        self.key = key
        self.left = left
        self.right = right
        self.parent = parent
def inorder_successor(node):
    # Case 1: Right subtree exists
    if node.right:
        cur = node.right
        while cur.left:
            cur = cur.left
        return cur
    # Case 2: No right subtree—climb up
    cur = node
    while cur.parent and cur.parent.right is cur:
        cur = cur.parent
    return cur.parent

def inorder_predecessor(node):
    if node.left:
        cur = node.left
        while cur.right:
            cur = cur.right
        return cur
    cur = node
    while cur.parent and cur.parent.left is cur:
        cur = cur.parent
    return cur.parent
\end{minted}

\noindent\textbf{Problem: AVL Tree Rotations}
\begin{minted}[
bgcolor=bgcolor7,
frame=lines,
framesep=5mm,
rulecolor=\color{black},
linenos,
numbersep=5pt,
fontsize=\normalsize
]{python}
class AVLNode:
    def __init__(self, key):
        self.key = key
        self.left = None
        self.right = None
        self.height = 1

class AVLTree:
    def insert(self, root, key):
        """Inserts a key and balances tree using rotations"""
        # Standard BST insertion
        if not root:
            return AVLNode(key)
        if key < root.key:
            root.left = self.insert(root.left, key)
        else:
            root.right = self.insert(root.right, key)
        
        # Update height
        root.height = 1 + max(self.get_height(root.left), 
                             self.get_height(root.right))
        
        # Balance factor
        balance = self.get_balance(root)
        
        # Left Left Case
        if balance > 1 and key < root.left.key:
            return self.right_rotate(root)
        # Right Right Case
        if balance < -1 and key > root.right.key:
            return self.left_rotate(root)
        # Left Right Case
        if balance > 1 and key > root.left.key:
            root.left = self.left_rotate(root.left)
            return self.right_rotate(root)
        # Right Left Case
        if balance < -1 and key < root.right.key:
            root.right = self.right_rotate(root.right)
            return self.left_rotate(root)
        return root

    def left_rotate(self, z):
        """Left rotation"""
        y = z.right
        T2 = y.left
        
        # Perform rotation
        y.left = z
        z.right = T2
        
        # Update heights
        z.height = 1 + max(self.get_height(z.left), 
                          self.get_height(z.right))
        y.height = 1 + max(self.get_height(y.left), 
                          self.get_height(y.right))
        return y

    def right_rotate(self, z):
        """Right rotation"""
        y = z.left
        T3 = y.right
        
        # Perform rotation
        y.right = z
        z.left = T3
        
        # Update heights
        z.height = 1 + max(self.get_height(z.left), 
                          self.get_height(z.right))
        y.height = 1 + max(self.get_height(y.left), 
                          self.get_height(y.right))
        return y

    def get_height(self, root):
        if not root:
            return 0
        return root.height

    def get_balance(self, root):
        if not root:
            return 0
        return self.get_height(root.left) - self.get_height(root.right)
\end{minted}

\noindent\textbf{Problem: Fix BST with Two Nodes Swapped}
\begin{minted}[
bgcolor=bgcolor2,
frame=lines,
framesep=5mm,
rulecolor=\color{black},
linenos,
numbersep=5pt,
fontsize=\normalsize
]{python}
def recover_bst(root: TreeNode) -> None:
    """
    Fixes BST where two nodes are swapped.
    Time Complexity: O(n)     Space Complexity: O(h)
    """
    stack = []
    curr = root
    prev = first = middle = last = None
    
    # Morris traversal to find swapped nodes
    while curr:
        if curr.left:
            pre = curr.left
            while pre.right and pre.right != curr:
                pre = pre.right
            if not pre.right:
                pre.right = curr
                curr = curr.left
            else:
                pre.right = None
                # Process current node
                if prev and prev.val > curr.val:
                    if not first:
                        first = prev
                        middle = curr
                    else:
                        last = curr
                prev = curr
                curr = curr.right
        else:
            if prev and prev.val > curr.val:
                if not first:
                    first = prev
                    middle = curr
                else:
                    last = curr
            prev = curr
            curr = curr.right
    
    # Swap nodes
    if last:
        first.val, last.val = last.val, first.val
    else:
        first.val, middle.val = middle.val, first.val
\end{minted}

\noindent\textbf{Problem: Pair Sum with Given BST (O(logn) space)}
\begin{minted}[
bgcolor=bgcolor5,
frame=lines,
framesep=5mm,
rulecolor=\color{black},
linenos,
numbersep=5pt,
fontsize=\normalsize
]{python}
class BSTIterator:
    def __init__(self, root, reverse=False):
        self.stack = []
        self.reverse = reverse
        self.push_all(root)
    
    def push_all(self, node):
        while node:
            self.stack.append(node)
            node = node.right if self.reverse else node.left
    
    def next(self) -> int:
        node = self.stack.pop()
        self.push_all(node.left if self.reverse else node.right)
        return node.val
    
    def has_next(self) -> bool:
        return bool(self.stack)

def pair_sum(root: TreeNode, k: int) -> bool:
    """
    Checks if two nodes sum to k using BST iterators.
    Time Complexity: O(n)     Space Complexity: O(logn)
    """
    left_iter = BSTIterator(root)
    right_iter = BSTIterator(root, True)
    
    left = left_iter.next() if left_iter.has_next() else None
    right = right_iter.next() if right_iter.has_next() else None
    
    while left < right:
        total = left + right
        if total == k:
            return True
        elif total < k:
            left = left_iter.next() if left_iter.has_next() else None
        else:
            right = right_iter.next() if right_iter.has_next() else None
    return False
\end{minted}

\noindent\textbf{Problem: Largest BST in Binary Tree}
\begin{minted}[
bgcolor=bgcolor,
frame=lines,
framesep=5mm,
rulecolor=\color{black},
linenos,
numbersep=5pt,
fontsize=\normalsize
]{python}
def largest_bst(root: TreeNode) -> int:
    """
    Finds size of largest BST subtree.
    Time Complexity: O(n)     Space Complexity: O(h)
    """
    max_size = 0
    
    def dfs(node):
        nonlocal max_size
        if not node:
            return 0, float('inf'), float('-inf')
        
        left_size, left_min, left_max = dfs(node.left)
        right_size, right_min, right_max = dfs(node.right)
        
        # Check BST property
        if left_max < node.val < right_min:
            size = 1 + left_size + right_size
            max_size = max(max_size, size)
            return size, min(left_min, node.val), max(right_max, node.val)
        
        # Return invalid BST
        return 0, float('-inf'), float('inf')
    
    dfs(root)
    return max_size
\end{minted}



\noindent\textbf{Problem: Find Kth Smallest in BST}
\begin{minted}[
bgcolor=bgcolor3,
frame=lines,
framesep=5mm,
rulecolor=\color{black},
linenos,
numbersep=5pt,
fontsize=\normalsize
]{python}
def kthSmallest(root: TreeNode, k: int) -> int:
    """
    Finds kth smallest element in BST (iterative inorder).
    For Kth largest element in BST (iterative inorder in reverse : right -> node -> left)
    Time Complexity: O(h + k)     Space Complexity: O(h)
    """
    stack = []
    curr = root
    while stack or curr:
        while curr:
            stack.append(curr)
            curr = curr.left
        curr = stack.pop()
        k -= 1
        if k == 0:
            return curr.val
        curr = curr.right
\end{minted}
\noindent\textbf{Problem: Check for BST}
\begin{minted}[
bgcolor=bgcolor4,
frame=lines,
framesep=5mm,
rulecolor=\color{black},
linenos,
numbersep=5pt,
fontsize=\normalsize
]{python}
def isValidBST(root: TreeNode) -> bool:
    """
    Validates if binary tree is a BST using inorder traversal.
    Time Complexity: O(n)     Space Complexity: O(h)
    """
    prev = None
    
    def inorder(node):
        nonlocal prev
        if not node:
            return True
        if not inorder(node.left):
            return False
        if prev and node.val <= prev.val:
            return False
        prev = node
        return inorder(node.right)
    
    return inorder(root)
\end{minted}
\noindent\textbf{Problem: Finding LCA in BST}
\begin{minted}[
bgcolor=bgcolor10,
frame=lines,
framesep=5mm,
rulecolor=\color{black},
linenos,
numbersep=5pt,
fontsize=\normalsize
]{python}
def lowestCommonAncestorBST(root: TreeNode, p: TreeNode, q: TreeNode) -> TreeNode:
    """
    Finds LCA in Binary Search Tree (iterative approach).
    Time Complexity: O(h)     Space Complexity: O(1)
    """
    while root:
        if p.val < root.val and q.val < root.val:
            root = root.left
        elif p.val > root.val and q.val > root.val:
            root = root.right
        else:
            return root
\end{minted}

% \end{document}
