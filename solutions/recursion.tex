% \documentclass[a4paper,10pt]{article}
% \usepackage[utf8]{inputenc}
% \usepackage{geometry}
% \usepackage[table]{xcolor}
% \usepackage{colortbl}
% \usepackage{color,soul}
% \geometry{margin=0.8in}
% \usepackage{xcolor}
% \usepackage{tikz}
% \usepackage{minted}
% \definecolor{bgcolor}{rgb}{0.8, 0.9, 0.5} % 
% \definecolor{bgcolor1}{rgb}{0.95, 0.95, 0.95} % Light Gray
% \definecolor{bgcolor2}{rgb}{0.85, 0.92, 1.0}  % Soft Blue
% \definecolor{bgcolor3}{rgb}{0.9, 0.85, 1.0}   % Light Purple
% \definecolor{bgcolor4}{rgb}{0.95, 0.88, 0.76} % Warm Beige
% \definecolor{bgcolor5}{rgb}{0.8, 0.95, 0.8}   % Gentle Green
% \definecolor{bgcolor6}{rgb}{1.0, 0.87, 0.87}  % Pastel Red
% \definecolor{bgcolor7}{rgb}{0.86, 0.93, 0.83} % Mint Green
% \definecolor{bgcolor8}{rgb}{0.98, 0.85, 0.94} % Soft Pink
% \definecolor{bgcolor9}{rgb}{0.87, 0.94, 0.98} % Sky Blue
% \definecolor{bgcolor10}{rgb}{0.96, 0.96, 0.82} % Pale Yellow
% 
% \begin{document}
% Palindrome Check (Recursion)
\noindent\textbf{Problem: Palindrome Check (Recursion)}
\begin{minted}[
bgcolor=bgcolor3,
frame=lines,
framesep=5mm,
rulecolor=\color{black},
linenos,
numbersep=5pt,
fontsize=\normalsize
]{python}
def is_palindrome(s: str, start: int, end: int) -> bool:
    """
    Checks if substring s[start:end+1] is palindrome using recursion.
    Time Complexity: O(n), Space Complexity: O(n) for recursion stack.
    """
    if start >= end:
        return True
    if s[start] != s[end]:
        return False
    return is_palindrome(s, start+1, end-1)
\end{minted}

% Count Set Bits from 1 to N
\noindent\textbf{Problem: Count Set Bits from 1 to N}
\begin{minted}[
bgcolor=bgcolor7,
frame=lines,
framesep=5mm,
rulecolor=\color{black},
linenos,
numbersep=5pt,
fontsize=\normalsize
]{python}
def count_set_bits(n: int) -> int:
    """
    Counts total set bits in all numbers from 1 to n using bit patterns.
    Time Complexity: O(log n), Space Complexity: O(1)
    """
    if n <= 0:
        return 0
        
    # Find highest power of 2 <= n
    x = 0
    while (1 << x) <= n:
        x += 1
    x -= 1
    
    # Calculate total set bits
    bits_till_2x = x * (1 << (x-1))
    msb_after_2x = n - (1 << x) + 1
    rest = n - (1 << x)
    return bits_till_2x + msb_after_2x + count_set_bits(rest)
\end{minted}

% Rope Cutting Problem
\noindent\textbf{Problem: Rope Cutting Problem}
\begin{minted}[
bgcolor=bgcolor5,
frame=lines,
framesep=5mm,
rulecolor=\color{black},
linenos,
numbersep=5pt,
fontsize=\normalsize
]{python}
def max_rope_cuts(n: int, a: int, b: int, c: int) -> int:
    """
    Maximizes number of rope cuts of lengths a, b, c from rope of length n.
    Time Complexity: O(3^n), Space Complexity: O(n) for recursion stack.
    """
    if n == 0:
        return 0
    if n < 0:
        return -1
        
    res = max(
        max_rope_cuts(n-a, a, b, c),
        max_rope_cuts(n-b, a, b, c),
        max_rope_cuts(n-c, a, b, c)
    )
    
    return res + 1 if res >= 0 else -1
\end{minted}

% Generate Subsets
\noindent\textbf{Problem: Generate Subsets}
\begin{minted}[
bgcolor=bgcolor8,
frame=lines,
framesep=5mm,
rulecolor=\color{black},
linenos,
numbersep=5pt,
fontsize=\normalsize
]{python}
def generate_subsets(nums: List[int]) -> List[List[int]]:
    """
    Generates all subsets of given list using recursion.
    Time Complexity: O(2^n), Space Complexity: O(n) for recursion stack.
    """
    def backtrack(start, path):
        result.append(path[:])
        #   For each choice of next element…
        for i in range(start, len(nums)):
            #    a) INCLUDE nums[i]
            path.append(nums[i])
            backtrack(i+1, path)
            #    b) EXCLUDE nums[i] (undo include)
            path.pop()
    ##############################ALTERNATE###########################
    def backtrack(idx, path):
        if idx == len(nums):
            result.append(path[:])
            return
        # exclude nums[idx]
        helper(idx+1, path)
        # include nums[idx]
        path.append(nums[idx])
        helper(idx+1, path)
        path.pop()

            
    result = []
    backtrack(0, [])
    return result
\end{minted}

% Subset Sum (Recursive)
\noindent\textbf{Problem: Number of Subsets with Given Sum (Recursive)}
\begin{minted}[
bgcolor=bgcolor1,
frame=lines,
framesep=5mm,
rulecolor=\color{black},
linenos,
numbersep=5pt,
fontsize=\normalsize
]{python}
def subset_sum_count(nums: List[int], target: int) -> int:
    """
    Counts number of subsets that sum to target (recursive).
    Time Complexity: O(2^n), Space Complexity: O(n) for recursion stack.
    """
    def helper(i, current_sum):
        if current_sum == target:
            return 1
        if i == len(nums) or current_sum > target:
            return 0
        return helper(i+1, current_sum + nums[i]) + helper(i+1, current_sum)
        
    return helper(0, 0)
\end{minted}

% Tower of Hanoi
\noindent\textbf{Problem: Tower of Hanoi}
\begin{minted}[
bgcolor=bgcolor9,
frame=lines,
framesep=5mm,
rulecolor=\color{black},
linenos,
numbersep=5pt,
fontsize=\normalsize
]{python}
def tower_of_hanoi(n: int, source: str, auxiliary: str, destination: str) -> None:
    """
    Prints steps to solve Tower of Hanoi problem.
    Time Complexity: O(2^n), Space Complexity: O(n) for recursion stack.
    """
    if n == 1:
        print(f"Move disk 1 from {source} to {destination}")
        return
    tower_of_hanoi(n-1, source, destination, auxiliary)
    print(f"Move disk {n} from {source} to {destination}")
    tower_of_hanoi(n-1, auxiliary, source, destination)
\end{minted}

% Josephus Problem
\noindent\textbf{Problem: Josephus Problem}
\begin{minted}[
bgcolor=bgcolor6,
frame=lines,
framesep=5mm,
rulecolor=\color{black},
linenos,
numbersep=5pt,
fontsize=\normalsize
]{python}
def josephus(n: int, k: int) -> int:
    """
    Finds last remaining person in Josephus circle.
    Time Complexity: O(n), Space Complexity: O(n) for recursion stack.
    """
    if n == 1:
        return 1
    return (josephus(n-1, k) + k-1) % n + 1
\end{minted}

% Print All Permutations
\noindent\textbf{Problem: Printing All Permutations}
\begin{minted}[
bgcolor=bgcolor4,
frame=lines,
framesep=5mm,
rulecolor=\color{black},
linenos,
numbersep=5pt,
fontsize=\normalsize
]{python}
def permute(nums: List[int]) -> List[List[int]]:
    """
    Generates all permutations of given list.
    Time Complexity: O(n!), Space Complexity: O(n) for recursion stack.
    """
    def backtrack(start):
        if start == len(nums):
            result.append(nums[:])
            return
        for i in range(start, len(nums)):
            nums[start], nums[i] = nums[i], nums[start]
            backtrack(start+1)
            nums[start], nums[i] = nums[i], nums[start]
            
    result = []
    backtrack(0)
    return result
\end{minted}
% \end{document}
