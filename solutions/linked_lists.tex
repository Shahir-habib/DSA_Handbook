% \documentclass[a4paper,10pt]{article}
% \usepackage[utf8]{inputenc}
% \usepackage{geometry}
% \usepackage[table]{xcolor}
% \usepackage{colortbl}
% \usepackage{color,soul}
% \geometry{margin=0.8in}
% \usepackage{xcolor}
% \usepackage{tikz}
% \usepackage{minted}
% \definecolor{bgcolor}{rgb}{0.8, 0.9, 0.5} % 
% \definecolor{bgcolor1}{rgb}{0.95, 0.95, 0.95} % Light Gray
% \definecolor{bgcolor2}{rgb}{0.85, 0.92, 1.0}  % Soft Blue
% \definecolor{bgcolor3}{rgb}{0.9, 0.85, 1.0}   % Light Purple
% \definecolor{bgcolor4}{rgb}{0.95, 0.88, 0.76} % Warm Beige
% \definecolor{bgcolor5}{rgb}{0.8, 0.95, 0.8}   % Gentle Green
% \definecolor{bgcolor6}{rgb}{1.0, 0.87, 0.87}  % Pastel Red
% \definecolor{bgcolor7}{rgb}{0.86, 0.93, 0.83} % Mint Green
% \definecolor{bgcolor8}{rgb}{0.98, 0.85, 0.94} % Soft Pink
% \definecolor{bgcolor9}{rgb}{0.87, 0.94, 0.98} % Sky Blue
% \definecolor{bgcolor10}{rgb}{0.96, 0.96, 0.82} % Pale Yellow
% 
% \begin{document}

% Insert at Given Position
\noindent\textbf{Problem: Insert at Given Position in Singly Linked List}
\begin{minted}[
bgcolor=bgcolor3,
frame=lines,
framesep=5mm,
rulecolor=\color{black},
linenos,
numbersep=5pt,
fontsize=\normalsize
]{python}
class ListNode:
    def __init__(self, val=0, next=None):
        self.val = val
        self.next = next

def insert_at_position(head: ListNode, pos: int, val: int) -> ListNode:
    """
    Inserts a new node at given position (1-indexed) in a linked list.
    Time Complexity: O(n), Space Complexity: O(1)
    """
    dummy = ListNode(0, head)
    curr = dummy
    
    # Traverse to position-1 node
    for _ in range(pos - 1):
        if not curr:
            return head
        curr = curr.next
    
    # Insert new node
    new_node = ListNode(val)
    new_node.next = curr.next
    curr.next = new_node
    
    return dummy.next
\end{minted}

% Reverse Doubly Linked List
\noindent\textbf{Problem: Reverse a Doubly Linked List}
\begin{minted}[
bgcolor=bgcolor7,
frame=lines,
framesep=5mm,
rulecolor=\color{black},
linenos,
numbersep=5pt,
fontsize=\normalsize
]{python}
class Node:
    def __init__(self, val=0, prev=None, next=None):
        self.val = val
        self.prev = prev
        self.next = next

def reverse_dll(head: Node) -> Node:
    """
    Reverses a doubly linked list in-place.
    Time Complexity: O(n), Space Complexity: O(1)
    """
    current = head
    prev = None
    
    while current:
        # Swap next and prev pointers
        next_node = current.next
        current.next = prev
        current.prev = next_node
        
        # Move pointers forward
        prev = current
        current = next_node
    
    return prev
\end{minted}
% Detect and Remove Loop
\noindent\textbf{Problem: Detect and Remove Loop in Linked List}
\begin{minted}[
bgcolor=bgcolor8,
frame=lines,
framesep=5mm,
rulecolor=\color{black},
linenos,
numbersep=5pt,
fontsize=\normalsize
]{python}
def detect_and_remove_cycle(head: ListNode) -> None:
    """
    Detects and removes cycle in linked list.
    Time Complexity: O(n), Space Complexity: O(1)
    """
    slow = fast = head
    
    # Find meeting point
    while fast and fast.next:
        slow = slow.next
        fast = fast.next.next
        if slow == fast:
            break
    else:
        return  # No cycle
    
    # Find cycle start
    ptr1 = head
    ptr2 = slow
    while ptr1 != ptr2:
        ptr1 = ptr1.next
        ptr2 = ptr2.next
    
    # Find last node of cycle
    while ptr2.next != ptr1:
        ptr2 = ptr2.next
    
    # Break the cycle
    ptr2.next = None
\end{minted}

% Intersection Point
\noindent\textbf{Problem: Find Intersection Point of Two Linked Lists}\\
Here’s the intuition behind the two-pointer “switch heads” trick:

1. Equalizing total travel  
   - Suppose List A is length m, List B is length n, and they intersect after k nodes (so the shared tail is length k).  
   - Pointer p1 walks A then B, so it travels m + n steps.  
   - Pointer p2 walks B then A, so it also travels n + m steps.  
   After those m+n steps they’ve each covered exactly the same total distance—so if there’s an intersection, they’ll land on it at the same time. If there isn’t one, they’ll both hit the end (`None`) together.

2. Step-by-step with an example  
   Let A = [1 → 2 → 3   8 → 9]  
       B = [4 → 5 → 6      8 ]  
   Intersection at value 8.  
   
   p1 path:  
   1 → 2 → 3 → 8 → 9 → None → 4 → 5 → 6 → 8 → …  \\
   p2 path:  
   4 → 5 → 6 → 8 → 9 → None → 1 → 2 → 3 → 8 → …  


3. No‐intersection case  
   If the lists don’t intersect, both pointers still walk exactly m+n nodes and then both become `None`. The loop `while p1 != p2` breaks and you return `None`.

ASCII timeline (indexes = steps):

   Step:   1  2  3  4  5  6  7  8  9 10  
   p1:     1 → 2 → 3 → 8 → 9 → — → 4 → 5 → 6 → 8  
   p2:     4 → 5 → 6 → 8 → 9 → — → 1 → 2 → 3 → 8  

Key takeaway: by swapping heads when you hit the end, you “sync up” the extra length of the longer list so both pointers cover identical distances and therefore meet at the intersection (or both end up `None`).
\begin{minted}[
bgcolor=bgcolor5,
frame=lines,
framesep=5mm,
rulecolor=\color{black},
linenos,
numbersep=5pt,
fontsize=\normalsize
]{python}
def get_intersection_node(headA: ListNode, headB: ListNode) -> ListNode:
    """
    Finds intersection node of two linked lists.
    Time Complexity: O(m+n), Space Complexity: O(1)
    """
    p1, p2 = headA, headB
    while p1 != p2:
        p1 = p1.next if p1 else headB
        p2 = p2.next if p2 else headA
        # After at most m+n steps, they either meet at the intersection node
        # or both become None (no intersection)

    # Return the intersection node or None
    return p1
\end{minted}

% Middle of Linked List
\noindent\textbf{Problem: Middle of Linked List}
\begin{minted}[
bgcolor=bgcolor1,
frame=lines,
framesep=5mm,
rulecolor=\color{black},
linenos,
numbersep=5pt,
fontsize=\normalsize
]{python}
def middle_node(head: ListNode) -> ListNode:
    """
    Finds middle node of linked list using slow-fast pointers.
    Time Complexity: O(n), Space Complexity: O(1)
    """
    slow = fast = head
    while fast and fast.next:
        slow = slow.next
        fast = fast.next.next
    return slow
\end{minted}

% Nth Node from End
\noindent\textbf{Problem: Nth Node from End of Linked List}
\begin{minted}[
bgcolor=bgcolor9,
frame=lines,
framesep=5mm,
rulecolor=\color{black},
linenos,
numbersep=5pt,
fontsize=\normalsize
]{python}
def remove_nth_from_end(head: ListNode, n: int) -> ListNode:
    """
    Removes nth node from end using two pointers.
    Time Complexity: O(n), Space Complexity: O(1)
    """
    dummy = ListNode(0, head)
    fast = slow = dummy
    
    # Advance fast by n+1 steps
    for _ in range(n + 1):
        fast = fast.next
    
    # Move both until fast reaches end
    while fast:
        slow = slow.next
        fast = fast.next
    
    # Remove nth node
    slow.next = slow.next.next
    return dummy.next
\end{minted}

% Reverse in Groups of k
\noindent\textbf{Problem: Reverse Linked List in Groups of k}
\begin{minted}[
bgcolor=bgcolor4,
frame=lines,
framesep=5mm,
rulecolor=\color{black},
linenos,
numbersep=5pt,
fontsize=\normalsize
]{python}
def reverse_k_group(head: ListNode, k: int) -> ListNode:
    """
    Reverses linked list in groups of k nodes.
    Time Complexity: O(n), Space Complexity: O(n/k)
    """
    # Count nodes in current group
    count = 0
    curr = head
    while curr and count < k:
        curr = curr.next
        count += 1
    
    # Base case: not enough nodes
    if count < k:
        return head
    
    # Reverse current group
    prev, curr = None, head
    for _ in range(k):
        nxt = curr.next
        curr.next = prev
        prev = curr
        curr = nxt
    
    # Recursively reverse remaining
    head.next = reverse_k_group(curr, k)
    return prev
\end{minted}

% Delete Node with Only Pointer
\noindent\textbf{Problem: Delete Node with Only Pointer to It}
\begin{minted}[
bgcolor=bgcolor10,
frame=lines,
framesep=5mm,
rulecolor=\color{black},
linenos,
numbersep=5pt,
fontsize=\normalsize
]{python}
def delete_node(node: ListNode) -> None:
    """
    Deletes node given only pointer to that node.
    Time Complexity: O(1), Space Complexity: O(1)
    """
    node.val = node.next.val
    node.next = node.next.next
\end{minted}

% Segregate Even and Odd Nodes
\noindent\textbf{Problem: Segregate Even and Odd Nodes}
\begin{minted}[
bgcolor=bgcolor6,
frame=lines,
framesep=5mm,
rulecolor=\color{black},
linenos,
numbersep=5pt,
fontsize=\normalsize
]{python}
def odd_even_list(head: ListNode) -> ListNode:
    """
    Groups all odd nodes followed by even nodes.
    Time Complexity: O(n), Space Complexity: O(1)
    """
    if not head:
        return None
    
    odd = head
    even = even_head = head.next
    
    while even and even.next:
        odd.next = even.next
        odd = odd.next
        even.next = odd.next
        even = even.next
    
    odd.next = even_head
    return head
#######################################SEGREGATING################################
class ListNode:
    def __init__(self, val=0, next=None):
        self.val = val
        self.next = next

def segregate_by_value(head: ListNode) -> ListNode:
    """
    Rearranges a singly-linked list so that all nodes with odd values
    come before all nodes with even values. Relative order within the
    odd group and even group is preserved.

    Time Complexity: O(n) — single traversal
    Space Complexity: O(1) — only a few pointers
    """
    if not head:
        return None

    # Create two dummy heads to build odd-valued and even-valued lists
    odd_dummy = ListNode(0)
    even_dummy = ListNode(0)

    odd_tail = odd_dummy    # tail pointer for odd list
    even_tail = even_dummy  # tail pointer for even list
    curr = head             # iterator

    while curr:
        nxt = curr.next     # save next, since we may rewire curr.next
        curr.next = None    # detach curr from original chain

        if curr.val & 1:    # odd check: (val % 2 == 1)
            odd_tail.next = curr
            odd_tail = curr
        else:               # even
            even_tail.next = curr
            even_tail = curr

        curr = nxt          # move on

    # concatenate odd-list with even-list
    odd_tail.next = even_dummy.next
    # (even_tail.next is already None)
    # if there were any odd nodes, that's the new head;
    # otherwise, fall back to the even list directly.
    return odd_dummy.next or even_dummy.next

    
\end{minted}

% Pairwise Swap Nodes
\noindent\textbf{Problem: Pairwise Swap Nodes of Linked List}
\begin{minted}[
bgcolor=bgcolor3,
frame=lines,
framesep=5mm,
rulecolor=\color{black},
linenos,
numbersep=5pt,
fontsize=\normalsize
]{python}
def swap_pairs(head: ListNode) -> ListNode:
    """
    Swaps every two adjacent nodes.
    Time Complexity: O(n), Space Complexity: O(1)
    """
    dummy = ListNode(0, head)
    prev = dummy
    
    while head and head.next:
        # Nodes to swap
        first = head
        second = head.next
        
        # Swap nodes
        prev.next = second
        first.next = second.next
        second.next = first
        
        # Move pointers
        prev = first
        head = first.next
    
    return dummy.next
\end{minted}

% Clone with Random Pointer
\noindent\textbf{Problem: Clone a Linked List with Random Pointer}
\begin{minted}[
bgcolor=bgcolor7,
frame=lines,
framesep=5mm,
rulecolor=\color{black},
linenos,
numbersep=5pt,
fontsize=\normalsize
]{python}
class Node:
    def __init__(self, x: int, next: 'Node' = None, random: 'Node' = None):
        self.val = x
        self.next = next
        self.random = random
def copy_random_list(head: 'Node') -> 'Node':
    """
    Deep copies linked list with random pointers using O(n) space.
    Time Complexity: O(n), Space Complexity: O(n)
    """
    if not head:
        return None
    
    # Create mapping: original -> clone
    mapping = {}
    curr = head
    while curr:
        mapping[curr] = Node(curr.val)
        curr = curr.next
    
    # Set next and random pointers
    curr = head
    while curr:
        clone = mapping[curr]
        clone.next = mapping.get(curr.next)
        clone.random = mapping.get(curr.random)
        curr = curr.next
    
    return mapping[head]

~~~~~~~~~~~~~~~~~~~~~~~~~~~~~~~~~~~~~WEAVING~~~~~~~~~~~~~~~~~~~~~~~~~~~~~~~~~~~~~~
    def copy_random_list(head: 'Node') -> 'Node':
    """
    Deep-copy a linked list where each node has a `next` and a `random` pointer.
    Achieves O(n) time and O(1) extra space by weaving the copied nodes
    into the original list, then unweaving.
    """
    if not head:
        return None

    # 1) Weave copy nodes into original list
    #    For each original node O, insert its copy C right after it:
    #    O -> C -> O.next_original
    curr = head
    while curr:
        copy = Node(curr.val)
        copy.next = curr.next
        curr.next = copy
        curr = copy.next

    # 2) Assign random pointers for each copy node
    #    If original O.random points to R, then O.next (the copy)
    #    should have random = R.next (the copy of R).
    curr = head
    while curr:
        copy = curr.next
        copy.random = curr.random.next if curr.random else None
        curr = copy.next

    # 3) Unweave the two lists: restore originals, extract copies
    orig = head
    copy_head = head.next
    while orig:
        copy = orig.next
        orig.next = copy.next              # restore original next
        copy.next = copy.next.next if copy.next else None
        orig = orig.next

    return copy_head
\end{minted}

% LRU Cache Design
\noindent\textbf{Problem: LRU Cache Design}
\begin{minted}[
bgcolor=bgcolor2,
frame=lines,
framesep=5mm,
rulecolor=\color{black},
linenos,
numbersep=5pt,
fontsize=\normalsize
]{python}
class ListNode:
    def __init__(self, key=0, val=0, prev=None, next=None):
        self.key = key
        self.val = val
        self.prev = prev
        self.next = next

class LRUCache:
    """
    LRU Cache implementation using doubly linked list and hashmap.
    Time Complexity: O(1) per operation, Space Complexity: O(capacity)
    """
    def __init__(self, capacity: int):
        self.cap = capacity
        self.cache = {}
        self.head = ListNode()
        self.tail = ListNode()
        self.head.next = self.tail
        self.tail.prev = self.head

    def _add_node(self, node):
        # Add after head
        node.prev = self.head
        node.next = self.head.next
        self.head.next.prev = node
        self.head.next = node

    def _remove_node(self, node):
        # Remove from any position
        prev = node.prev
        nxt = node.next
        prev.next = nxt
        nxt.prev = prev

    def _move_to_front(self, node):
        self._remove_node(node)
        self._add_node(node)

    def get(self, key: int) -> int:
        if key in self.cache:
            node = self.cache[key]
            self._move_to_front(node)
            return node.val
        return -1

    def put(self, key: int, value: int) -> None:
        if key in self.cache:
            node = self.cache[key]
            node.val = value
            self._move_to_front(node)
        else:
            if len(self.cache) == self.cap:
                # Evict LRU
                lru = self.tail.prev
                self._remove_node(lru)
                del self.cache[lru.key]
            
            # Add new node
            new_node = ListNode(key, value)
            self.cache[key] = new_node
            self._add_node(new_node)
\end{minted}

% Merge Two Sorted Lists
\noindent\textbf{Problem: Merge Two Sorted Linked Lists}
\begin{minted}[
bgcolor=bgcolor8,
frame=lines,
framesep=5mm,
rulecolor=\color{black},
linenos,
numbersep=5pt,
fontsize=\normalsize
]{python}
def merge_two_lists(l1: ListNode, l2: ListNode) -> ListNode:
    """
    Merges two sorted linked lists iteratively.
    Time Complexity: O(m+n), Space Complexity: O(1)
    """
    dummy = ListNode()
    curr = dummy
    
    while l1 and l2:
        if l1.val <= l2.val:
            curr.next = l1
            l1 = l1.next
        else:
            curr.next = l2
            l2 = l2.next
        curr = curr.next
    
    curr.next = l1 if l1 else l2
    return dummy.next
\end{minted}

% Palindrome Linked List
\noindent\textbf{Problem: Palindrome Linked List}
\begin{minted}[
bgcolor=bgcolor5,
frame=lines,
framesep=5mm,
rulecolor=\color{black},
linenos,
numbersep=5pt,
fontsize=\normalsize
]{python}
def is_palindrome(head: ListNode) -> bool:
    """
    Checks if linked list is palindrome using O(1) space.
    Time Complexity: O(n), Space Complexity: O(1)
    """
    # Find middle
    slow = fast = head
    while fast and fast.next:
        slow = slow.next
        fast = fast.next.next
    
    # Reverse second half
    prev = None
    while slow:
        nxt = slow.next
        slow.next = prev
        prev = slow
        slow = nxt
    
    # Compare halves
    left, right = head, prev
    while right:
        if left.val != right.val:
            return False
        left = left.next
        right = right.next
    return True
\end{minted}

% Add One to Linked List
\noindent\textbf{Problem: Add One to Linked List}
\begin{minted}[
bgcolor=bgcolor1,
frame=lines,
framesep=5mm,
rulecolor=\color{black},
linenos,
numbersep=5pt,
fontsize=\normalsize
]{python}
def add_one(head: ListNode) -> ListNode:
    """
    Adds 1 to number represented as linked list (MSD first).
    Time Complexity: O(n), Space Complexity: O(1)
    """
    # Reverse list (to LSD first)
    prev, curr = None, head
    while curr:
        nxt = curr.next
        curr.next = prev
        prev = curr
        curr = nxt
    
    # Add one with carry
    curr = prev
    carry = 1
    while curr and carry:
        total = curr.val + carry
        curr.val = total % 10
        carry = total // 10
        if not curr.next and carry:
            curr.next = ListNode(0)
        curr = curr.next
    
    # Reverse back
    new_head = None
    while prev:
        nxt = prev.next
        prev.next = new_head
        new_head = prev
        prev = nxt
    
    return new_head
##############################################################
def addTwoNumbers(l1, l2):
    # Step 1: Reverse both lists
    l1 = reverse_list(l1)
    l2 = reverse_list(l2)

    carry = 0
    dummy = ListNode(0)
    current = dummy

    # Step 2: Add digits with carry
    while l1 or l2 or carry:
        sum_val = carry
        if l1:
            sum_val += l1.val
            l1 = l1.next
        if l2:
            sum_val += l2.val
            l2 = l2.next

        carry = sum_val // 10
        current.next = ListNode(sum_val % 10)
        current = current.next

    # Step 3: Reverse the result to restore forward order
    return reverse_list(dummy.next)
\end{minted}

% Rotate Linked List by K
\noindent\textbf{Problem: Rotate Linked List by K}
\begin{minted}[
bgcolor=bgcolor9,
frame=lines,
framesep=5mm,
rulecolor=\color{black},
linenos,
numbersep=5pt,
fontsize=\normalsize
]{python}
def rotate_right(head: ListNode, k: int) -> ListNode:
    """
    Rotates linked list to the right by k places.
    Time Complexity: O(n), Space Complexity: O(1)
    """
    if not head or not head.next or k == 0:
        return head
    
    # Get length and find tail
    n = 1
    tail = head
    while tail.next:
        tail = tail.next
        n += 1
    
    # Adjust k
    k %= n
    if k == 0:
        return head
    
    # Find new tail (n-k-1 steps from head)
    new_tail = head
    for _ in range(n - k - 1):
        new_tail = new_tail.next
    
    # Reconnect nodes
    new_head = new_tail.next
    new_tail.next = None
    tail.next = head
    
    return new_head
\end{minted}

% Flatten a Linked List
\noindent\textbf{Problem: Flatten a Linked List}
\begin{minted}[
bgcolor=bgcolor4,
frame=lines,
framesep=5mm,
rulecolor=\color{black},
linenos,
numbersep=5pt,
fontsize=\normalsize
]{python}
class Node:
    def __init__(self, val=0, next=None, child=None):
        self.val = val
        self.next = next
        self.child = child

def flatten(head: 'Node') -> 'Node':
    """
    Flattens multilevel doubly linked list depth-first.
    Time Complexity: O(n), Space Complexity: O(1)
    """
    curr = head
    while curr:
        # If node has child, merge child list
        if curr.child:
            nxt = curr.next
            child_head = flatten(curr.child)
            curr.child = None
            
            # Connect current to child head
            curr.next = child_head
            child_head.prev = curr
            
            # Find tail of child list
            tail = child_head
            while tail.next:
                tail = tail.next
                
            # Connect child tail to next node
            tail.next = nxt
            if nxt:
                nxt.prev = tail
                
        curr = curr.next
        
    return head
\end{minted}

% LFU Cache Design
\noindent\textbf{Problem: LFU Cache Design}
\begin{minted}[
bgcolor=bgcolor10,
frame=lines,
framesep=5mm,
rulecolor=\color{black},
linenos,
numbersep=5pt,
fontsize=\normalsize
]{python}
class Node:
    def __init__(self, key=0, value=0, freq=0, prev=None, next=None):
        self.key = key
        self.value = value
        self.freq = freq
        self.prev = prev
        self.next = next

class DoublyLinkedList:
    def __init__(self):
        self.head = Node()
        self.tail = Node()
        self.head.next = self.tail
        self.tail.prev = self.head
        self.size = 0

    def add_node(self, node):
        node.next = self.head.next
        node.prev = self.head
        self.head.next.prev = node
        self.head.next = node
        self.size += 1

    def remove_node(self, node):
        node.prev.next = node.next
        node.next.prev = node.prev
        self.size -= 1

    def remove_last(self):
        last = self.tail.prev
        self.remove_node(last)
        return last

class LFUCache:
    """
    LFU Cache implementation using frequency dictionaries and doubly linked lists.
    Time Complexity: O(1) per operation, Space Complexity: O(capacity)
    """
    def __init__(self, capacity: int):
        self.cap = capacity
        self.min_freq = 0
        self.key_map = {}  # key:node
        self.freq_map = defaultdict(DoublyLinkedList)  # freq:list

    def _update_node(self, node):
        freq = node.freq
        self.freq_map[freq].remove_node(node)
        
        if self.min_freq == freq and self.freq_map[freq].size == 0:
            self.min_freq += 1
        
        node.freq += 1
        self.freq_map[node.freq].add_node(node)

    def get(self, key: int) -> int:
        if key not in self.key_map:
            return -1
        
        node = self.key_map[key]
        self._update_node(node)
        return node.value

    def put(self, key: int, value: int) -> None:
        if self.cap == 0:
            return
            
        if key in self.key_map:
            node = self.key_map[key]
            node.value = value
            self._update_node(node)
        else:
            if len(self.key_map) == self.cap:
                # Evict LFU (and LRU if tie)
                node = self.freq_map[self.min_freq].remove_last()
                del self.key_map[node.key]
            
            # Add new node
            new_node = Node(key, value, 1)
            self.key_map[key] = new_node
            self.freq_map[1].add_node(new_node)
            self.min_freq = 1
\end{minted}

% \end{document}
